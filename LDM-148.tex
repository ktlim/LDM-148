\documentclass[DM,toc,lsstdraft]{lsstdoc}
\usepackage{xtab}
\usepackage{enumitem}

\title{Data Management System Design}
\setDocRef{LDM-148}
\author{
  K.-T.~Lim,
  J.~Bosch,
  G.~Dubois-Felsmann,
  T.~Jenness,
  J.~Kantor,
  W.~O'Mullane,
  D.~Petravick,
  and
  the DM Leadership Team}
\setDocCurator{Kian-Tat Lim}
\setDocUpstreamLocation{\url{https://github.com/lsst/LDM-148}}

\date{\today}
\setDocAbstract{%
  The LSST Data Management System (DMS) is a set of services
  employing a variety of software components running on
  computational and networking infrastructure that combine to
  deliver science data products to the observatory's users and
  support observatory operations.  This document describes the
  components, their service instances, and their deployment
  environments as well as the interfaces among them, the rest
  of the LSST system, and the outside world.
}
\setDocChangeRecord{%
  \addtohist{2}{2011-08-09}{Copied from MREFC Proposal into LDM-148 handle, reformatted}{Robert McKercher}
  \addtohist{3}{2011-08-15}{Updated for Preliminary Design Review}{Tim Axelrod, K-T Lim, Mike Freemon, Jeffrey Kantor}
  \addtohist{4}{2013-10-09}{Updated for Final Design Review}{Mario Juric, K-T Lim, Jeffrey Kantor}
  \addtohist{5.0}{2017-07-04}{Rewritten for Construction and Operations. Approved in \href{https://jira.lsstcorp.org/browse/RFC-358}{RFC-358}.}{K-T Lim}
}

\begin{document}
\maketitle

\section{Introduction}\label{introduction}

The purpose of the LSST Data Management System (DMS) is to deliver science data
products to the observatory's users and to support observatory operations.  The
DMS is a set of services employing a variety of software components running on
computational and networking infrastructure.  The DMS is constructed by the DM
subsystem in the NSF MREFC project; in the Operations era, it is operated by a
combination of the LSST Data Facility, Science Operations, and Observatory
Operations departments.

The data products to be delivered are defined and described in the \textit{Data
Products Definition Document} \citedsp{LSE-163}. These are divided into three
major categories.

One category of data products is generated on a nightly or daily cadence
and comprises raw, processed/calibrated, and difference images as well as alerts
of transient, moving, and variable objects detected from the images,
published within 60 seconds, and recorded in searchable catalogs. These
data products can be considered ``online'', as they are driven primarily
by the observing cadence of the observatory. This category of Prompt data products has
historically been referred to as ``Level 1''.  These products are intended to
enable detection and follow-up of time-sensitive time-domain events.

A second category of data products is generated on an annual cadence and
represents a complete reprocessing of the set of images taken to date to
generate astronomical catalogs containing measurements and
characterization of tens of billions of stars and galaxies with high and
uniform astrometric and photometric accuracy. As part of this
reprocessing, all of the first category of data products is regenerated,
often using more accurate algorithms. This category also includes other
data products such as calibration products and templates that are
generated in an ``offline'' mode, not directly tied to the observing
cadence. This category of Data Release data products has historically been referred to as ``Level 2'',
including the regenerated data products from the first category.

The third category of data products is not generated by the LSST DMS but is
instead generated, created, or imported by science users for their own science
goals. These products derive value from their close association with or
derivation from other LSST data products. The DMS is responsible for providing
facilities, services, and software for their generation and storage.  This
category of User Generated data products has historically been referred to as ``Level 3''.

Data products are delivered to science users through Data Access
Centers (DACs). In addition, streams of near-realtime alerts and planned
observations are provided.  Each LSST data product has associated
metadata providing provenance and quality metrics and tracing it to relevant
calibration information in the archive. The DACs are composed of modest but
significant computational, storage, networking, and other resources intended
for use as a flexible, multi-tenant environment for professional astronomers
with LSST data rights to retrieve, manipulate, and annotate LSST data products
in order to perform scientific discovery and inquiry.

The first section of this document describes how the DMS components work
together to generate and distribute the data products.  The next section
describes how the size of the DMS computing environments was estimated.
Subsequent sections describe the individual components of the DMS in more
detail, including their interfaces with each other, with other LSST subsystems,
and with the outside world.

\section{Summary Concept of Operations}\label{summary-concept-of-operations}

The principal functions of the DMS are to:
\begin{itemize}
	\item Process the incoming stream of images generated by the camera system during observing by archiving raw images, generating transient alerts, and updating difference source and object catalogs.
	\item Periodically (at least annually) process the accumulated survey data to provide a uniform photometric and astrometric calibration, measure the properties of fainter objects, and characterize the time-dependent behavior of objects. The results of such a processing run form a data release (DR), which is a static, self-consistent data set for use in performing scientific analysis of LSST data and publication of the results. All data releases are archived for the entire operational life of the LSST archive.
	\item Periodically create new calibration data products, such as bias frames and flat fields, to be used by the other processing functions.
	\item Make all LSST data available through an interface that utilizes, to the maximum practical extent, community-based standards such as those being developed by the Virtual Observatory (VO) in collaboration with the International Virtual Observatory Alliance (IVOA).  Provide enough processing, storage, and network bandwidth to enable user analysis of the data without petabyte-scale data transfers.
\end{itemize}

The latency requirements for alerts determine several aspects of the DMS design
and overall cost.  An alert is triggered by an unexpected excursion in
brightness of a known object or the appearance of a previously undetected
object such as a supernova or a GRB. The astrophysical time scale of some of
these events may warrant follow-up by other telescopes on short time scales.
These excursions in brightness must be recognized by the pipeline, and the
resulting alert data product sent on its way, within 60 seconds. This drives
the DMS design in the decision to acquire high-bandwidth/high-reliability
long-haul networking from the Summit at Cerro Pachon to the Base in La Serena and from Chile to the U.S. These networks allow the significant computational
resources necessary for promptly processing incoming images to be located in
cost-effective locations: the Base has far fewer limitations on power, cooling,
and rack space capacity than the Summit, and placing the scientific
processing at NCSA allows for far greater flexibility in the allocation of
resources to ensure that deadlines are met. Performing cross-talk correction
on the data in the data acquisition system and parallelizing the alert
processing at the amplifier and CCD levels, where possible, also help to
minimize the latency to alert delivery.

The Data Release processing requires extensive computation, combining
information from all images of an object in order to measure it as
accurately as possible.  A sophisticated workload and workflow management
system and Task Framework are used to divide the processing into
manageable units of work that can be assigned to available resources,
including the two dedicated processing clusters at NCSA and CC-IN2P3.

Calibration data products must be created and updated at cadences in between
the Alert and Data Release periods.  The stability of the system is expected to
require daily, monthly, and annual calibration productions.  The daily
production must be synchronized with the observatory schedule, occurring after
raw calibration frames have been taken but well before science observing is
planned.  This requirement necessitates the inclusion of a service that allows
the Observatory Control System to trigger remote calibration processing at
NCSA.

The DACs are a key component of the DMS, giving the community resources and an
interface to interact with and utilize the LSST data products to perform
science.  An instance of the LSST Science Platform (LSP) is deployed in each
DAC to support science users with its Portal, JupyterLab (notebook), and Web
API aspects.  Substantial compute, storage, and storage bandwidth is devoted to
ensuring that the LSP is responsive and allows for exploration of the vast
LSST data products.

Underlying all of the above is a Data Backbone that provides storage, tracking,
and replication for all LSST data products.  The Data Backbone links all of the
computational environments and the Data Access Centers, acting as the spine that
supports them all.

\section{Sizing}\label{sizing}

A fundamental question is how large the LSST Data Management System must be. To
this end, a complex analytical model has been developed driven by input from
the requirements specifications. Specifications from the science requirements
and other subsystem designs, and the observing strategy, translate directly
into numbers of detected sources and astronomical objects, and ultimately into
required network bandwidths and the size of storage systems. Specific science
requirements of the survey determine the data quality that must be maintained
in the DMS products, which in turn determine the algorithmic requirements and
the computer power necessary to execute them. The relationship of the elements
of this model and their flow-down from systems and DMS requirements is shown in
Figure \ref{fig:sizing-model}. Detailed sizing computations and associated
explanations appear in LSST Documents listed on the Figure.

\begin{figure}
\centering
\includegraphics[width=\textwidth]{images/SizingModel.pdf}
\caption{DMS Infrastructure Sizing and Estimation.}
\label{fig:sizing-model}
% Silently cite documents mentioned in diagram
\nocite{LDM-148,LDM-151,LDM-152,LDM-135,LDM-138,LDM-140,LDM-129}
\nocite{LSE-81,LSE-82,LDM-141,LDM-139,LDM-144,LDM-143}
\nocite{LPM-17,LSE-29,LSE-30,LSE-61,LSE-163}
\nocite{Document-5373,LDM-142,LSE-78}
\end{figure}

Key input parameters include camera characteristics, the expected cadence of
observations, the number of observed stars and galaxies expected per band, the
processing operations per data element, the data transfer rates between and
within processing locations, the ingest and query rates of input and output
data, the alert generation rates, and latency and throughput requirements for
all data products.

Processing requirements were extrapolated from the functional model of
operations, prototype pipelines and algorithms, and existing precursor
pipelines adjusted to LSST scale.  As a part of every Data Release, all data
previously processed are reprocessed with the latest algorithms, calibration
products, and parameters. This causes the processing requirements to increase
with time.  Advances in hardware performance, however, are expected to reduce
the number of nodes needed and the power and cooling devoted to them. This causes some of the performance figures in Table \ref{table:compute-sizing} to reach
a high-water mark during the survey.

\begin{table}
\centering
\caption{DMS Compute Infrastructure Sizing; growth from Survey Year 1 to Year 10; hwm = high-water mark \label{table:compute-sizing}}
\begin{tabular}{|c|c|c|c|}
 \hline
	& & \textit{Archive Site} & \textit{Base Site} \\ \hline
\multirow{4}{*}{Compute} & TeraFLOPS (sustained) & 197 $\rightarrow$ 970 & 30 $\rightarrow$ 62 \\ \cline{2-4}
  & Nodes & 436 $\rightarrow$ 305 (455 hwm) & 56 $\rightarrow$ 17 (59 hwm) \\ \cline{2-4}
  & Cores & 18K $\rightarrow$ 62K & 3K $\rightarrow$ 4K \\ \hline
\multirow{2}{*}{Database} & TeraFLOPS (sustained) & 40 $\rightarrow$ 310 & 55 $\rightarrow$ 306 \\ \cline{2-4}
  & Nodes & 94 $\rightarrow$ 113 (148 hwm) & 108 $\rightarrow$ 98 (133 hwm) \\ \cline{2-4}
\multirow{3}{*}{Facilities} & Floor Space & 826 $\rightarrow$ 744 ft$^2$ (834 hwm) & 278 $\rightarrow$ 195 ft$^2$ (435 hwm) \\ \cline{2-4}
  & Power & 274 $\rightarrow$ 273 kW (309 hwm) & 158 $\rightarrow$ 248 kW (248 hwm) \\ \cline{2-4}
  & Cooling & 0.9 $\rightarrow$ 0.9 mmbtu (1.1 hwm) & 0.5 $\rightarrow$ 0.8 mmbtu (0.8 hwm) \\ \hline
\end{tabular}
\end{table}

Storage and input/output requirements were extrapolated from the data model of
LSST data products, the DMS and precursor database schemas, and
existing database management system overhead factors in precursor
surveys and experiments adjusted to LSST scale. A summary of key numbers is
in Table \ref{table:storage-sizing}.

\begin{table}
\centering
\caption{DMS Storage Infrastructure Sizing; growth from Survey Year 1 to Year 10 \label{table:storage-sizing}}
\begin{tabular}{|c|c|c|c|}
\hline
	& & \textit{Archive Site} & \textit{Base Site} \\ \hline
\multirow{3}{*}{File Storage} & Capacity & 24 $\rightarrow$ 81 PB & \\ \cline{2-4}
  & Drives & 1602 $\rightarrow$ 862 & 597 $\rightarrow$ 249 \\ \cline{2-4}
  & Bandwidth & 493 $\rightarrow$ 714 GB/s (752 hwm) & 223 $\rightarrow$ 231 GB/s (236 hwm) \\ \hline
\multirow{3}{*}{Database} & Capacity & 29 $\rightarrow$ 99 PB & 16 $\rightarrow$ 72 PB \\ \cline{2-4}
  & Drives & 3921 $\rightarrow$ 2288 & 2190 $\rightarrow$ 1642 \\ \cline{2-4}
  & Bandwidth & 1484 $\rightarrow$ 2040 GB/s (2163 hwm) & 829 $\rightarrow$ 1169 GB/s (1615 hwm) \\ \hline
\multirow{3}{*}{Tape Storage} & Capacity & 31 $\rightarrow$ 242 PB & \\ \cline{2-4}
  & Tapes & 2413 $\rightarrow$ 3691 (4117 hwm) & \\ \cline{2-4}
  & Tape Bandwidth & 36 $\rightarrow$ 65 GB/s & \\ \hline
\end{tabular}
\end{table}


Communications requirements were developed and modeled for the data transfers
and user query/response load, extrapolated from existing surveys and adjusted
to LSST scale.  These requirements are illustrated in Figure
\ref{fig:near-real-time-flows} for per-visit transfers.  Peak bandwidths assume
a 3 second budget for Summit to Base image transfer and a 5 second budget for
international image transfer.
The peak alert bandwidth assumes 40K alerts delivered in 15 seconds.

The Summit to Base and Base to NCSA network links have been significantly
over-engineered for four main reasons: first, because the incremental costs of
higher bandwidth once the link has been provisioned at all have been small;
second, to allow key functions, such as interfacing with the Camera Data System
or performing image analysis and measurement to generate alerts, to be
performed in appropriate locations; third, to increase reliability of the
system; and fourth, to simplify certain components such as the Forwarders and
Archivers that interface with or depend on the networks.  A high-speed Base to
NCSA link also enables Data Releases to be transferred south to the Chilean DAC
over the network rather than through physical media, as originally planned,
decoupling science from maintenance and upgrade activities to a greater extent.

The external bandwidth from NCSA to the community alert brokers is baselined at 10 Gbit/sec.
Given an estimate of the peak alert bandwidth of up to 1.8 Gbit/sec (\citeds{LDM-151}), at least 5 community brokers can be supported with this allocation.
Additional external outbound bandwidth is needed for NCSA to deliver results to science users from the US Data Access Center and to deliver bulk downloads to partners; this has been estimated as 52 Gbit/sec in \citeds{LDM-141}.


\begin{figure}
\centering
\includegraphics[width=\textwidth]{images/NearRealTimeDataFlow.pdf}
\caption{Near Real-Time Data Flows}
\label{fig:near-real-time-flows}
\end{figure}

In all of the above, industry-provided technology trends (\citeds{LDM-143})
were used to extrapolate to the LSST construction and operations phases in
which the technology will be acquired, configured, deployed, operated, and
maintained. A just-in-time acquisition strategy is employed to leverage
favorable cost/performance trends.

The resulting performance and sizing requirements show the DMS to be a
supercomputing­-class system with correspondingly large data input/output and
network bandwidth rates.  Despite this size, technology trends show this to be
well within the anticipated performance of commodity-based systems during the
construction and operations time frame.


\section{Component Overview}\label{component-overview}

The services that make up the DMS are in turn made up of software and
underlying service components, instantiated in a particular
configuration in a particular computing environment to perform a
particular function. Some software components are specific to a service;
others are general-purpose and reused across multiple services. Many
services have only one instance in the production system; others have
several, and all have additional instances in the development and
integration environments for testing purposes.

The DMS services can be considered to consist of four tiers of software
components. The top tier is the LSST Science Platform, which is deployed
in the DACs and other computational environments to provide a user
interface and analysis environment for science users and LSST staff. The
detailed design of this tier is given in \textit{LSST Science Platform} \citedsp{LDM-542}. The next
tier is composed of science ``applications'' software that generates
data products. This software is used to build ``payloads'', sequences of
pipelines, that perform particular data analysis and product generation
tasks. It is also used by science users and staff to analyze the data
products. The detailed design of the components in this tier is given in
\textit{Data Management Science Pipelines Design} \citedsp{LDM-151}. A lower tier is
``middleware'' software components and services that execute the science
application payloads and isolate them from their environment, including
changes to underlying technologies. These components also provide data
access for science users and staff. The detailed design of the
components in this tier is given in \textit{Data Management Middleware Design} \citedsp{LDM-152}.
The bottom tier is ``infrastructure'': hardware, networking,
and low-level software and services that provide a computing
environment. The detailed design of components in this tier is given in
\textit{LSST Data Facility Logical Information Technology and Communications Design} \citedsp{LDM-129} and \textit{Network Design} \citedsp{LSE-78}.

The DMS computing environments reside in four main physical locations:
the Summit Site including the main Observatory and Auxiliary Telescope
buildings on Cerro Pachon, Chile; the Base Facility data center located
at the Base Site in La Serena, Chile; the NCSA Facility data center
at the National Center for Supercomputing Applications (NCSA) in Urbana,
Illinois, USA; and the Satellite Facility at CC-IN2P3 in Lyon,
France. These are linked by high-speed networks to allow rapid data
movement.

The Base Facility includes four enclaves: the Base portion of the Prompt Enclave directly supporting Observatory operations, the Commissioning Cluster, an Archive Enclave holding data products, and the Chilean Data Access Center.

The NCSA Facility also includes four enclaves: the NCSA portion of the Prompt Enclave, the Offline Production Enclave hosting all offline "data release" and calibration activities, another Archive Enclave, and the US Data Access Center.

Additionally, a separate Development and Integration Enclave at NCSA hosts many of the services and tools necessary to build and test the DMS.

These enclaves are distinguished by having different users, operations timescales, interfaces, and often components.

DMS services are assigned to each of these enclaves.  Some enclave hardware may be dedicated; the remainder is allocated from Master Provisioning hardware pools at each Facility.

The service instances that make up the DMS include (with the
enclave they are in noted):
\begin{itemize}
\item
  Archiving (Prompt Base)
\item
  Planned Observation Publication (Prompt Base)
\item
  Prompt Processing Ingest (Prompt Base)
\item
  Observatory Operations Data (Prompt Base)
\item
  Observatory Control System (OCS) Driven Batch (Prompt Base)
\item
  Telemetry Gateway (Prompt Base)
\item
  Prompt Processing (Prompt NCSA)
\item
  Alert Distribution (Prompt NCSA)
\item
  Prompt Quality Control (QC) (Prompt NCSA)
\item
  Batch Production (Offline Production, Satellite Facility)
\item
  Data Release QC (Offline Production)
\item
  LSST Science Platform Commissioning Cluster instance (Commissioning
  Cluster)
\item
  LSST Science Platform Data Access Center instances (DACs)
\item
  Bulk Distribution (DAC)
\item
  Data Backbone (Archive Base and NCSA)
\end{itemize}

The relationships between these services, their deployment enclaves
physical facilities, and science application ``payloads'' can be
visualized in Figure~\ref{fig:deployment}.

\begin{figure}
\centering
\includegraphics[height=0.9\textheight]{images/DMSDeployment.pdf}
\caption{Data Management System Deployment}
\label{fig:deployment}
\end{figure}

Other services necessary to build, test, and operate the DMS include:
\begin{itemize}
\item
  LSST Science Platform Science Validation instance (Development and Integration)
\item
  Developer Services (Development and Integration)
\item
  Managed Database (Infrastructure)
\item
  Batch Computing (Infrastructure)
\item
  Containerized Application Management (Infrastructure)
\item
  IT Security (Infrastructure)
\item
  Identity Management (Infrastructure)
\item
  ITC Provisioning and Management (Infrastructure)
\item
  Service Management/Monitoring (Infrastructure)
\end{itemize}

The common infrastructure services are illustrated in Figure~\ref{fig:commonservices}.

\begin{figure}
\centering
\includegraphics[height=0.9\textheight]{images/DMSCommonServices.pdf}
\caption{Data Management System Common Infrastructure Services}
\label{fig:commonservices}
\end{figure}

The science application software for the Alert Production, daytime
processing, Data Release Production, and calibration processing is built
out of a set of frameworks that accept plugins. In turn, those
frameworks build on middleware that provides portability and
scalability.  The relationships between the packages implementing
these frameworks and plugins and the underlying middleware packages
are shown in Figure~\ref{fig:scipi}.

Key applications software components include:
\begin{itemize}
\item
  Low-level astronomical software primitives and data structures
  (\texttt{afw})
\item
  Image processing and measurement framework with core algorithms
  (\texttt{ip\_*}, \texttt{meas\_*})
\item
  Additional image processing and measurement algorithms
  (\texttt{meas\_extensions\_*})
\item
  High-level algorithms and driver scripts that define pipelines
  (\texttt{pipe\_tasks}, \texttt{pipe\_drivers})
\item
  Camera-specific customizations (\texttt{obs\_*})
\end{itemize}

\begin{figure}
\centering
\includegraphics[width=\textwidth]{images/DM_Application_Software_Arch.png}
\caption{Data Management Science Pipelines Software ``Stack''}
\label{fig:scipi}
\end{figure}

Key middleware components include:
\begin{itemize}
\item
  Data access client (Data Butler) (\texttt{daf\_persistence})
\item
  Parallel distributed database (\texttt{qserv})
\item
  Task framework (\texttt{pex\_*}, \texttt{log}, \texttt{pipe\_base},
  \texttt{ctrl\_pool})
\item
  Campaign management and workflow for production control (\texttt{ctrl\_*})
\end{itemize}

Infrastructure components include:
\begin{itemize}
\item
  Other databases (typically relational)
\item
  Filesystems
\item
  Authentication and authorization (identity management)
\item
  Provisioning and resource management
\item
  Monitoring
\end{itemize}

The relationships between the middleware and infrastructure components
are illustrated in Figure~\ref{fig:mwandinfra}.

\begin{figure}
\centering
\includegraphics[width=\textwidth]{images/MiddlewareInfrastructure.pdf}
\caption{Data Management Middleware and Infrastructure}
\label{fig:mwandinfra}
\end{figure}

\section{Prompt Base Enclave}\label{prompt-base-enclave}

Services located in this enclave are located at the Base solely because
they must interact with the OCS or the Camera Data System (also known as
the Camera DAQ) or both. In several cases, services located here
interact closely with corresponding services in the Prompt NCSA Enclave,
to the point where the Base service cannot function if the
NCSA service is not operational. This reliance has been taken into
account in the fault tolerance strategies used.

The primary goals of the services in this enclave are to transfer data
to appropriate locations, either to NCSA, from NCSA, or to the Data
Backbone.

The services in this enclave and their partners in the Prompt NCSA Enclave
need to run rapidly and reliably. They run at times
(outside office hours) and with latencies that are not amenable to a
human-in-the-loop design. Instead, they are designed to execute
autonomously, often under the control of the OCS, with human oversight,
monitoring, and control only at the highest level.

\subsection{Service Descriptions}\label{base-service-descriptions}

Detailed concepts of operations for each service can be found in
\textit{Concept of Operations for the LSST Production Services} \citedsp{LDM-230}.


\subsubsection{Archiving}\label{archiving}

This component is composed of several Image Archiver service and
Catch-Up Image Archiver instances: one pair each for the LSSTCam, the
ComCam, and the Auxiliary Telescope Spectrograph, all of which may be
operated simultaneously. These capture raw images taken by each camera,
including the wavefront sensors and the guide sensors of the LSSTCam or
ComCam when so configured, retrieving them from their respective Camera
Data System instances.

A Header Generator written by Data Management but operated by the Observatory captures specific sets of metadata
associated with the images, including telemetry values and event
timings, from the OCS publish/subscribe middleware and/or from the EFD.
It formats these into a metadata package that is recorded in the EFD Large File Annex.
The Archiver and Catch-Up Archiver instances retrieve this metadata package and attach it to the captured image pixels.

The image pixels and metadata are then passed to the Observatory Operations Data Service (OODS), which serves as a buffer from which observing-critical data can be retrieved as well as a staging area for ingestion into the permanent archive in the Data Backbone.
The catch-up versions archive into the OODS and Data Backbone any raw
images and metadata that were missed by the primary archiving services
due to network or other outage, retrieving them from the flash storage
in the Camera Data System instances and the EFD.

This component also includes an EFD Transformation service that extracts
all information (including telemetry, events, configurations, and
commands) from the EFD and its large file annex, transforms it into a
form more suitable for querying by image timestamp, and loads it into
the permanently archived ``Transformed EFD'' database in the Data
Backbone.

\subsubsection{Planned Observation Publication}\label{planned-observation-publication}

This service receives telemetry from the OCS describing the next visit location and the telescope scheduler's predictions of its future observations.
It publishes these as an unauthenticated, globally-accessible web service comprising both a web page for human inspection and a web API for usage by automated tools.

\subsubsection{Prompt Processing Ingest}\label{prompt-processing-ingest}

This component is composed of two instances that capture
crosstalk-corrected images from the LSSTCam and ComCam Camera Data
Systems along with selected metadata from the OCS and/or EFD and
transfer them to the Prompt Processing service in the NCSA Level
1 Domain.

There is no Prompt Processing Ingest instance for the auxiliary
telescope spectrograph.

\subsubsection{Observatory Operations Data}\label{obs-ops-data}

This service provides low-latency access to images and metadata for use by Observatory systems and the Commissioning Cluster LSP instance.
It maintains a higher level of service availability than the Data Backbone and so provides a staging area for to-be-archived data prior to ingestion into the Data Backbone.
After images and metadata are ingested, they remain available through the OODS for a policy-configured amount of time.

\subsubsection{OCS Driven Batch}\label{ocs-driven-batch}

This service receives commands from the OCS and invokes a Batch Computing service to execute configured science payloads.
The service can be configured to execute on the Commissioning Cluster at the Base or in the Offline Production Enclave at NCSA.
It is used for modest-latency analysis of images during Commissioning and for processing daily calibration images in normal observing operations.
Images and metadata are taken from the Data Backbone, and results are provided back to the Data Backbone; there is no direct connection from this service to the Camera Data System.
This obviously bounds the minimum latency from image acquisition to processing start by the latency of the Archiving service and Data Backbone transfer.
A summary status for the processing performed is sent to the OCS Driven Batch Control service to be returned to the OCS.

\subsubsection{Telemetry Gateway}\label{telemetry-gateway}

This service obtains information from the Prompt NCSA Enclave,
particularly status and quality metrics from Prompt Processing of images
and the Prompt Quality Control service, and transmits it to the OCS as
specified in the \textit{Data Management-OCS Software Communication Interface}
\citedsp{LSE-72}. Note that more detailed information on the status and
performance of DMS services will also be available to Observatory
operators through remote displays originated from the
Service Management/Monitoring infrastructure services in all DMS enclaves.

\subsection{Interfaces}\label{base-interfaces}

OCS to all Prompt Base Enclave services: these interface through the SAL
library provided by the OCS subsystem.

Archiver and Catch-Up Archiver to Observatory Operations Data Service:
image files with associated metadata are written to OODS storage.

Observatory Operations Data Service to Data Backbone: files are copied to
Data Backbone storage via a file transfer mechanism, and their
information and metadata are registered with Data Backbone management
dataabases. The Data Butler is not used for this low-level,
non-science-payload interface.

Observatory Operations Data Service to Commissioning Cluster LSP and other users: the OODS provides a mountable POSIX filesystem interface.
A web interface (e.g. WebDAV) may also be provided, but this will be as simple as possible to enable maintaining a very high level of service availability and reliability.

EFD to EFD Transformer: this interface is via connection to the
databases that make up the EFD as well as file transfer from the EFD's
Large File Annex.

EFD Transformer to Data Backbone: Transformed EFD entries are inserted
into the ``Transformed EFD'' database resident within the Data Backbone.

Camera Data System to Archiver, Catch-Up Archiver, Prompt Processing
Ingest: these interface through the custom library provided by the
Camera Data System.

Prompt Processing Ingest to Prompt Processing: BBFTP is used to transfer
files over the international network from the ingest service to the
processing service.

OCS Driven Batch to Batch Computing: HTCondor is
used to transfer execution instructions
from the control service to the batch service, whether to the Commissioning Cluster or over the international network to the Offline Production Enclave, and return status and
result information.

Telemetry Gateway from Prompt NCSA Enclave services: RabbitMQ is
used to transfer status and quality metrics to the gateway over the
international network.

\section{Prompt NCSA Enclave}\label{prompt-ncsa-enclave}

This enclave is responsible for the compute-intensive processing for all
near-realtime operations and other operations closely tied with the
Observatory. Its primary goals are to process images and metadata from
the Observatory into ``online'' science data products and publish them
to the DACs, alert subscribers, and back to the OCS.

The Prompt Processing service executes science payloads that are tightly integrated with the observing cadence and is intended to function in near-realtime with strict result deadlines for both science and raw calibration images.

Note that offline (typically daytime) processing to generate Prompt data products occurs under the control of the Batch Production service in the Offline Production Enclave using its Batch Computing resources.

The Alert Distribution service receives batches of alerts resulting from Prompt Processing of each science visit; it then provides bulk alert streams to community alert brokers and filtered alert streams to LSST data rights holders.

The Prompt Quality Control service monitors the ``online'' science data
products, including alerts, notifying operators if any anomalies are
found.

Like the services in the Base Center, these services need to run
rapidly and reliably and so are designed to execute autonomously.

\subsection{Service Descriptions}\label{prompt-ncsa-service-descriptions}

Detailed concepts of operations for each service can be found in
\textit{Concept of Operations for the LSST Production Services} \citedsp{LDM-230}.

\subsubsection{Prompt Processing}\label{prompt-processing}

This service receives crosstalk-corrected images and metadata from the
Prompt Processing Ingest service at the Base and executes the Alert
Production science payload on them, generating ``online'' data products
that are stored in the Data Backbone. The Alert Production payload then
sends alerts to the Alert Distribution service.

The Prompt Processing service has calibration (including Collimated Beam
Projector images), science, and deep drilling modes. In calibration
mode, it executes a Raw Calibration Validation payload that provides
rapid feedback of raw calibration image quality. In normal science mode,
two consecutive exposures are grouped and processed as a single visit.
Definitions of exposure groupings to be processed as visits in deep
drilling and other modes are TBD. The service is required to deliver
Alerts within 60 seconds of the final camera readout of a standard
science visit with 98\% reliability.

There is no Prompt Processing service instance for the Auxiliary
Telescope Spectrograph.

\subsubsection{Alert Distribution}\label{alert-distribution}

This service obtains alerts generated by the Alert Production science payload and distributes them to community alert brokers.
A full Alert stream is also fed to a filtering component (also known as the "mini-broker"); that component allows individual LSST data rights holders to execute limited filters against the stream, producing filtered feeds that are then distributed to the individuals.

\subsubsection{Prompt Quality Control}\label{prompt-quality-control}

This service collects information on Prompt science and calibration
payload execution, post-processes the science data products from the
Data Backbone to generate additional measurements, and monitors the
measurement values against defined thresholds, providing an automated
quality control capability for potentially detecting issues with the
environment, telescope, camera, data acquisition, or data processing.
Alarms stemming from threshold crossings are delivered to Observatory
operators and to LSST Data Facility Production Scientists for
verification, analysis, and resolution.

\subsection{Interfaces}\label{prompt-ncsa-interfaces}

Prompt Processing to Alert Distribution: these
interface through a reliable transport system.

Prompt Processing to Batch Production: in the event that Prompt
Processing runs over its allotted time window, processing can be
cancelled and the failure recorded, after which Offline Processing within
the Batch Production service will
redo the processing at a later time. Note that it may be possible, if
sufficient computational resources have been provisioned, for the Prompt
Processing to be allowed to continue to run, with spare capacity used to
maintain latency for future visits. In that case, there would
effectively be an infinite time window.

Science Payloads to Data Backbone: payloads use the Data Butler as a
client to access files and catalog databases within the Data Backbone.

\section{Offline Production Enclave}\label{offline-production-enclave}

This enclave is responsible for all longer-period data processing
operations, including the largest and most complex payloads supported by
the DMS: the annual Data Release Production (DRP) and periodic
Calibration Products Productions (CPPs). Note that CPPs will execute
even while the annual DRP is executing.
 The Offline Quality Control Service monitors the science data
products, notifying operators if any anomalies are found.

The services in this enclave need to run efficiently and reliably over
long periods of time, spanning weeks or months. They need to execute
millions or billions of tasks when their input data becomes available
while tracking the status of each and preserving its output. They are
designed to execute autonomously with human oversight, monitoring, and
control primarily at the highest level, although provisions are made for
manual intervention if absolutely necessary.

This enclave does not have direct users (besides the operators of its
services); the services within it obtain inputs from the Data Backbone
and place their outputs into the Data Backbone.

\subsection{Service Descriptions}\label{ncsa-gen-prod-service-descriptions}

\subsubsection{Batch Production}\label{batch-production}

This service executes science payloads as "campaigns" consisting of a defined pipeline, a defined configuration, and defined inputs and outputs.
Many different payloads may be executed on many different campaign cadences.
These include:
\begin{itemize}
	\item Offline processing for Prompt data products
	\item Calibration Products Production
	\item Template Production
	\item Special Programs Production
	\item Data Release Production
\end{itemize}

The service is able to handle massively distributed computing, executing jobs when their inputs become available and tracking their status and outputs.
It ensures that the data needed for a job is accessible to it and that outputs (including log files, if any) are preserved.
It can allocate work across multiple enclaves, in particular between NCSA and the Satellite Facility at CC-IN2P3, which will have capacity for half of the DRP processing.

Offline processing ensures that Prompt data products are generated within the nominal 24 hours.
It includes catch-up of missed nightly processing as well as daytime processing such as the Moving Object Processing System.

Calibration Products Production campaigns execute various CPP science payloads at intervals to
generate Master Calibration Images and populate the Calibration Database
with information derived from analysis of raw calibration images from
the Data Backbone and information in the Transformed EFD. This includes
the computation of crosstalk correction matrices.
Additional information such as external catalogs are also
taken from the Data Backbone. The intervals at which this service
executes will depend on the stability of Observatory systems but are
expected to include at least monthly and annual executions. The annual
execution is a prerequisite for the subsequent execution of the Data
Release Production. The service involves human scientist/operator input
to determine initial configurations of the payload, to monitor and
analyze the results, and possibly to provide additional configuration
information during execution.

Template Production campaigns, typically run annually, generate the static sky templates used by Alert Production, based on raw science images from the Data Backbone.

Special Programs Production campaigns perform custom analyses on raw science images taken for these programs.
The cadence for these campaigns may vary based on the particular Special Program.

Data Release Production campaigns execute the DRP science payload annually to generate all
Data Release data products after the annual CPP is executed. A small-scale
(about 10\% of the sky) mini-production is executed first to ensure
readiness, followed by the full production. Raw science images are taken
from the Data Backbone along with Master Calibration Images and
information from the Transformed EFD. Additional information such as
external catalogs may also be taken from the Data Backbone.

Output data products from both the mini-production and the main
production are loaded into the Data Backbone, including both images and
catalogs. From there, they are analyzed by LSST staff scientists and
selected external scientists using the Science Validation instance of
the LSST Science Platform to ensure quality and readiness for release.
The to-be-released data products are loaded into the Data Access Center
services, and access is then enabled on the release date. The service
involves human scientist/operator/programmer input to determine initial
configurations of the payload, to monitor and analyze results, and, when
absolutely necessary, to make ``hot fixes'' during execution that
maintain adequate consistency of the resulting data products.

\subsubsection{Offline Quality Control}\label{offline-quality-control}

This collects information on Calibration, Template Generation, and Data Release science payload execution,
post-processes the science data products from the Data Backbone to
generate additional measurements, and monitors the measurement values
against defined thresholds, providing an automated quality control
capability for potentially detecting issues with the data processing but
also the environment, telescope, camera, or data acquisition. Alarms
stemming from threshold crossings are delivered to LSST Data Facility
Production Scientists for verification, analysis, and resolution.

\subsection{Interfaces}\label{ncsa-general-production-interfaces}

Batch Production to Data Backbone: for large-scale productions, a workflow
system is expected to stage files and selected database entries from the
Data Backbone to local storage for access by the science payloads via
the Data Butler. Similarly, the staging system will ingest output images
and catalogs into the Data Backbone.

Batch Production to Satellite Facility: the Data Backbone will transfer raw data, including images, metadata, and the Transformed EFD, to the Satellite Facility.
Intermediate data products will be transferred back via the Data Backbone for further computations in the Offline Production Enclave.

\section{Data Access Centers}\label{data-access-centers}

There are two Data Access Centers, one in the US at NCSA and one in
Chile at the Base. These DACs are responsible for all
science-user-facing services, primarily instances of the LSST Science
Platform (LSP). The LSP is the preferred analytic interface to LSST data
products in the DAC. It provides computation and data access on both
interactive and asynchronous timescales. The US DAC also includes a
service for distributing bulk data on daily and annual (Data Release)
timescales to partner institutions, collaborations, and LSST Education
and Public Outreach (EPO).

The services in each enclave must support multiple users simultaneously
and securely. The LSP must be responsive to science user needs; updates
are likely to occur at a different cadence from the other enclaves as a
result. The LSP must operate reliably enough that scientific work is not
impeded.

\subsection{Service Descriptions}\label{dac-service-descriptions}

\subsubsection{LSST Science Platform DAC
instances}\label{lsst-science-platform-dac-instances}

This service provides an exploratory analysis environment for science
users. It can be further broken down into three ``Aspects'' that it
presents to end users, along with underlying ``backend services'' that
users can take advantage of, as illustrated in Figure~\ref{fig:lsp}.

\begin{figure}
\centering
\includegraphics[width=0.5\textwidth]{images/SciencePlatform.pdf}
\caption{LSST Science Platform}
\label{fig:lsp}
\end{figure}

The ``Portal'' Aspect provides a pre-specified yet flexible discovery,
query, and viewing tool. The ``JupyterLab'' Aspect provides a fully
flexible (``notebook'') environment incorporating rendering of images,
catalogs, and plots and providing for execution of LSST-provided and
custom algorithms. The ``Web API'' Aspect provides a
language-independent, VO-compliant Web Services data access API with
extensions for LSST capabilities and volumes. Access is provided via all
three Aspects to all data products, including images, catalogs, and
metadata. The Web API Aspect regenerates ``virtual'' data products on
demand when required.

The backend services provide general-purpose user computation, including
batch job submission from the Containerized Application Management Service and Batch Computing Service pools; file storage for User Generated data products which is accessible to all three Aspects and in particular exposed through one or more Web API Aspect services; and database storage for User Generated relational tables which is also accessible to all three Aspects.
User Generated data may be shared with
individual users, with groups, or with all DAC users (data rights
holders). Resource management of the backend services is based on a
small ``birthright'' quota with additional resources allocated by a
committee.

LSST Science Platform instances access read-only data products, including files and databases, from the Data Backbone.
In addition, files may be cached within the LSP instance for speed, and large-scale catalogs are typically loaded into an instance of the Qserv database management system for efficient query and analysis.

All usage of any LSST Science Platform instance requires authentication
to ensure availability only to LSST data rights holders or LSST
operations staff.

\subsubsection{Bulk Distribution}\label{bulk-distribution}

This service is used to transmit Prompt and Data Release data products to partners such as LSST Education and Public Outreach, the UK LSST project, and the Dark Energy Science Collaboration.
It extracts data products from the Data Backbone and transmits them over high bandwidth connections to designated, pre-subscribed partners.


\subsection{Interfaces}\label{dac-interfaces}

Bulk Distribution and LSST Science Platform to Data Backbone: Both
DAC-resident services retrieve their data, including raw images, nightly and
annual image and catalog data products, metadata, and provenance, from the Data
Backbone.  The LSP Portal Aspect uses the LSP Web APIs to retrieve data.  The
LSP JupyterLab Aspect can use the LSP Web APIs and also can use the Data Butler
client library to access the Data Backbone.

Bulk Distribution to partners: The exact delivery mechanism for
large-scale data distribution is TBD.


\section{Commissioning Cluster}\label{commissioning-cluster}

\subsection{Service Description}\label{commcluster-service}

\subsubsection{LSST Science Platform Commissioning
instance}\label{lsst-science-platform-commissioning-instance}

This instance of the LSST Science Platform for Science Validation runs
on the Commissioning Cluster at the Base Facility (but also has access
to computational resources at the Archive) and accesses a Base endpoint
for the Data Backbone. This location at the Base lowers the latency of
both access to Data Backbone-resident data (which does not have to wait
for transfer over the international network) and, perhaps more
importantly, for user interface operations for staff in Chile, which are
served locally. Note that the Commissioning Cluster does not have direct
access to the Camera Data System; it relies on the Archiver service to
obtain data. The Commissioning Cluster will have direct access to the
OCS's Base replica of the EFD (before transformation).

\subsection{Interfaces}\label{commcluster-interfaces}

Commissioning Cluster to Data Backbone: The Commissioning Cluster relies on the
Data Backbone for its data, like the other instances of the LSST Science
Platform.

Commissioning Cluster to EFD: The Commissioning Cluster has direct read-only
client access to the Base replica of the EFD (before transformation).

\section{Backbone Services}\label{backbone-services}

Detailed concepts of operations for each service can be found in \textit{Concept of Operations for the LSST Production Services} \citedsp{LDM-230}.

\subsection{Service Descriptions}\label{backbone-service-descriptions}

The Data Backbone is a key component that provides for data storage, transport, and replication, allowing data products to move between computational environments.
This service provides policy-based replication of files (including images and flat files to be loaded into databases as well as other raw and intermediate files) and databases (including metadata about files as well as other miscellaneous databases but not including the large Data Release catalogs) across multiple physical locations, including the Base, Commissioning Cluster, NCSA, and DACs.
It manages caches of files at each endpoint as well as persistence to long-term archival storage (e.g. tape).
It provides a registration mechanism for new datasets and database entries and a retrieval mechanism compatible with the Data Butler.

The relationships between the Data Backbone components are illustrated
in Figure~\ref{fig:dbb}.

\begin{figure}
\centering
\includegraphics[width=0.7\textwidth]{images/DataBackbone.pdf}
\caption{Data Backbone}
\label{fig:dbb}
\end{figure}

\subsubsection{DBB Ingest/Metadata Management}\label{dbb-ingest-metadata}

...

\subsubsection{DBB Lifetime Management}\label{dbb-lifetime-metadata}

...

\subsubsection{DBB Transport/Replication/Backup}\label{dbb-transport-repl}

...

\subsubsection{DBB Storage}\label{dbb-storage}

...

\subsection{Interfaces}\label{backbone-interfaces}

The Data Backbone services interact with most other deployed services.


\section{Software Components}\label{software-components}

\subsection{Science Payloads}\label{science-payloads}

These payloads are described in more detail in the DM Applications Design Document \citedsp{LDM-151}.
Payloads are built from application software components.
Most payloads execute under control of the Batch Production service.
Exceptions include the Alert Production Payload, the Raw Calibration Validation Payload, and the Daily Calibration Products Update Payload.

\subsubsection{Alert Production Payload}\label{alert-production-payload}

Executes under control of the Prompt Processing service. Generates all
Prompt science data products including Alerts (with the exception of
Solar System object orbits) and loads them into the Data Backbone and
Prompt Products Database. Transmits Alerts to Alert Distribution service.
Generates image quality feedback to the OCS and observers via the
Telemetry Gateway. Uses crosstalk-corrected science images and
associated metadata delivered by the Prompt Processing service; uses
Master Calibration Images, Template Images, Prompt Products Database, and
Calibration Database information from the Data Backbone.

\subsubsection{MOPS Payload}\label{mops-payload}

Executes after a night's
observations are complete. Generates entries in the MOPS Database and
the Prompt Products Database, including Solar System Object records,
measurements, and orbits. Performs precovery forced photometry of
transients. Uses Prompt Products Database entries and images from the Data
Backbone.

\subsubsection{Raw Calibration Validation
Payload}\label{raw-calibration-validation-payload}

Executes under control of the Prompt Processing service. Generates raw
calibration image quality feedback to the OCS and observers via the
Telemetry Gateway. Uses crosstalk-corrected science images and
associated metadata delivered by the Prompt Processing service, Master
Calibration Images, and Calibration Database information from the Data
Backbone.

\subsubsection{Daily Calibration Products Update
Payload}\label{daily-calibration-products-update-payload}

Executes under control of the OCS-controlled batch processing service so
that its execution can be synchronized with the observing schedule. Uses
raw calibration images and information from the Transformed EFD to
generate a subset of Master Calibration Images and Calibration Database
entries in the Data Backbone.

\subsubsection{Periodic Calibration Products Production
Payload}\label{periodic-calibration-products-production-payload}

Executes at nominally monthly intervals but perhaps as frequently as weekly or as
infrequently as quarterly, depending on the stability of Observatory
systems and their calibrations. Uses raw calibration images and
information from the Transformed EFD to generate a subset of Master
Calibration Images and Calibration Database entries in the Data
Backbone.

\subsubsection{Special Programs Production
Payload}\label{special-programs-production-payload}

Executes at program-defined intervals.
Uses raw science images to generate special programs science data products, placing them in the Data Backbone.

\subsubsection{Template Generation
Payload}\label{template-generation-payload}

Executes if necessary to generate templates for Alert Production in between annual
Data Release Productions. Uses raw science images to generate the
templates, placing them in the Data Backbone.

\subsubsection{Annual Calibration Products Production
Payload}\label{annual-calibration-products-production-payload}

Executes at annual intervals prior to the start of the Data Release Production. Uses
raw calibration images, information from the Transformed EFD,
information from the Auxiliary Telescope Spectrograph, and external
catalogs to generate Master Calibration Images and Calibration Database
entries in the Data Backbone.

\subsubsection{Data Release Production
Payload}\label{data-release-production-payload}

Executes at annual intervals,
first running a ``mini-DRP'' over a small portion of the sky, followed
by the full DRP over the entire sky. Produces science data products in
the Data Backbone.

\subsection{SUIT}\label{suit}

The Science User Interface and Tools provide visualization, plotting,
catalog rendering, browsing, and searching elements that can be
assembled into predetermined ``portals'' but can also be used flexibly
within dynamic ``notebook'' environments.

\subsection{Middleware}\label{middleware}

The detailed design of the Middleware components is in \textit{Data Management Middleware Design} \citedsp{LDM-152}.

\subsubsection{Data Butler Access
Client}\label{data-butler-access-client}

The Data Butler provides an access abstraction for all science payloads
that enables their underlying data sources and destinations to be
configured at runtime with a variety of back-ends ranging from local
disk to network locations and a variety of serializations ranging from
YAML and FITS files (extensible to HDF5 or ASDF) to database tables. The
Butler client is also available within the LSST Science Platform
JupyterLab environment.

\subsubsection{Parallel Distributed Database
(Qserv)}\label{parallel-distributed-database-qserv}

Underlying the catalog data access web service is a parallel distributed
database required to handle the petabyte-scale,
tens-of-trillions-of-rows catalogs produced by LSST.

\subsubsection{Task Framework}\label{task-framework}

The Task Framework is a Python class library that provides a structure
(standardized class entry points and conventions) to organize low-level
algorithms into potentially-reusable algorithmic components (Tasks; e.g.
dark frame subtraction, object detection, object measurement), and to
organize tasks into basic pipelines (SuperTasks; e.g., process a single
visit, build a coadd, difference a visit). The algorithmic code is
written into (Super)Tasks by overriding classes and providing
implementation for standard entry points. The Task Framework allows the
pipelines to be constructed and run at the level of a single node or a
group of tightly-synchronized nodes. It allows for sub-node
parallelization: trivial parallelization of Task execution, as well as
providing (in the future) parallelization primitives for development of
multi-core Tasks and synchronized multi-node Tasks.

The Task Framework serves as an interface layer between orchestration
and the algorithmic code. It exposes a standard interface to
``activators'' (command-line runners as well as the orchestration layer
and QA systems), which use it to execute the code wrapped in tasks. The
Task Framework does not concern itself with fault-tolerant massively
parallel execution of the pipelines over multiple (thousands) of nodes
nor any staging of data that might be required; this is the concern of
the orchestration middleware.

The Task Framework exposes to the orchestration system needs and
capabilities of the underlying algorithmic code (i.e., the number of
cores needed, expected memory-per-core, expected need for data). It may
also receive from the orchestration layer the information on how to
optimally run the particular task (i.e., which level of intra-node
parallelization is be desired).

It also includes a configuration API and a logging API.


\section{Infrastructure Services}\label{infrastructure-services}

The information technology and communications infrastructure is composed of services and systems that form the computing environments on top of which the services in the enclaves described above are deployed and operate.

\subsection{Service Descriptions}\label{infrastructure-service-descriptions}

Detailed concepts of operations for each service can be found in \textit{Concept of Operations for the LSST Production Services} \citedsp{LDM-230}.
The detailed design of the infrastructure is in \textit{LSST Data Facility Logical Information Technology and Communications Design} \citedsp{LDM-129} and \textit{Network Design} \citedsp{LSE-78}.

\subsubsection{Managed Database}\label{managed-database}

This service provides general-purpose relational database management that supports other services.
It includes metadata and provenance, but it does not include the large catalog science data products that are generated as files and loaded into the Qserv parallel distributed database.
For efficiency of resource usage and management, most databases are consolidated into a single RDBMS instance.

\subsubsection{Batch Processing}\label{batch-processing}

This service provides execution of batch jobs with a variety of priorities from a variety of users in a variety of environments (e.g. OS and software configurations) on the underlying provisioned compute resources.
It will use containerization to handle heterogeneity of environments.
HTCondor is the baseline technology choice for this service.

Some compute resources are reserved for particular uses, but others can
be flexibly provisioned, up to a certain maximum quota, if needed to
deal with surges in processing.

The priority order for processing is:
\begin{itemize}
\item
  Prompt processing
\item
  Offline processing for Prompt data products
\item
  OCS-controlled batch processing
\item
  LSP Commissioning Cluster processing
\item
  LSP Science Validation processing
\item
  LSP Data Access Center processing
\item
  Template and Calibration Products Production
\item
  Special Programs Production
\item
  Data Release Production
\end{itemize}

\subsubsection{Containerized Application Management}\label{containerized-application-management}

This service provides compute, local-to-node storage, and local-to-LAN storage resources for deploying containerized service-oriented systems, especially the Science Platforms but also including parts of the Quality Control systems and Developer Services.
It allows allocation of compute and storage resources as well as reproducible, controlled deployment of services onto those resources.
Kubernetes is the baseline technology choice for this service.

\subsubsection{IT Security}\label{it-security}

This service provides security, including monitoring, vulnerability management, incident detection and response, access controls, intrusion detection, and configuration management.
Information security is provided in accordance with the \textit{LSST Master Infromation Security Policy} \citedsp{LPM-121}.

\subsubsection{Identity Management}\label{identity-management}

This service provides authentication and authorization for all users of
any DMS component, especially the LSST Science Platform instances.

\subsubsection{ITC Provisioning and Management}\label{itc-provisioning-management}

This service provides for acquisition, change and configuration management, and provisioning of information technology components, whether purchased ``bare metal'', provided by the NCSA Commons, or provided by agreements with outside providers.

This service includes deployment of non-containerized services.
For example, many of the services at the Base Facility are not highly dynamic or flexible, as they primarily provide interfacing to the OCS and Camera Data System.
The baseline provisioning mechanism for them is vSphere; they will be deployed using Puppet.

\subsubsection{Service Management/Monitoring}\label{service-management-monitoring}

These services provide management and monitoring at the service level for each computing environment.
They include standard IT processes such as service design, service transition (including change, release, and configuration management), and service delivery (including incident and request response plus problem management).

Service monitoring reports in dashboard and document form on the health and status of all services.

\subsection{Interfaces}\label{infrastructure-interfaces}

The infrastructure services generally interact with all other deployed
services.

Identity management instances are present in the Base Center and at NCSA.
(Another replica will be maintained at the Summit.)  These are used to support
authentication and authorization for the other physically co-located environments:
the Commissioning Cluster and the two Data Access Centers.


\section{NCSA Development and Integration Enclave}\label{ncsa-development-integration-enclave}

This enclave encompasses environments for analysts, developers, and
integration and test. Its users are the Observatory staff as they
analyze raw data and processed data products to characterize them,
develop new algorithms and systems, and test new versions of components
and services before deployment.

Integration environments in this enclave represent various deployment environments, deployment services, test datasets, test execution services, metric measurement and tracking services.
This environment includes the ``Level 1'' Test Stand, which includes a DM instance of the Camera DAQ used in an integration environment simulating the Base Center.
It also includes the Prototype Data Access Center (PDAC), which is an integration environment simulating a Data Access Center.

\subsection{Service Descriptions}\label{ncsa-dev-int-service-descriptions}

\subsubsection{LSST Science Platform Science Validation
instance}\label{lsst-science-platform-science-validation-instance}

This instance of the LSST Science Platform is customized to allow access
to unreleased and intermediate data products from the Alert, Calibration
Products, and Data Release Productions. It is optimized for usage by
scientists within the LSST Operations team, although selected external
scientists can be granted access to assist with Science Validation. Part
of the optimization is to size and configure the three Aspects of the
LSP appropriately; in particular, more JupyterLab usage and less portal
usage is expected.

\subsubsection{Developer Services}\label{developer-services}

These services include a software version control service, a build, unit test, and continuous integration service, a documentation publication service, developer communications services, an issue/ticket tracking service, etc., necessary to support science and system developers as they create and debug new versions of the Data Management System.

\subsection{Interfaces}\label{ncsa-development-integration-interfaces}

LSP Science Validation instance and Developer Services to Data Backbone:
All services in this enclave interface with the Data Backbone.  The LSP
Science Validation instance is used to inspect, analyze, and validate the data
products of the Data Release Production prior to their release and so has
access to those products in the Data Backbone; since it may be used to annotate
the data products, it can also write to the Data Backbone.  The Developer
Services may use raw data, intermediate data products, and final
data products to do development, perform tests, and debug problems.

Developer Services do not have direct interfaces with the rest of the
operational system; they communicate via the distributed source version control
system, the package management system, and the configuration system.



\section{Design and Implementation Standards}\label{design-implementation-standards}

Standards have been adopted by the DM Change Control Board (CCB) that apply to
all component designs within the LSST DM System.  Coding standards and the like
that are not pertinent to design may be found in the LSST DM Developer Guide
(\citep{DevGuide}).

\subsection{HTTPS Protocol}

In the absence of a specific technical justification and acceptance by the LSST
Information Security Officer and DM Change Control Board, all Web-enabled user
interfaces and Web services exposed to users and the public Internet will use
the HTTPS protocol and not the HTTP protocol.  To reiterate: this is only a
default, and exceptions can be made when justified.

The covered interfaces include those of the three LSP Aspects (Portal,
JupyterLab, and Web APIs).

The requirement to implement data access policies limiting data access to
identified rights holders will require all, or nearly all, data access to be
authenticated provides a strong technical justification.  In addition, it
appears to be appropriate "technical best practice" in the current Internet
environment, in the absence of good reasons to do otherwise.


\newpage
\section{Appendix: Traceability}\label{appendix-traceability}

\subsection{Requirement to Component
Traceability}\label{requirement-to-component-traceability}

\footnotesize
\begin{longtable}{p{0.4\textwidth}p{0.55\textwidth}}
\hline
\multicolumn{1}{c}{\textbf{Requirement}} &
\multicolumn{1}{c}{\textbf{Components}} \\ \hline
\endhead

CA-DM-CON-ICD-0007 Provide Data Management Conditions data & Prompt Processing Ingest, Prompt Processing, Telemetry Gateway, Prompt Base Enclave \\ \hline
CA-DM-CON-ICD-0008 Data Management Conditions data latency & Prompt Processing Ingest, Prompt Processing, Telemetry Gateway, Prompt Base Enclave \\ \hline
CA-DM-CON-ICD-0013 Transport Camera image data &  \\ \hline
CA-DM-CON-ICD-0014 Provide science sensor data & Archiving, Prompt Processing Ingest, Image Ingest and Processing, Prompt Base Enclave \\ \hline
CA-DM-CON-ICD-0015 Provide wavefront sensor data & Archiving, Prompt Processing Ingest, Image Ingest and Processing, Prompt Base Enclave \\ \hline
CA-DM-CON-ICD-0016 Provide guide sensor data & Archiving, Prompt Processing Ingest, Image Ingest and Processing, Prompt Base Enclave \\ \hline
CA-DM-CON-ICD-0017 Data Management load on image data interfaces & Archiving, Prompt Processing Ingest, Image Ingest and Processing, Prompt Base Enclave \\ \hline
CA-DM-CON-ICD-0019 Camera engineering image data archiving & Archiving, Image Ingest and Processing, Archive Base Enclave, Archive NCSA Enclave, Prompt Base Enclave, Prompt NCSA Enclave \\ \hline
CA-DM-DAQ-ICD-0051 Crosstalk Correction &  \\ \hline
CA-DM-DAQ-ICD-0052 Correction constants for science sensors sourced by Data Management & Observatory Operations Data, Observatory Operations Data Service SW, Science Pipelines Distribution \\ \hline
DM-TS-CON-ICD-0002 Timing & Prompt Processing Ingest, Prompt Processing, Telemetry Gateway, Image Ingest and Processing, Prompt Base Enclave \\ \hline
DM-TS-CON-ICD-0003 Wavefront image archive access & Observatory Operations Data, Observatory Operations Data Service SW, Base Facility, NCSA Facility, Archive Base Enclave, Archive NCSA Enclave, Offline Production Enclave, Prompt Base Enclave \\ \hline
DM-TS-CON-ICD-0004 Use OCS for data transport & Telemetry Gateway, Prompt Base Enclave \\ \hline
DM-TS-CON-ICD-0006 Data & Prompt Processing, Telemetry Gateway, Prompt Base Enclave \\ \hline
DM-TS-CON-ICD-0007 Timing & Prompt Processing Ingest, Prompt Processing, Telemetry Gateway, Image Ingest and Processing, Prompt Base Enclave \\ \hline
DM-TS-CON-ICD-0008 LSST Stack Availability & Science Pipelines Libraries \\ \hline
DM-TS-CON-ICD-0009 Calibration Data Products & Observatory Operations Data, Prompt Base Enclave \\ \hline
DM-TS-CON-ICD-0011 Data Format & Prompt Processing, Telemetry Gateway, Prompt Base Enclave \\ \hline
DMS-REQ-0002 Transient Alert Distribution & Prompt Processing, Alert Distribution, Alert Distribution SW, Alert Production, Prompt NCSA Enclave \\ \hline
DMS-REQ-0004 Nightly Data Accessible Within 24 hrs & Prompt Processing Ingest, Prompt Processing, Alert Distribution, Alert Distribution SW, EFD Transformation, Image Ingest and Processing, Alert Production, MOPS and Forced Photometry, Archive Base Enclave, Archive NCSA Enclave, DAC Chile Enclave, DAC US Enclave, Offline Production Enclave, Prompt Base Enclave, Prompt NCSA Enclave \\ \hline
DMS-REQ-0008 Pipeline Availability & Archiving, Prompt Processing Ingest, Prompt Processing, OCS-Driven Batch, Telemetry Gateway, Alert Distribution, Batch Production, DBB Ingest/Metadata Management, DBB Transport/Replication/Backup, DBB Storage, Alert Distribution SW, EFD Transformation, Image Ingest and Processing, OCS Batch SW, Campaign Management, Workload/Workflow Management, DBB Ingest/Metadata Management SW, DBB Transport/Replication/Backup SW, Network Management, Base Facility, NCSA Facility, Offline Production Enclave, Prompt Base Enclave, Prompt NCSA Enclave \\ \hline
DMS-REQ-0009 Simulated Data & Alert Production, Data Release Production \\ \hline
DMS-REQ-0010 Difference Exposures & Alert Production, Archive Base Enclave, Archive NCSA Enclave, Prompt NCSA Enclave \\ \hline
DMS-REQ-0018 Raw Science Image Data Acquisition & Archiving, Image Ingest and Processing, Prompt Base Enclave \\ \hline
DMS-REQ-0020 Wavefront Sensor Data Acquisition & Archiving, Image Ingest and Processing, Prompt Base Enclave \\ \hline
DMS-REQ-0022 Crosstalk Corrected Science Image Data Acquisition & Prompt Processing Ingest, Image Ingest and Processing, Prompt Base Enclave \\ \hline
DMS-REQ-0024 Raw Image Assembly & Archiving, Image Ingest and Processing, Prompt Base Enclave \\ \hline
DMS-REQ-0029 Generate Photometric Zeropoint for Visit Image & Alert Production, Archive Base Enclave, Archive NCSA Enclave, Prompt NCSA Enclave \\ \hline
DMS-REQ-0030 Generate WCS for Visit Images & Alert Production, Archive Base Enclave, Archive NCSA Enclave, Prompt NCSA Enclave \\ \hline
DMS-REQ-0032 Image Differencing & Alert Production, Data Release Production, Science Pipelines Libraries \\ \hline
DMS-REQ-0033 Provide Source Detection Software & Alert Production, Data Release Production, Science Pipelines Libraries \\ \hline
DMS-REQ-0034 Associate Sources to Objects & Data Release Production, Offline Production Enclave \\ \hline
DMS-REQ-0042 Provide Astrometric Model & Alert Production, Data Release Production, Science Pipelines Libraries \\ \hline
DMS-REQ-0043 Provide Calibrated Photometry & Alert Production, Data Release Production, Science Pipelines Libraries \\ \hline
DMS-REQ-0046 Provide Photometric Redshifts of Galaxies & Data Release Production, Distributed Database, Offline Production Enclave \\ \hline
DMS-REQ-0047 Provide PSF for Coadded Images & Data Release Production, Offline Production Enclave \\ \hline
DMS-REQ-0052 Enable a Range of Shape Measurement Approaches & Alert Production, Data Release Production, Science Pipelines Libraries \\ \hline
DMS-REQ-0059 Bad Pixel Map & Science Plugins, Science Pipelines Distribution, Offline Production Enclave \\ \hline
DMS-REQ-0060 Bias Residual Image & Science Plugins, Science Pipelines Distribution, Offline Production Enclave \\ \hline
DMS-REQ-0061 Crosstalk Correction Matrix & Science Plugins, Science Pipelines Distribution, Offline Production Enclave \\ \hline
DMS-REQ-0062 Illumination Correction Frame & Science Plugins, Science Pipelines Distribution, Offline Production Enclave \\ \hline
DMS-REQ-0063 Monochromatic Flatfield Data Cube & Science Plugins, Science Pipelines Distribution, Offline Production Enclave \\ \hline
DMS-REQ-0065 Provide Image Access Services & LSP Web API, LSP Web API SW, Image/Cutout Server \\ \hline
DMS-REQ-0068 Raw Science Image Metadata & Archiving, DBB Ingest/Metadata Management, Image Ingest and Processing, DBB Ingest/Metadata Management SW, Prompt Base Enclave \\ \hline
DMS-REQ-0069 Processed Visit Images & Alert Production, Archive Base Enclave, Archive NCSA Enclave, Prompt NCSA Enclave \\ \hline
DMS-REQ-0070 Generate PSF for Visit Images & Alert Production, Prompt NCSA Enclave \\ \hline
DMS-REQ-0072 Processed Visit Image Content & Alert Production, Prompt NCSA Enclave \\ \hline
DMS-REQ-0074 Difference Exposure Attributes & DBB Ingest/Metadata Management, DBB Ingest/Metadata Management SW, Alert Production, Prompt NCSA Enclave \\ \hline
DMS-REQ-0075 Catalog Queries & LSP Web API, LSP Web API SW, Distributed Database, ADQL Translator, DAC Chile Enclave, DAC US Enclave \\ \hline
DMS-REQ-0077 Maintain Archive Publicly Accessible & DBB Ingest/Metadata Management, DBB Storage, DBB Ingest/Metadata Management SW, Distributed Database, Archive Base Enclave, Archive NCSA Enclave, DAC Chile Enclave, DAC US Enclave \\ \hline
DMS-REQ-0078 Catalog Export Formats & LSP Web API, LSP Web API SW, ADQL Translator, Archive Base Enclave, Archive NCSA Enclave, DAC Chile Enclave, DAC US Enclave \\ \hline
DMS-REQ-0089 Solar System Objects Available Within Specified Time & DBB Ingest/Metadata Management, DBB Transport/Replication/Backup, DBB Storage, LSP Web API, DBB Ingest/Metadata Management SW, DBB Transport/Replication/Backup SW, LSP Web API SW, MOPS and Forced Photometry, Archive Base Enclave, Archive NCSA Enclave, DAC Chile Enclave, DAC US Enclave \\ \hline
DMS-REQ-0094 Keep Historical Alert Archive & DBB Ingest/Metadata Management, DBB Storage, DBB Ingest/Metadata Management SW, Archive Base Enclave, Archive NCSA Enclave, DAC Chile Enclave, DAC US Enclave \\ \hline
DMS-REQ-0096 Generate Data Quality Report Within Specified Time & Prompt Quality Control, Quality Control SW, Prompt Base Enclave, Prompt NCSA Enclave \\ \hline
DMS-REQ-0097 Level 1 Data Quality Report Definition & Prompt Quality Control, Quality Control SW, Alert Production, Prompt Base Enclave, Prompt NCSA Enclave \\ \hline
DMS-REQ-0098 Generate DMS Performance Report Within Specified Time & Prompt Quality Control, Quality Control SW, Prompt Base Enclave, Prompt NCSA Enclave \\ \hline
DMS-REQ-0099 Level 1 Performance Report Definition & Archiving, Prompt Processing Ingest, Prompt Processing, Prompt Quality Control, Image Ingest and Processing, Quality Control SW, Prompt Base Enclave, Prompt NCSA Enclave \\ \hline
DMS-REQ-0100 Generate Calibration Report Within Specified Time & Prompt Quality Control, Quality Control SW, Science Plugins, Prompt Base Enclave, Prompt NCSA Enclave \\ \hline
DMS-REQ-0101 Level 1 Calibration Report Definition & OCS-Driven Batch, Prompt Quality Control, OCS Batch SW, Quality Control SW, Calibration SW, Science Plugins, Prompt Base Enclave, Prompt NCSA Enclave \\ \hline
DMS-REQ-0102 Provide Engineering and Facility Database Archive & Archiving, DBB Ingest/Metadata Management, DBB Transport/Replication/Backup, DBB Storage, EFD Transformation, DBB Ingest/Metadata Management SW, DBB Transport/Replication/Backup SW, Archive Base Enclave, Archive NCSA Enclave, DAC Chile Enclave, DAC US Enclave, Prompt Base Enclave, Prompt NCSA Enclave \\ \hline
DMS-REQ-0103 Produce Images for EPO & Data Release Production, Offline Production Enclave \\ \hline
DMS-REQ-0106 Coadded Image Provenance & Data Release Production, Archive Base Enclave, Archive NCSA Enclave, Offline Production Enclave \\ \hline
DMS-REQ-0119 DAC resource allocation for Level 3 processing & LSP Portal, LSP JupyterLab, LSP Web API, SUIT, LSP JupyterLab SW, LSP Web API SW, DAC Chile Enclave, DAC US Enclave \\ \hline
DMS-REQ-0120 Level 3 Data Product Self Consistency & DBB Ingest/Metadata Management, LSP Web API, DBB Ingest/Metadata Management SW, LSP Web API SW, DAC Chile Enclave, DAC US Enclave \\ \hline
DMS-REQ-0121 Provenance for Level 3 processing at DACs & LSP Web API, LSP Web API SW, Data Butler, Task Framework, DAC Chile Enclave, DAC US Enclave \\ \hline
DMS-REQ-0122 Access to catalogs for external Level 3 processing & Bulk Distribution, DBB Ingest/Metadata Management, DBB Transport/Replication/Backup, DBB Storage, DBB Ingest/Metadata Management SW, DBB Transport/Replication/Backup SW \\ \hline
DMS-REQ-0123 Access to input catalogs for DAC-based Level 3 processing & Bulk Distribution, DBB Ingest/Metadata Management, DBB Transport/Replication/Backup, DBB Storage, LSP Portal, LSP JupyterLab, LSP Web API, DBB Ingest/Metadata Management SW, DBB Transport/Replication/Backup SW, SUIT, LSP JupyterLab SW, LSP Web API SW, ADQL Translator, DAC Chile Enclave, DAC US Enclave \\ \hline
DMS-REQ-0124 Federation with external catalogs & LSP Portal, LSP JupyterLab, LSP Web API, SUIT, LSP JupyterLab SW, LSP Web API SW \\ \hline
DMS-REQ-0125 Software framework for Level 3 catalog processing & Data Butler, Task Framework \\ \hline
DMS-REQ-0126 Access to images for external Level 3 processing & Bulk Distribution, DBB Ingest/Metadata Management, DBB Transport/Replication/Backup, DBB Storage, DBB Ingest/Metadata Management SW, DBB Transport/Replication/Backup SW \\ \hline
DMS-REQ-0127 Access to input images for DAC-based Level 3 processing & Bulk Distribution, DBB Ingest/Metadata Management, DBB Transport/Replication/Backup, DBB Storage, LSP Portal, LSP JupyterLab, LSP Web API, DBB Ingest/Metadata Management SW, DBB Transport/Replication/Backup SW, SUIT, LSP JupyterLab SW, LSP Web API SW, Image/Cutout Server, DAC Chile Enclave, DAC US Enclave \\ \hline
DMS-REQ-0128 Software framework for Level 3 image processing & Data Butler, Task Framework \\ \hline
DMS-REQ-0130 Calibration Data Products & DBB Ingest/Metadata Management, DBB Storage, DBB Ingest/Metadata Management SW, Science Plugins, Science Pipelines Distribution, Offline Production Enclave \\ \hline
DMS-REQ-0131 Calibration Images Available Within Specified Time & Prompt Processing Ingest, Prompt Processing, OCS-Driven Batch, DBB Ingest/Metadata Management, DBB Transport/Replication/Backup, DBB Storage, Image Ingest and Processing, OCS Batch SW, DBB Ingest/Metadata Management SW, DBB Transport/Replication/Backup SW, Calibration SW, Science Plugins, Archive Base Enclave, Archive NCSA Enclave, DAC Chile Enclave, DAC US Enclave, Offline Production Enclave, Prompt NCSA Enclave \\ \hline
DMS-REQ-0132 Calibration Image Provenance & DBB Ingest/Metadata Management, DBB Ingest/Metadata Management SW, Science Plugins, Science Pipelines Distribution, Offline Production Enclave \\ \hline
DMS-REQ-0155 Provide Data Access Services & LSP Web API, LSP Web API SW, ADQL Translator, Image/Cutout Server \\ \hline
DMS-REQ-0156 Provide Pipeline Execution Services & Batch Production, Campaign Management, Workload/Workflow Management \\ \hline
DMS-REQ-0158 Provide Pipeline Construction Services & Task Framework \\ \hline
DMS-REQ-0160 Provide User Interface Services & LSP Portal, SUIT \\ \hline
DMS-REQ-0161 Optimization of Cost, Reliability and Availability in Order & Alert Distribution, DBB Ingest/Metadata Management, DBB Transport/Replication/Backup, DBB Storage, LSP Portal, LSP JupyterLab, LSP Web API, Alert Distribution SW, DBB Ingest/Metadata Management SW, DBB Transport/Replication/Backup SW, SUIT, LSP JupyterLab SW, LSP Web API SW, Base Facility, NCSA Facility, Archive Base Enclave, Archive NCSA Enclave, Commissioning Cluster Enclave, DAC Chile Enclave, DAC US Enclave, Offline Production Enclave, Prompt Base Enclave, Prompt NCSA Enclave \\ \hline
DMS-REQ-0162 Pipeline Throughput & Archiving, Prompt Processing Ingest, Prompt Processing, OCS-Driven Batch, Alert Distribution, DBB Ingest/Metadata Management, DBB Transport/Replication/Backup, DBB Storage, Alert Distribution SW, OCS Batch SW, DBB Ingest/Metadata Management SW, DBB Transport/Replication/Backup SW, Base Facility, NCSA Facility, Archive Base Enclave, Archive NCSA Enclave, DAC Chile Enclave, DAC US Enclave, Offline Production Enclave, Prompt Base Enclave, Prompt NCSA Enclave \\ \hline
DMS-REQ-0163 Re-processing Capacity & Batch Production, DBB Ingest/Metadata Management, DBB Transport/Replication/Backup, DBB Storage, Campaign Management, Workload/Workflow Management, DBB Ingest/Metadata Management SW, DBB Transport/Replication/Backup SW, NCSA LAN Network, Base Facility, NCSA Facility, Archive Base Enclave, Archive NCSA Enclave, Offline Production Enclave \\ \hline
DMS-REQ-0164 Temporary Storage for Communications Links & DBB Ingest/Metadata Management, DBB Transport/Replication/Backup, DBB Storage, DBB Ingest/Metadata Management SW, DBB Transport/Replication/Backup SW, Base Facility, NCSA Facility, Prompt Base Enclave \\ \hline
DMS-REQ-0165 Infrastructure Sizing for "catching up" & Archiving, DBB Ingest/Metadata Management, DBB Transport/Replication/Backup, DBB Storage, EFD Transformation, Image Ingest and Processing, DBB Ingest/Metadata Management SW, DBB Transport/Replication/Backup SW, Summit to Base Network, Base to Archive Network, Base LAN Network, NCSA LAN Network, Base Facility, NCSA Facility, Prompt Base Enclave, Prompt NCSA Enclave \\ \hline
DMS-REQ-0166 Incorporate Fault-Tolerance & DBB Ingest/Metadata Management, DBB Transport/Replication/Backup, DBB Storage, DBB Ingest/Metadata Management SW, DBB Transport/Replication/Backup SW, Summit to Base Network, Base to Archive Network, Base LAN Network, NCSA LAN Network, Network Management, Base Facility, NCSA Facility, Archive Base Enclave, Archive NCSA Enclave, Commissioning Cluster Enclave, DAC Chile Enclave, DAC US Enclave, Offline Production Enclave, Prompt Base Enclave, Prompt NCSA Enclave \\ \hline
DMS-REQ-0167 Incorporate Autonomics & Archiving, Prompt Processing Ingest, Prompt Processing, Observatory Operations Data, Alert Distribution, Batch Production, DBB Ingest/Metadata Management, DBB Transport/Replication/Backup, DBB Storage, Alert Distribution SW, EFD Transformation, Image Ingest and Processing, Observatory Operations Data Service SW, Campaign Management, Workload/Workflow Management, DBB Ingest/Metadata Management SW, DBB Transport/Replication/Backup SW, Summit to Base Network, Base to Archive Network, Base LAN Network, NCSA LAN Network, Network Management, Base Facility, NCSA Facility, Archive Base Enclave, Archive NCSA Enclave, Commissioning Cluster Enclave, DAC Chile Enclave, DAC US Enclave, Offline Production Enclave, Prompt Base Enclave, Prompt NCSA Enclave \\ \hline
DMS-REQ-0168 Summit Facility Data Communications & Summit to Base Network \\ \hline
DMS-REQ-0170 Prefer Computing and Storage Down & Base Facility, NCSA Facility \\ \hline
DMS-REQ-0171 Summit to Base Network & Summit to Base Network \\ \hline
DMS-REQ-0172 Summit to Base Network Availability & Summit to Base Network \\ \hline
DMS-REQ-0173 Summit to Base Network Reliability & Summit to Base Network \\ \hline
DMS-REQ-0174 Summit to Base Network Secondary Link & Summit to Base Network \\ \hline
DMS-REQ-0175 Summit to Base Network Ownership and Operation & Summit to Base Network, Network Management \\ \hline
DMS-REQ-0176 Base Facility Infrastructure & DBB Ingest/Metadata Management, DBB Ingest/Metadata Management SW, Base Facility, Archive Base Enclave, Commissioning Cluster Enclave, DAC Chile Enclave, Prompt Base Enclave \\ \hline
DMS-REQ-0178 Base Facility Co-Location with Existing Facility & Base Facility \\ \hline
DMS-REQ-0180 Base to Archive Network & Base to Archive Network \\ \hline
DMS-REQ-0181 Base to Archive Network Availability & Base to Archive Network \\ \hline
DMS-REQ-0182 Base to Archive Network Reliability & Base to Archive Network \\ \hline
DMS-REQ-0183 Base to Archive Network Secondary Link & Base to Archive Network \\ \hline
DMS-REQ-0185 Archive Center & Bulk Distribution, DBB Ingest/Metadata Management, DBB Transport/Replication/Backup, DBB Storage, DBB Ingest/Metadata Management SW, DBB Transport/Replication/Backup SW, NCSA Facility, Archive NCSA Enclave, DAC US Enclave, Offline Production Enclave, Prompt NCSA Enclave \\ \hline
DMS-REQ-0186 Archive Center Disaster Recovery & Bulk Distribution, DBB Ingest/Metadata Management, DBB Transport/Replication/Backup, DBB Storage, DBB Ingest/Metadata Management SW, DBB Transport/Replication/Backup SW, NCSA Facility, Archive Base Enclave, Archive NCSA Enclave, DAC US Enclave, Offline Production Enclave \\ \hline
DMS-REQ-0187 Archive Center Co-Location with Existing Facility & NCSA Facility \\ \hline
DMS-REQ-0188 Archive to Data Access Center Network & Base to Archive Network, NCSA LAN Network \\ \hline
DMS-REQ-0189 Archive to Data Access Center Network Availability & Base to Archive Network, NCSA LAN Network \\ \hline
DMS-REQ-0190 Archive to Data Access Center Network Reliability & Base to Archive Network, NCSA LAN Network \\ \hline
DMS-REQ-0191 Archive to Data Access Center Network Secondary Link & Base to Archive Network, NCSA LAN Network \\ \hline
DMS-REQ-0193 Data Access Centers & Bulk Distribution, LSP Portal, LSP JupyterLab, LSP Web API, SUIT, LSP JupyterLab SW, LSP Web API SW, Base LAN Network, NCSA LAN Network, Base Facility, NCSA Facility, DAC Chile Enclave, DAC US Enclave \\ \hline
DMS-REQ-0194 Data Access Center Simultaneous Connections & LSP Portal, LSP JupyterLab, LSP Web API, SUIT, LSP JupyterLab SW, LSP Web API SW, Base LAN Network, NCSA LAN Network, DAC Chile Enclave, DAC US Enclave \\ \hline
DMS-REQ-0196 Data Access Center Geographical Distribution & Base Facility, NCSA Facility, DAC Chile Enclave, DAC US Enclave \\ \hline
DMS-REQ-0197 No Limit on Data Access Centers & DBB Ingest/Metadata Management, DBB Transport/Replication/Backup, DBB Storage, LSP Portal, LSP JupyterLab, LSP Web API, DBB Ingest/Metadata Management SW, DBB Transport/Replication/Backup SW, SUIT, LSP JupyterLab SW, LSP Web API SW \\ \hline
DMS-REQ-0265 Guider Calibration Data Acquisition & Archiving, OCS-Driven Batch, Image Ingest and Processing, OCS Batch SW, Calibration SW, Science Plugins, Science Pipelines Distribution, Prompt Base Enclave \\ \hline
DMS-REQ-0266 Exposure Catalog & DBB Ingest/Metadata Management, Header Service SW, DBB Ingest/Metadata Management SW, Alert Production, Archive Base Enclave, Archive NCSA Enclave, Prompt NCSA Enclave \\ \hline
DMS-REQ-0267 Source Catalog & Data Release Production, Distributed Database, Archive Base Enclave, Archive NCSA Enclave, Offline Production Enclave \\ \hline
DMS-REQ-0268 Forced-Source Catalog & Data Release Production, Distributed Database, Archive Base Enclave, Archive NCSA Enclave, Offline Production Enclave \\ \hline
DMS-REQ-0269 DIASource Catalog & DBB Ingest/Metadata Management, DBB Ingest/Metadata Management SW, Alert Production, Archive Base Enclave, Archive NCSA Enclave, Prompt NCSA Enclave \\ \hline
DMS-REQ-0270 Faint DIASource Measurements & Alert Production, Archive Base Enclave, Archive NCSA Enclave, Prompt NCSA Enclave \\ \hline
DMS-REQ-0271 DIAObject Catalog & DBB Ingest/Metadata Management, DBB Ingest/Metadata Management SW, Alert Production, Archive Base Enclave, Archive NCSA Enclave, Prompt NCSA Enclave \\ \hline
DMS-REQ-0272 DIAObject Attributes & Alert Production, Archive Base Enclave, Archive NCSA Enclave, Prompt NCSA Enclave \\ \hline
DMS-REQ-0273 SSObject Catalog & DBB Ingest/Metadata Management, DBB Ingest/Metadata Management SW, MOPS and Forced Photometry, Archive Base Enclave, Archive NCSA Enclave, Prompt NCSA Enclave \\ \hline
DMS-REQ-0274 Alert Content & Alert Production, Archive Base Enclave, Archive NCSA Enclave, Prompt NCSA Enclave \\ \hline
DMS-REQ-0275 Object Catalog & Data Release Production, Distributed Database, Archive Base Enclave, Archive NCSA Enclave, Offline Production Enclave \\ \hline
DMS-REQ-0276 Object Characterization & Data Release Production, Distributed Database \\ \hline
DMS-REQ-0277 Coadd Source Catalog & Data Release Production, Distributed Database, Offline Production Enclave \\ \hline
DMS-REQ-0278 Coadd Image Method Constraints & Data Release Production, Offline Production Enclave \\ \hline
DMS-REQ-0279 Deep Detection Coadds & Data Release Production, Offline Production Enclave \\ \hline
DMS-REQ-0280 Template Coadds & Data Release Production, Template Generation, Offline Production Enclave \\ \hline
DMS-REQ-0281 Multi-band Coadds & Data Release Production, Offline Production Enclave \\ \hline
DMS-REQ-0282 Dark Current Correction Frame & Science Plugins, Science Pipelines Distribution, Offline Production Enclave \\ \hline
DMS-REQ-0283 Fringe Correction Frame & Science Plugins, Science Pipelines Distribution, Offline Production Enclave \\ \hline
DMS-REQ-0284 Level-1 Production Completeness & Prompt Processing Ingest, Prompt Processing, Image Ingest and Processing, Archive Base Enclave, Archive NCSA Enclave, DAC Chile Enclave, DAC US Enclave, Offline Production Enclave, Prompt Base Enclave, Prompt NCSA Enclave \\ \hline
DMS-REQ-0285 Level 1 Source Association & Alert Production \\ \hline
DMS-REQ-0286 SSObject Precovery & MOPS and Forced Photometry, Offline Production Enclave \\ \hline
DMS-REQ-0287 DIASource Precovery & DBB Ingest/Metadata Management, DBB Transport/Replication/Backup, DBB Storage, LSP Web API, DBB Ingest/Metadata Management SW, DBB Transport/Replication/Backup SW, LSP Web API SW, MOPS and Forced Photometry, Archive Base Enclave, Archive NCSA Enclave, DAC Chile Enclave, DAC US Enclave, Offline Production Enclave \\ \hline
DMS-REQ-0288 Use of External Orbit Catalogs & Alert Production, MOPS and Forced Photometry \\ \hline
DMS-REQ-0289 Calibration Production Processing & OCS-Driven Batch, OCS Batch SW, Science Plugins, Science Pipelines Distribution, Offline Production Enclave \\ \hline
DMS-REQ-0290 Level 3 Data Import & LSP Web API, LSP Web API SW, Distributed Database \\ \hline
DMS-REQ-0291 Query Repeatability & DBB Ingest/Metadata Management, LSP Web API, DBB Ingest/Metadata Management SW, LSP Web API SW, Distributed Database, ADQL Translator, Archive Base Enclave, Archive NCSA Enclave, DAC Chile Enclave, DAC US Enclave \\ \hline
DMS-REQ-0292 Uniqueness of IDs Across Data Releases & DBB Ingest/Metadata Management, LSP Web API, DBB Ingest/Metadata Management SW, LSP Web API SW, Distributed Database \\ \hline
DMS-REQ-0293 Selection of Datasets & DBB Ingest/Metadata Management, LSP Web API, DBB Ingest/Metadata Management SW, LSP Web API SW, Data Butler, Distributed Database, Image/Cutout Server \\ \hline
DMS-REQ-0294 Processing of Datasets & Prompt Processing Ingest, Prompt Processing, OCS-Driven Batch, Batch Production, Image Ingest and Processing, OCS Batch SW, Campaign Management, Workload/Workflow Management, Task Framework \\ \hline
DMS-REQ-0295 Transparent Data Access & LSP Web API, LSP Web API SW, Data Butler \\ \hline
DMS-REQ-0296 Pre-cursor, and Real Data & Science Pipelines Libraries, Data Butler \\ \hline
DMS-REQ-0297 DMS Initialization Component & Base Facility, NCSA Facility \\ \hline
DMS-REQ-0298 Data Product and Raw Data Access & LSP Web API, LSP Web API SW, ADQL Translator, Image/Cutout Server \\ \hline
DMS-REQ-0299 Data Product Ingest & DBB Ingest/Metadata Management, DBB Ingest/Metadata Management SW, LSP Web API SW \\ \hline
DMS-REQ-0300 Bulk Download Service & Bulk Distribution \\ \hline
DMS-REQ-0301 Control of Level-1 Production & Archiving, Prompt Processing Ingest, Prompt Processing, OCS-Driven Batch, Image Ingest and Processing, OCS Batch SW \\ \hline
DMS-REQ-0302 Production Orchestration & Batch Production, Campaign Management, Workload/Workflow Management \\ \hline
DMS-REQ-0303 Production Monitoring & Batch Production, Campaign Management, Workload/Workflow Management \\ \hline
DMS-REQ-0304 Production Fault Tolerance & Batch Production, Campaign Management, Workload/Workflow Management, Task Framework \\ \hline
DMS-REQ-0305 Task Specification & Task Framework \\ \hline
DMS-REQ-0306 Task Configuration & Task Framework \\ \hline
DMS-REQ-0307 Unique Processing Coverage & Batch Production \\ \hline
DMS-REQ-0308 Software Architecture to Enable Community Re-Use & Batch Production, Workload/Workflow Management, Alert Production, Calibration SW, Data Release Production, MOPS and Forced Photometry, Template Generation, Science Plugins, Science Pipelines Distribution, Science Pipelines Libraries, Data Butler, Task Framework \\ \hline
DMS-REQ-0309 Raw Data Archiving Reliability & Archiving, DBB Ingest/Metadata Management, DBB Transport/Replication/Backup, DBB Storage, EFD Transformation, Image Ingest and Processing, DBB Ingest/Metadata Management SW, DBB Transport/Replication/Backup SW, Archive Base Enclave, Archive NCSA Enclave, DAC Chile Enclave, DAC US Enclave, Prompt Base Enclave, Prompt NCSA Enclave \\ \hline
DMS-REQ-0310 Un-Archived Data Product Cache & DBB Ingest/Metadata Management, DBB Lifetime Management, DBB Storage, DBB Ingest/Metadata Management SW, DBB Lifetime Management SW, Archive Base Enclave, Archive NCSA Enclave, DAC Chile Enclave, DAC US Enclave \\ \hline
DMS-REQ-0311 Regenerate Un-archived Data Products & LSP Web API, LSP Web API SW, Image/Cutout Server, DAC Chile Enclave, DAC US Enclave \\ \hline
DMS-REQ-0312 Level 1 Data Product Access & Prompt Processing, LSP Web API, LSP Web API SW, Alert Production, Archive Base Enclave, Archive NCSA Enclave, DAC Chile Enclave, DAC US Enclave \\ \hline
DMS-REQ-0313 Level 1 and 2 Catalog Access & DBB Ingest/Metadata Management, DBB Transport/Replication/Backup, DBB Storage, LSP Web API, DBB Ingest/Metadata Management SW, DBB Transport/Replication/Backup SW, LSP Web API SW, Distributed Database, DAC Chile Enclave, DAC US Enclave \\ \hline
DMS-REQ-0314 Compute Platform Heterogeneity & Archiving, Prompt Processing Ingest, Prompt Processing, Observatory Operations Data, OCS-Driven Batch, Telemetry Gateway, Alert Distribution, Prompt Quality Control, Batch Production, Bulk Distribution, DBB Ingest/Metadata Management, DBB Transport/Replication/Backup, DBB Storage, LSP Portal, LSP JupyterLab, LSP Web API, Alert Distribution SW, EFD Transformation, Image Ingest and Processing, Observatory Operations Data Service SW, OCS Batch SW, Campaign Management, Workload/Workflow Management, Quality Control SW, DBB Ingest/Metadata Management SW, DBB Transport/Replication/Backup SW, SUIT, LSP JupyterLab SW, LSP Web API SW, Data Butler, Task Framework, Base Facility, NCSA Facility, Archive Base Enclave, Archive NCSA Enclave, Commissioning Cluster Enclave, DAC Chile Enclave, DAC US Enclave, Offline Production Enclave, Prompt Base Enclave, Prompt NCSA Enclave \\ \hline
DMS-REQ-0315 DMS Communication with OCS & Archiving, Prompt Processing Ingest, OCS-Driven Batch, Telemetry Gateway, EFD Transformation, Image Ingest and Processing, OCS Batch SW, Base Facility, Prompt Base Enclave \\ \hline
DMS-REQ-0316 Commissioning Cluster & Base Facility, Commissioning Cluster Enclave \\ \hline
DMS-REQ-0317 DIAForcedSource Catalog & DBB Ingest/Metadata Management, DBB Ingest/Metadata Management SW, Alert Production, MOPS and Forced Photometry, Archive Base Enclave, Archive NCSA Enclave, Prompt NCSA Enclave \\ \hline
DMS-REQ-0318 Data Management Unscheduled Downtime & Archiving, Prompt Processing Ingest, Prompt Processing, Observatory Operations Data, OCS-Driven Batch, Telemetry Gateway, Alert Distribution, Prompt Quality Control, Batch Production, Bulk Distribution, DBB Ingest/Metadata Management, DBB Transport/Replication/Backup, DBB Storage, LSP Portal, LSP JupyterLab, LSP Web API, Alert Distribution SW, EFD Transformation, Image Ingest and Processing, Observatory Operations Data Service SW, OCS Batch SW, Campaign Management, Workload/Workflow Management, Quality Control SW, DBB Ingest/Metadata Management SW, DBB Transport/Replication/Backup SW, SUIT, LSP JupyterLab SW, LSP Web API SW, Base Facility, NCSA Facility, Archive Base Enclave, Archive NCSA Enclave, Commissioning Cluster Enclave, DAC Chile Enclave, DAC US Enclave, Offline Production Enclave, Prompt Base Enclave, Prompt NCSA Enclave \\ \hline
DMS-REQ-0319 Characterizing Variability & Alert Production, MOPS and Forced Photometry, Prompt NCSA Enclave \\ \hline
DMS-REQ-0320 Processing of Data From Special Programs & Data Release Production, Special Programs Productions, Task Framework, Offline Production Enclave, Prompt NCSA Enclave \\ \hline
DMS-REQ-0321 Level 1 Processing of Special Programs Data & Prompt Processing, Image Ingest and Processing, Alert Production, MOPS and Forced Photometry, Prompt NCSA Enclave \\ \hline
DMS-REQ-0322 Special Programs Database & DBB Ingest/Metadata Management, LSP Web API, DBB Ingest/Metadata Management SW, LSP Web API SW, Special Programs Productions, Distributed Database, ADQL Translator, Archive Base Enclave, Archive NCSA Enclave, DAC Chile Enclave, DAC US Enclave \\ \hline
DMS-REQ-0323 Calculating SSObject Parameters & LSP Web API, LSP Web API SW, ADQL Translator, DAC Chile Enclave, DAC US Enclave \\ \hline
DMS-REQ-0324 Matching DIASources to Objects & LSP Web API, LSP Web API SW, Alert Production, Archive Base Enclave, Archive NCSA Enclave, DAC Chile Enclave, DAC US Enclave \\ \hline
DMS-REQ-0325 Regenerating L1 Data Products During Data Release Processing & Data Release Production, Offline Production Enclave \\ \hline
DMS-REQ-0326 Storing Approximations of Per-pixel Metadata & Data Release Production \\ \hline
DMS-REQ-0327 Background Model Calculation & Alert Production, Archive Base Enclave, Archive NCSA Enclave, Prompt NCSA Enclave \\ \hline
DMS-REQ-0328 Documenting Image Characterization & Alert Production, Prompt NCSA Enclave \\ \hline
DMS-REQ-0329 All-Sky Visualization of Data Releases & Data Release Production, Offline Production Enclave \\ \hline
DMS-REQ-0330 Best Seeing Coadds & Data Release Production, Offline Production Enclave \\ \hline
DMS-REQ-0331 Computing Derived Quantities & LSP Web API, LSP Web API SW, Data Release Production, Distributed Database, ADQL Translator \\ \hline
DMS-REQ-0332 Denormalizing Database Tables & LSP Web API, LSP Web API SW, Distributed Database \\ \hline
DMS-REQ-0333 Maximum Likelihood Values and Covariances & Alert Production, Data Release Production, Distributed Database \\ \hline
DMS-REQ-0334 Persisting Data Products & DBB Ingest/Metadata Management, DBB Lifetime Management, DBB Storage, LSP Web API, DBB Ingest/Metadata Management SW, DBB Lifetime Management SW, LSP Web API SW, Image/Cutout Server, Archive Base Enclave, Archive NCSA Enclave, DAC Chile Enclave, DAC US Enclave, Offline Production Enclave \\ \hline
DMS-REQ-0335 PSF-Matched Coadds & Data Release Production, Offline Production Enclave \\ \hline
DMS-REQ-0336b Regenerating Data Products from Previous Data Releases & LSP Web API, LSP Web API SW, Image/Cutout Server, DAC Chile Enclave, DAC US Enclave \\ \hline
DMS-REQ-0337 Detecting faint variable objects & Data Release Production \\ \hline
DMS-REQ-0338 Targeted Coadds & DBB Ingest/Metadata Management, DBB Lifetime Management, DBB Storage, LSP Web API, DBB Ingest/Metadata Management SW, DBB Lifetime Management SW, LSP Web API SW, Image/Cutout Server, Archive Base Enclave, Archive NCSA Enclave, DAC Chile Enclave, DAC US Enclave \\ \hline
DMS-REQ-0339 Tracking Characterization Changes Between Data Releases & DBB Ingest/Metadata Management, DBB Lifetime Management, LSP Web API, DBB Ingest/Metadata Management SW, DBB Lifetime Management SW, LSP Web API SW, Image/Cutout Server, DAC Chile Enclave, DAC US Enclave \\ \hline
DMS-REQ-0340 Access Controls of Level 3 Data Products & Distributed Database, ADQL Translator, DAC Chile Enclave, DAC US Enclave \\ \hline
DMS-REQ-0341 Providing a Precovery Service & LSP Portal, SUIT, MOPS and Forced Photometry, DAC Chile Enclave, DAC US Enclave, Offline Production Enclave \\ \hline
DMS-REQ-0342 Alert Filtering Service & Alert Distribution, Alert Distribution SW \\ \hline
DMS-REQ-0343 Performance Requirements for LSST Alert Filtering Service & Alert Distribution, Alert Distribution SW, Prompt NCSA Enclave \\ \hline
DMS-REQ-0344 Constraints on Level 1 Special Program Products Generation & Prompt Processing, DBB Transport/Replication/Backup, DBB Storage, LSP Web API, Image Ingest and Processing, DBB Transport/Replication/Backup SW, LSP Web API SW, Alert Production, MOPS and Forced Photometry, Archive Base Enclave, Archive NCSA Enclave, DAC Chile Enclave, DAC US Enclave, Prompt NCSA Enclave \\ \hline
DMS-REQ-0345 Logging of catalog queries & LSP Web API, LSP Web API SW, Distributed Database, ADQL Translator, DAC Chile Enclave, DAC US Enclave \\ \hline
DMS-REQ-0346 Data Availability & Archiving, Prompt Processing, Observatory Operations Data, Bulk Distribution, DBB Ingest/Metadata Management, DBB Lifetime Management, DBB Storage, LSP Web API, EFD Transformation, Image Ingest and Processing, Observatory Operations Data Service SW, DBB Ingest/Metadata Management SW, DBB Lifetime Management SW, LSP Web API SW, Image/Cutout Server \\ \hline
DMS-REQ-0347 Measurements in catalogs & LSP Web API, LSP Web API SW, Alert Production, Data Release Production, Distributed Database \\ \hline
DMS-REQ-0348 Pre-defined alert filters & Alert Distribution, Alert Distribution SW \\ \hline
DMS-REQ-0349 Detecting extended  low surface brightness objects & Data Release Production \\ \hline
DMS-REQ-0350 Associating Objects across data releases & Data Release Production, Distributed Database \\ \hline
DMS-REQ-0351 Provide Beam Projector Coordinate Calculation Software & Science Pipelines Libraries \\ \hline
DMS-REQ-0352 Base Wireless LAN (WiFi) & Base LAN Network, Base Facility \\ \hline
DMS-REQ-0353 Publishing predicted visit schedule & Planned Observation Publication, Planned Observation Publication SW, Prompt Base Enclave \\ \hline
DMS-REQ-0363 Access to Previous Data Releases & DBB Ingest/Metadata Management, DBB Storage, LSP Portal, LSP Web API, DBB Ingest/Metadata Management SW, SUIT, LSP Web API SW, Distributed Database, Archive Base Enclave, Archive NCSA Enclave, DAC Chile Enclave, DAC US Enclave \\ \hline
DMS-REQ-0364 Data Access Services & DBB Ingest/Metadata Management, DBB Storage, LSP Portal, LSP Web API, DBB Ingest/Metadata Management SW, SUIT, LSP Web API SW, ADQL Translator, Archive Base Enclave, Archive NCSA Enclave, DAC Chile Enclave, DAC US Enclave \\ \hline
DMS-REQ-0365 Operations Subsets & DBB Ingest/Metadata Management, DBB Storage, LSP Portal, LSP Web API, DBB Ingest/Metadata Management SW, SUIT, LSP Web API SW, Distributed Database, Archive Base Enclave, Archive NCSA Enclave, DAC Chile Enclave, DAC US Enclave \\ \hline
DMS-REQ-0366 Subsets Support & DBB Ingest/Metadata Management, DBB Transport/Replication/Backup, DBB Storage, LSP Portal, LSP Web API, DBB Ingest/Metadata Management SW, DBB Transport/Replication/Backup SW, SUIT, LSP Web API SW, Distributed Database, Archive Base Enclave, Archive NCSA Enclave, DAC Chile Enclave, DAC US Enclave \\ \hline
DMS-REQ-0367 Access Services Performance & DBB Storage, LSP Portal, LSP Web API, SUIT, LSP Web API SW, Distributed Database, Archive Base Enclave, Archive NCSA Enclave, DAC Chile Enclave, DAC US Enclave \\ \hline
DMS-REQ-0368 Implementation Provisions & DBB Storage, LSP Portal, LSP Web API, SUIT, LSP Web API SW, Distributed Database, ADQL Translator, Image/Cutout Server, Archive Base Enclave, Archive NCSA Enclave, DAC Chile Enclave, DAC US Enclave \\ \hline
DMS-REQ-0369 Evolution & DBB Ingest/Metadata Management, LSP Portal, LSP Web API, DBB Ingest/Metadata Management SW, SUIT, LSP Web API SW, Distributed Database, ADQL Translator, Archive Base Enclave, Archive NCSA Enclave, DAC Chile Enclave, DAC US Enclave \\ \hline
DMS-REQ-0370 Older Release Behavior & DBB Ingest/Metadata Management, DBB Transport/Replication/Backup, DBB Storage, LSP Portal, LSP Web API, DBB Ingest/Metadata Management SW, DBB Transport/Replication/Backup SW, SUIT, LSP Web API SW, Distributed Database, Archive Base Enclave, Archive NCSA Enclave, DAC Chile Enclave, DAC US Enclave \\ \hline
DMS-REQ-0371 Query Availability & LSP Portal, LSP Web API, SUIT, LSP Web API SW, Distributed Database \\ \hline
DMS-REQ-0372a Archiving Camera Test Data & DBB Ingest/Metadata Management, DBB Storage \\ \hline
DMS-REQ-0379 Produce All-Sky HiPS Map & Data Release Production \\ \hline
DMS-REQ-0380 HiPS Service & LSP Web API \\ \hline
DMS-REQ-0381 HiPS Linkage to Coadds & LSP Web API, Data Release Production \\ \hline
DMS-REQ-0382 HiPS Visualization & LSP Portal, SUIT \\ \hline
DMS-REQ-0383 Produce MOC Maps & Data Release Production \\ \hline
DMS-REQ-0384 Export MOCs As FITS & LSP Web API \\ \hline
DMS-REQ-0385 MOC Visualization & LSP Portal \\ \hline
DMS-REQ-0386a Archive Processing Provenance & DBB Ingest/Metadata Management, DBB Storage, Workload/Workflow Management \\ \hline
DMS-REQ-0387b Serve Archived Provenance & DBB Ingest/Metadata Management, DBB Storage, LSP Web API \\ \hline
DMS-REQ-0388 Provide Re-Run Tools & Batch Production, DBB Ingest/Metadata Management, Workload/Workflow Management \\ \hline
DMS-REQ-0389 Re-Runs on Similar Systems & Batch Production, Workload/Workflow Management \\ \hline
DMS-REQ-0390 Re-Runs on Other Systems & Batch Production, Workload/Workflow Management \\ \hline
EP-DM-CON-ICD-0002 EPO is an Authorized Science User & DAC US Enclave \\ \hline
EP-DM-CON-ICD-0004 DM Transfer of Catalog Data to EPO & Bulk Distribution \\ \hline
EP-DM-CON-ICD-0009 Catalog Format & Bulk Distribution \\ \hline
EP-DM-CON-ICD-0019 DM to EPO Data Transfer Cadence & Bulk Distribution \\ \hline
EP-DM-CON-ICD-0021 DM Generation of a Color Hierarchical Progressive Survey for EPO & Bulk Distribution \\ \hline
EP-DM-CON-ICD-0034 Citizen Science Data & LSP Web API, LSP Web API SW, ADQL Translator, Image/Cutout Server, DAC US Enclave \\ \hline
EP-DM-CON-ICD-0035 DM Software & Data Butler, Task Framework \\ \hline
EP-DM-CON-ICD-0036 DM Services & LSP Web API \\ \hline
EP-DM-CON-ICD-0037 EPO Compute Cluster & Offline Production Enclave \\ \hline
OCS-DM-COM-ICD-0003 Data Management CSC Command Response Model & Archiving, Prompt Processing Ingest, OCS-Driven Batch, EFD Transformation, Header Service SW, Image Ingest and Processing, OCS Batch SW, Prompt Base Enclave \\ \hline
OCS-DM-COM-ICD-0004 Data Management Exposed CSCs & Archiving, Prompt Processing Ingest, EFD Transformation, Image Ingest and Processing, Prompt Base Enclave \\ \hline
OCS-DM-COM-ICD-0005 Main Camera Archiver & Archiving, Image Ingest and Processing, Prompt Base Enclave \\ \hline
OCS-DM-COM-ICD-0006 Catch-up Archiver & Archiving, Image Ingest and Processing, Prompt Base Enclave \\ \hline
OCS-DM-COM-ICD-0007 Prompt Processing CSC & Prompt Processing Ingest, Prompt Processing, Image Ingest and Processing, Prompt Base Enclave \\ \hline
OCS-DM-COM-ICD-0008 EFD Transformation Service CSC & Archiving, EFD Transformation, Prompt Base Enclave \\ \hline
OCS-DM-COM-ICD-0009 Command Set Implementation by Data Management & Archiving, Prompt Processing Ingest, OCS-Driven Batch, Header Service SW, Image Ingest and Processing, OCS Batch SW, Prompt Base Enclave \\ \hline
OCS-DM-COM-ICD-0012 Start Command & Archiving, Prompt Processing Ingest, OCS-Driven Batch, Header Service SW, Image Ingest and Processing, OCS Batch SW \\ \hline
OCS-DM-COM-ICD-0013 configure Successful Completion Response & Archiving, Prompt Processing Ingest, OCS-Driven Batch, Header Service SW, Image Ingest and Processing, OCS Batch SW \\ \hline
OCS-DM-COM-ICD-0014 enable Command & Archiving, Prompt Processing Ingest, OCS-Driven Batch, Header Service SW, Image Ingest and Processing, OCS Batch SW \\ \hline
OCS-DM-COM-ICD-0015 disable Command & Archiving, Prompt Processing Ingest, OCS-Driven Batch, Header Service SW, Image Ingest and Processing, OCS Batch SW \\ \hline
OCS-DM-COM-ICD-0017 Data Management Telemetry Interface Model & Telemetry Gateway, Prompt Base Enclave \\ \hline
OCS-DM-COM-ICD-0018 Data Management Telemetry Time Stamp & Telemetry Gateway, Prompt Base Enclave \\ \hline
OCS-DM-COM-ICD-0019 Data Management Events and Telemetry Required by the OCS & Telemetry Gateway, Prompt Base Enclave \\ \hline
OCS-DM-COM-ICD-0020 Image and Visit Processing and Archiving Status & Telemetry Gateway, Prompt Base Enclave \\ \hline
OCS-DM-COM-ICD-0021 Data Quality Metrics & Telemetry Gateway, Prompt Base Enclave \\ \hline
OCS-DM-COM-ICD-0022 System Health Metrics & Telemetry Gateway, Prompt Base Enclave \\ \hline
OCS-DM-COM-ICD-0025 Expected Load of Queries from DM & Archiving, EFD Transformation, Prompt Base Enclave \\ \hline
OCS-DM-COM-ICD-0026 Engineering and Facilities Database Archiving by Data Management & Archiving, EFD Transformation, Prompt Base Enclave \\ \hline
OCS-DM-COM-ICD-0027 Multiple Physically Separated Copies & Archiving, EFD Transformation, Base Facility, NCSA Facility, Archive Base Enclave, Archive NCSA Enclave, Prompt Base Enclave \\ \hline
OCS-DM-COM-ICD-0028 Expected Data Volume & Archiving, EFD Transformation, Archive Base Enclave, Archive NCSA Enclave, Prompt Base Enclave \\ \hline
OCS-DM-COM-ICD-0029 Archive Latency & Archiving, EFD Transformation, Archive Base Enclave, Archive NCSA Enclave, DAC Chile Enclave, DAC US Enclave \\ \hline
OCS-DM-COM-ICD-0030 EFD Transformation Service Interface & Archiving, EFD Transformation, Archive Base Enclave, Archive NCSA Enclave, Prompt Base Enclave \\ \hline
OCS-DM-COM-ICD-0032 Auxiliary Telescope Archiver CSC & Archiving, Image Ingest and Processing \\ \hline
OCS-DM-COM-ICD-0033 Header Service CSC & Archiving, Header Service SW \\ \hline
OCS-DM-COM-ICD-0034 Auxiliary Header Service CSC & Archiving, Header Service SW \\ \hline
OCS-DM-COM-ICD-0035 OCS-Driven Batch CSC & OCS-Driven Batch, OCS Batch SW \\ \hline
OCS-DM-COM-ICD-0036 standby Command & Archiving, Prompt Processing Ingest, OCS-Driven Batch, Header Service SW, Image Ingest and Processing, OCS Batch SW \\ \hline
OCS-DM-COM-ICD-0037 exit Command & Archiving, Prompt Processing Ingest, OCS-Driven Batch, Header Service SW, Image Ingest and Processing, OCS Batch SW \\ \hline
OCS-DM-COM-ICD-0038 enterControl Command & Archiving, Prompt Processing Ingest, OCS-Driven Batch, Header Service SW, Image Ingest and Processing, OCS Batch SW \\ \hline
OCS-DM-COM-ICD-0039 enterControl Successful Completion Response & Archiving, Prompt Processing Ingest, OCS-Driven Batch, Header Service SW, Image Ingest and Processing, OCS Batch SW \\ \hline
OCS-DM-COM-ICD-0040 Command Completion Response & Archiving, Prompt Processing Ingest, OCS-Driven Batch, Header Service SW, Image Ingest and Processing, OCS Batch SW \\ \hline
OCS-DM-COM-ICD-0041 Large File Annex Replication Interface & Archiving \\ \hline
OCS-DM-COM-ICD-0042 EFD Disaster Recovery by Data Management & Archiving \\ \hline
OCS-DM-COM-ICD-0043 Image Retrieval for Archiving Event & Archiving, Telemetry Gateway, Image Ingest and Processing \\ \hline
OCS-DM-COM-ICD-0044 Image Retrieval For Processing Event & Archiving, Telemetry Gateway, Image Ingest and Processing \\ \hline
OCS-DM-COM-ICD-0045 Image in OODS Event & Archiving, Telemetry Gateway, Image Ingest and Processing \\ \hline
OCS-DM-COM-ICD-0046 Image Forwarded Event & Prompt Processing Ingest, Telemetry Gateway \\ \hline
OCS-DM-COM-ICD-0047 Image Archived Event & Telemetry Gateway, DBB Ingest/Metadata Management, DBB Ingest/Metadata Management SW \\ \hline
OCS-DM-COM-ICD-0048 Alert Production Complete Event & Prompt Processing, Telemetry Gateway, Alert Distribution SW \\ \hline
OCS-DM-COM-ICD-0049 WCS Information & Prompt Processing, Telemetry Gateway, Alert Production \\ \hline
OCS-DM-COM-ICD-0050 PSF Information & Prompt Processing, Telemetry Gateway, Alert Production \\ \hline
OCS-DM-COM-ICD-0051 Photometric Zeropoint Information & Prompt Processing, Telemetry Gateway, Alert Production \\ \hline
OCS-DM-COM-ICD-0052 Number of Alerts Information & Prompt Processing, Telemetry Gateway, Alert Production \\ \hline
OCS-DM-COM-ICD-0053 Summit-Base Network Utilization & Telemetry Gateway, Summit to Base Network \\ \hline
OCS-DM-COM-ICD-0054 Base-Archive Network Utilization & Telemetry Gateway, Base to Archive Network \\ \hline
OCS-DM-COM-ICD-0055 Archiver Resource Availability & Archiving, Telemetry Gateway, Image Ingest and Processing \\ \hline
OCS-DM-COM-ICD-0056 Prompt Processing Resource Availability & Prompt Processing Ingest, Prompt Processing, Telemetry Gateway, Image Ingest and Processing \\ \hline
OCS-EFD-HS-0001 Fulfill requirements of a Commandable SAL Component (CSC) & Archiving, Header Service SW \\ \hline
OCS-EFD-HS-0002 Critical System & Archiving, Header Service SW \\ \hline
OCS-EFD-HS-0003 Write Headers for all images taken by all Cameras supported by LSST & Archiving, Header Service SW \\ \hline
OCS-EFD-HS-0004 Ability to capture metadata at the beginning of exposure & Archiving, Header Service SW \\ \hline
OCS-EFD-HS-0005 Ability to capture metadata during of exposure integration & Archiving, Header Service SW \\ \hline
OCS-EFD-HS-0006 Ability to capture metadata at end of readout & Archiving, Header Service SW \\ \hline
OCS-EFD-HS-0007 Write header and Publish Event after end of telemetry event & Archiving, Header Service SW \\ \hline
OCS-EFD-HS-0008 Write header and Publish Event within specified time of the end-of-telemetry Event & Archiving, Header Service SW \\ \hline
OCS-EFD-HS-0009 Adherence to the FITS Standard & Archiving, Header Service SW \\ \hline
OCS-EFD-HS-0010 Configuration of Header Keywords and source & Archiving, Header Service SW \\ \hline
OCS-EFD-HS-0011 Produce header even if some meta-data not avaiable & Archiving, Header Service SW \\ \hline
OCS-EFD-HS-0012 Publish an Event if monitoring detects any failure of the service. & Archiving, Header Service SW \\ \hline
OCS-EFD-HS-0013 Extract metadata from published configuration & Archiving, Header Service SW \\ \hline
OCS-EFD-HS-0014 Metadata Capture & Archiving, Header Service SW \\ \hline
OCS-EFD-HS-0015 Generate on-the-fly additional metadata as approved by the Project CCB. & Archiving, Header Service SW \\ \hline

\end{longtable}
\normalsize

\subsection{Component to Requirement
Traceability}\label{component-to-requirement-traceability}

Note that only ``leaf'' components are traced to requirements.

\setitemize{noitemsep,topsep=0pt,parsep=0pt,partopsep=0pt}
\footnotesize

Archiving \begin{itemize}
\item CA-DM-CON-ICD-0014 Provide science sensor data
\item CA-DM-CON-ICD-0015 Provide wavefront sensor data
\item CA-DM-CON-ICD-0016 Provide guide sensor data
\item CA-DM-CON-ICD-0017 Data Management load on image data interfaces
\item CA-DM-CON-ICD-0019 Camera engineering image data archiving
\item DMS-REQ-0008 Pipeline Availability
\item DMS-REQ-0018 Raw Science Image Data Acquisition
\item DMS-REQ-0020 Wavefront Sensor Data Acquisition
\item DMS-REQ-0024 Raw Image Assembly
\item DMS-REQ-0068 Raw Science Image Metadata
\item DMS-REQ-0099 Level 1 Performance Report Definition
\item DMS-REQ-0102 Provide Engineering and Facility Database Archive
\item DMS-REQ-0162 Pipeline Throughput
\item DMS-REQ-0165 Infrastructure Sizing for "catching up"
\item DMS-REQ-0167 Incorporate Autonomics
\item DMS-REQ-0265 Guider Calibration Data Acquisition
\item DMS-REQ-0301 Control of Level-1 Production
\item DMS-REQ-0309 Raw Data Archiving Reliability
\item DMS-REQ-0314 Compute Platform Heterogeneity
\item DMS-REQ-0315 DMS Communication with OCS
\item DMS-REQ-0318 Data Management Unscheduled Downtime
\item DMS-REQ-0346 Data Availability
\item OCS-DM-COM-ICD-0003 Data Management CSC Command Response Model
\item OCS-DM-COM-ICD-0004 Data Management Exposed CSCs
\item OCS-DM-COM-ICD-0005 Main Camera Archiver
\item OCS-DM-COM-ICD-0006 Catch-up Archiver
\item OCS-DM-COM-ICD-0008 EFD Transformation Service CSC
\item OCS-DM-COM-ICD-0009 Command Set Implementation by Data Management
\item OCS-DM-COM-ICD-0012 Start Command
\item OCS-DM-COM-ICD-0013 configure Successful Completion Response
\item OCS-DM-COM-ICD-0014 enable Command
\item OCS-DM-COM-ICD-0015 disable Command
\item OCS-DM-COM-ICD-0025 Expected Load of Queries from DM
\item OCS-DM-COM-ICD-0026 Engineering and Facilities Database Archiving by Data Management
\item OCS-DM-COM-ICD-0027 Multiple Physically Separated Copies
\item OCS-DM-COM-ICD-0028 Expected Data Volume
\item OCS-DM-COM-ICD-0029 Archive Latency
\item OCS-DM-COM-ICD-0030 EFD Transformation Service Interface
\item OCS-DM-COM-ICD-0032 Auxiliary Telescope Archiver CSC
\item OCS-DM-COM-ICD-0033 Header Service CSC
\item OCS-DM-COM-ICD-0034 Auxiliary Header Service CSC
\item OCS-DM-COM-ICD-0036 standby Command
\item OCS-DM-COM-ICD-0037 exit Command
\item OCS-DM-COM-ICD-0038 enterControl Command
\item OCS-DM-COM-ICD-0039 enterControl Successful Completion Response
\item OCS-DM-COM-ICD-0040 Command Completion Response
\item OCS-DM-COM-ICD-0041 Large File Annex Replication Interface
\item OCS-DM-COM-ICD-0042 EFD Disaster Recovery by Data Management
\item OCS-DM-COM-ICD-0043 Image Retrieval for Archiving Event
\item OCS-DM-COM-ICD-0044 Image Retrieval For Processing Event
\item OCS-DM-COM-ICD-0045 Image in OODS Event
\item OCS-DM-COM-ICD-0055 Archiver Resource Availability
\item OCS-EFD-HS-0001 Fulfill requirements of a Commandable SAL Component (CSC)
\item OCS-EFD-HS-0002 Critical System
\item OCS-EFD-HS-0003 Write Headers for all images taken by all Cameras supported by LSST
\item OCS-EFD-HS-0004 Ability to capture metadata at the beginning of exposure
\item OCS-EFD-HS-0005 Ability to capture metadata during of exposure integration
\item OCS-EFD-HS-0006 Ability to capture metadata at end of readout
\item OCS-EFD-HS-0007 Write header and Publish Event after end of telemetry event
\item OCS-EFD-HS-0008 Write header and Publish Event within specified time of the end-of-telemetry Event
\item OCS-EFD-HS-0009 Adherence to the FITS Standard
\item OCS-EFD-HS-0010 Configuration of Header Keywords and source
\item OCS-EFD-HS-0011 Produce header even if some meta-data not avaiable
\item OCS-EFD-HS-0012 Publish an Event if monitoring detects any failure of the service.
\item OCS-EFD-HS-0013 Extract metadata from published configuration
\item OCS-EFD-HS-0014 Metadata Capture
\item OCS-EFD-HS-0015 Generate on-the-fly additional metadata as approved by the Project CCB.
\end{itemize}
Planned Observation Publication \begin{itemize}
\item DMS-REQ-0353 Publishing predicted visit schedule
\end{itemize}
Prompt Processing Ingest \begin{itemize}
\item CA-DM-CON-ICD-0007 Provide Data Management Conditions data
\item CA-DM-CON-ICD-0008 Data Management Conditions data latency
\item CA-DM-CON-ICD-0014 Provide science sensor data
\item CA-DM-CON-ICD-0015 Provide wavefront sensor data
\item CA-DM-CON-ICD-0016 Provide guide sensor data
\item CA-DM-CON-ICD-0017 Data Management load on image data interfaces
\item DM-TS-CON-ICD-0002 Timing
\item DM-TS-CON-ICD-0007 Timing
\item DMS-REQ-0004 Nightly Data Accessible Within 24 hrs
\item DMS-REQ-0008 Pipeline Availability
\item DMS-REQ-0022 Crosstalk Corrected Science Image Data Acquisition
\item DMS-REQ-0099 Level 1 Performance Report Definition
\item DMS-REQ-0131 Calibration Images Available Within Specified Time
\item DMS-REQ-0162 Pipeline Throughput
\item DMS-REQ-0167 Incorporate Autonomics
\item DMS-REQ-0284 Level-1 Production Completeness
\item DMS-REQ-0294 Processing of Datasets
\item DMS-REQ-0301 Control of Level-1 Production
\item DMS-REQ-0314 Compute Platform Heterogeneity
\item DMS-REQ-0315 DMS Communication with OCS
\item DMS-REQ-0318 Data Management Unscheduled Downtime
\item OCS-DM-COM-ICD-0003 Data Management CSC Command Response Model
\item OCS-DM-COM-ICD-0004 Data Management Exposed CSCs
\item OCS-DM-COM-ICD-0007 Prompt Processing CSC
\item OCS-DM-COM-ICD-0009 Command Set Implementation by Data Management
\item OCS-DM-COM-ICD-0012 Start Command
\item OCS-DM-COM-ICD-0013 configure Successful Completion Response
\item OCS-DM-COM-ICD-0014 enable Command
\item OCS-DM-COM-ICD-0015 disable Command
\item OCS-DM-COM-ICD-0036 standby Command
\item OCS-DM-COM-ICD-0037 exit Command
\item OCS-DM-COM-ICD-0038 enterControl Command
\item OCS-DM-COM-ICD-0039 enterControl Successful Completion Response
\item OCS-DM-COM-ICD-0040 Command Completion Response
\item OCS-DM-COM-ICD-0046 Image Forwarded Event
\item OCS-DM-COM-ICD-0056 Prompt Processing Resource Availability
\end{itemize}
Prompt Processing \begin{itemize}
\item CA-DM-CON-ICD-0007 Provide Data Management Conditions data
\item CA-DM-CON-ICD-0008 Data Management Conditions data latency
\item DM-TS-CON-ICD-0002 Timing
\item DM-TS-CON-ICD-0006 Data
\item DM-TS-CON-ICD-0007 Timing
\item DM-TS-CON-ICD-0011 Data Format
\item DMS-REQ-0002 Transient Alert Distribution
\item DMS-REQ-0004 Nightly Data Accessible Within 24 hrs
\item DMS-REQ-0008 Pipeline Availability
\item DMS-REQ-0099 Level 1 Performance Report Definition
\item DMS-REQ-0131 Calibration Images Available Within Specified Time
\item DMS-REQ-0162 Pipeline Throughput
\item DMS-REQ-0167 Incorporate Autonomics
\item DMS-REQ-0284 Level-1 Production Completeness
\item DMS-REQ-0294 Processing of Datasets
\item DMS-REQ-0301 Control of Level-1 Production
\item DMS-REQ-0312 Level 1 Data Product Access
\item DMS-REQ-0314 Compute Platform Heterogeneity
\item DMS-REQ-0318 Data Management Unscheduled Downtime
\item DMS-REQ-0321 Level 1 Processing of Special Programs Data
\item DMS-REQ-0344 Constraints on Level 1 Special Program Products Generation
\item DMS-REQ-0346 Data Availability
\item OCS-DM-COM-ICD-0007 Prompt Processing CSC
\item OCS-DM-COM-ICD-0048 Alert Production Complete Event
\item OCS-DM-COM-ICD-0049 WCS Information
\item OCS-DM-COM-ICD-0050 PSF Information
\item OCS-DM-COM-ICD-0051 Photometric Zeropoint Information
\item OCS-DM-COM-ICD-0052 Number of Alerts Information
\item OCS-DM-COM-ICD-0056 Prompt Processing Resource Availability
\end{itemize}
Observatory Operations Data \begin{itemize}
\item CA-DM-DAQ-ICD-0052 Correction constants for science sensors sourced by Data Management
\item DM-TS-CON-ICD-0003 Wavefront image archive access
\item DM-TS-CON-ICD-0009 Calibration Data Products
\item DMS-REQ-0167 Incorporate Autonomics
\item DMS-REQ-0314 Compute Platform Heterogeneity
\item DMS-REQ-0318 Data Management Unscheduled Downtime
\item DMS-REQ-0346 Data Availability
\end{itemize}
OCS-Driven Batch \begin{itemize}
\item DMS-REQ-0008 Pipeline Availability
\item DMS-REQ-0101 Level 1 Calibration Report Definition
\item DMS-REQ-0131 Calibration Images Available Within Specified Time
\item DMS-REQ-0162 Pipeline Throughput
\item DMS-REQ-0265 Guider Calibration Data Acquisition
\item DMS-REQ-0289 Calibration Production Processing
\item DMS-REQ-0294 Processing of Datasets
\item DMS-REQ-0301 Control of Level-1 Production
\item DMS-REQ-0314 Compute Platform Heterogeneity
\item DMS-REQ-0315 DMS Communication with OCS
\item DMS-REQ-0318 Data Management Unscheduled Downtime
\item OCS-DM-COM-ICD-0003 Data Management CSC Command Response Model
\item OCS-DM-COM-ICD-0009 Command Set Implementation by Data Management
\item OCS-DM-COM-ICD-0012 Start Command
\item OCS-DM-COM-ICD-0013 configure Successful Completion Response
\item OCS-DM-COM-ICD-0014 enable Command
\item OCS-DM-COM-ICD-0015 disable Command
\item OCS-DM-COM-ICD-0035 OCS-Driven Batch CSC
\item OCS-DM-COM-ICD-0036 standby Command
\item OCS-DM-COM-ICD-0037 exit Command
\item OCS-DM-COM-ICD-0038 enterControl Command
\item OCS-DM-COM-ICD-0039 enterControl Successful Completion Response
\item OCS-DM-COM-ICD-0040 Command Completion Response
\end{itemize}
Telemetry Gateway \begin{itemize}
\item CA-DM-CON-ICD-0007 Provide Data Management Conditions data
\item CA-DM-CON-ICD-0008 Data Management Conditions data latency
\item DM-TS-CON-ICD-0002 Timing
\item DM-TS-CON-ICD-0004 Use OCS for data transport
\item DM-TS-CON-ICD-0006 Data
\item DM-TS-CON-ICD-0007 Timing
\item DM-TS-CON-ICD-0011 Data Format
\item DMS-REQ-0008 Pipeline Availability
\item DMS-REQ-0314 Compute Platform Heterogeneity
\item DMS-REQ-0315 DMS Communication with OCS
\item DMS-REQ-0318 Data Management Unscheduled Downtime
\item OCS-DM-COM-ICD-0017 Data Management Telemetry Interface Model
\item OCS-DM-COM-ICD-0018 Data Management Telemetry Time Stamp
\item OCS-DM-COM-ICD-0019 Data Management Events and Telemetry Required by the OCS
\item OCS-DM-COM-ICD-0020 Image and Visit Processing and Archiving Status
\item OCS-DM-COM-ICD-0021 Data Quality Metrics
\item OCS-DM-COM-ICD-0022 System Health Metrics
\item OCS-DM-COM-ICD-0043 Image Retrieval for Archiving Event
\item OCS-DM-COM-ICD-0044 Image Retrieval For Processing Event
\item OCS-DM-COM-ICD-0045 Image in OODS Event
\item OCS-DM-COM-ICD-0046 Image Forwarded Event
\item OCS-DM-COM-ICD-0047 Image Archived Event
\item OCS-DM-COM-ICD-0048 Alert Production Complete Event
\item OCS-DM-COM-ICD-0049 WCS Information
\item OCS-DM-COM-ICD-0050 PSF Information
\item OCS-DM-COM-ICD-0051 Photometric Zeropoint Information
\item OCS-DM-COM-ICD-0052 Number of Alerts Information
\item OCS-DM-COM-ICD-0053 Summit-Base Network Utilization
\item OCS-DM-COM-ICD-0054 Base-Archive Network Utilization
\item OCS-DM-COM-ICD-0055 Archiver Resource Availability
\item OCS-DM-COM-ICD-0056 Prompt Processing Resource Availability
\end{itemize}
Alert Distribution \begin{itemize}
\item DMS-REQ-0002 Transient Alert Distribution
\item DMS-REQ-0004 Nightly Data Accessible Within 24 hrs
\item DMS-REQ-0008 Pipeline Availability
\item DMS-REQ-0161 Optimization of Cost, Reliability and Availability in Order
\item DMS-REQ-0162 Pipeline Throughput
\item DMS-REQ-0167 Incorporate Autonomics
\item DMS-REQ-0314 Compute Platform Heterogeneity
\item DMS-REQ-0318 Data Management Unscheduled Downtime
\item DMS-REQ-0342 Alert Filtering Service
\item DMS-REQ-0343 Performance Requirements for LSST Alert Filtering Service
\item DMS-REQ-0348 Pre-defined alert filters
\end{itemize}
Prompt Quality Control \begin{itemize}
\item DMS-REQ-0096 Generate Data Quality Report Within Specified Time
\item DMS-REQ-0097 Level 1 Data Quality Report Definition
\item DMS-REQ-0098 Generate DMS Performance Report Within Specified Time
\item DMS-REQ-0099 Level 1 Performance Report Definition
\item DMS-REQ-0100 Generate Calibration Report Within Specified Time
\item DMS-REQ-0101 Level 1 Calibration Report Definition
\item DMS-REQ-0314 Compute Platform Heterogeneity
\item DMS-REQ-0318 Data Management Unscheduled Downtime
\end{itemize}
Batch Production \begin{itemize}
\item DMS-REQ-0008 Pipeline Availability
\item DMS-REQ-0156 Provide Pipeline Execution Services
\item DMS-REQ-0163 Re-processing Capacity
\item DMS-REQ-0167 Incorporate Autonomics
\item DMS-REQ-0294 Processing of Datasets
\item DMS-REQ-0302 Production Orchestration
\item DMS-REQ-0303 Production Monitoring
\item DMS-REQ-0304 Production Fault Tolerance
\item DMS-REQ-0307 Unique Processing Coverage
\item DMS-REQ-0308 Software Architecture to Enable Community Re-Use
\item DMS-REQ-0314 Compute Platform Heterogeneity
\item DMS-REQ-0318 Data Management Unscheduled Downtime
\item DMS-REQ-0388 Provide Re-Run Tools
\item DMS-REQ-0389 Re-Runs on Similar Systems
\item DMS-REQ-0390 Re-Runs on Other Systems
\end{itemize}
Bulk Distribution \begin{itemize}
\item DMS-REQ-0122 Access to catalogs for external Level 3 processing
\item DMS-REQ-0123 Access to input catalogs for DAC-based Level 3 processing
\item DMS-REQ-0126 Access to images for external Level 3 processing
\item DMS-REQ-0127 Access to input images for DAC-based Level 3 processing
\item DMS-REQ-0185 Archive Center
\item DMS-REQ-0186 Archive Center Disaster Recovery
\item DMS-REQ-0193 Data Access Centers
\item DMS-REQ-0300 Bulk Download Service
\item DMS-REQ-0314 Compute Platform Heterogeneity
\item DMS-REQ-0318 Data Management Unscheduled Downtime
\item DMS-REQ-0346 Data Availability
\item EP-DM-CON-ICD-0004 DM Transfer of Catalog Data to EPO
\item EP-DM-CON-ICD-0009 Catalog Format
\item EP-DM-CON-ICD-0019 DM to EPO Data Transfer Cadence
\item EP-DM-CON-ICD-0021 DM Generation of a Color Hierarchical Progressive Survey for EPO
\end{itemize}
DBB Ingest/Metadata Management \begin{itemize}
\item DMS-REQ-0008 Pipeline Availability
\item DMS-REQ-0068 Raw Science Image Metadata
\item DMS-REQ-0074 Difference Exposure Attributes
\item DMS-REQ-0077 Maintain Archive Publicly Accessible
\item DMS-REQ-0089 Solar System Objects Available Within Specified Time
\item DMS-REQ-0094 Keep Historical Alert Archive
\item DMS-REQ-0102 Provide Engineering and Facility Database Archive
\item DMS-REQ-0120 Level 3 Data Product Self Consistency
\item DMS-REQ-0122 Access to catalogs for external Level 3 processing
\item DMS-REQ-0123 Access to input catalogs for DAC-based Level 3 processing
\item DMS-REQ-0126 Access to images for external Level 3 processing
\item DMS-REQ-0127 Access to input images for DAC-based Level 3 processing
\item DMS-REQ-0130 Calibration Data Products
\item DMS-REQ-0131 Calibration Images Available Within Specified Time
\item DMS-REQ-0132 Calibration Image Provenance
\item DMS-REQ-0161 Optimization of Cost, Reliability and Availability in Order
\item DMS-REQ-0162 Pipeline Throughput
\item DMS-REQ-0163 Re-processing Capacity
\item DMS-REQ-0164 Temporary Storage for Communications Links
\item DMS-REQ-0165 Infrastructure Sizing for "catching up"
\item DMS-REQ-0166 Incorporate Fault-Tolerance
\item DMS-REQ-0167 Incorporate Autonomics
\item DMS-REQ-0176 Base Facility Infrastructure
\item DMS-REQ-0185 Archive Center
\item DMS-REQ-0186 Archive Center Disaster Recovery
\item DMS-REQ-0197 No Limit on Data Access Centers
\item DMS-REQ-0266 Exposure Catalog
\item DMS-REQ-0269 DIASource Catalog
\item DMS-REQ-0271 DIAObject Catalog
\item DMS-REQ-0273 SSObject Catalog
\item DMS-REQ-0287 DIASource Precovery
\item DMS-REQ-0291 Query Repeatability
\item DMS-REQ-0292 Uniqueness of IDs Across Data Releases
\item DMS-REQ-0293 Selection of Datasets
\item DMS-REQ-0299 Data Product Ingest
\item DMS-REQ-0309 Raw Data Archiving Reliability
\item DMS-REQ-0310 Un-Archived Data Product Cache
\item DMS-REQ-0313 Level 1 and 2 Catalog Access
\item DMS-REQ-0314 Compute Platform Heterogeneity
\item DMS-REQ-0317 DIAForcedSource Catalog
\item DMS-REQ-0318 Data Management Unscheduled Downtime
\item DMS-REQ-0322 Special Programs Database
\item DMS-REQ-0334 Persisting Data Products
\item DMS-REQ-0338 Targeted Coadds
\item DMS-REQ-0339 Tracking Characterization Changes Between Data Releases
\item DMS-REQ-0346 Data Availability
\item DMS-REQ-0363 Access to Previous Data Releases
\item DMS-REQ-0364 Data Access Services
\item DMS-REQ-0365 Operations Subsets
\item DMS-REQ-0366 Subsets Support
\item DMS-REQ-0369 Evolution
\item DMS-REQ-0370 Older Release Behavior
\item DMS-REQ-0372a Archiving Camera Test Data
\item DMS-REQ-0386a Archive Processing Provenance
\item DMS-REQ-0387b Serve Archived Provenance
\item DMS-REQ-0388 Provide Re-Run Tools
\item OCS-DM-COM-ICD-0047 Image Archived Event
\end{itemize}
DBB Lifetime Management \begin{itemize}
\item DMS-REQ-0310 Un-Archived Data Product Cache
\item DMS-REQ-0334 Persisting Data Products
\item DMS-REQ-0338 Targeted Coadds
\item DMS-REQ-0339 Tracking Characterization Changes Between Data Releases
\item DMS-REQ-0346 Data Availability
\end{itemize}
DBB Transport/Replication/Backup \begin{itemize}
\item DMS-REQ-0008 Pipeline Availability
\item DMS-REQ-0089 Solar System Objects Available Within Specified Time
\item DMS-REQ-0102 Provide Engineering and Facility Database Archive
\item DMS-REQ-0122 Access to catalogs for external Level 3 processing
\item DMS-REQ-0123 Access to input catalogs for DAC-based Level 3 processing
\item DMS-REQ-0126 Access to images for external Level 3 processing
\item DMS-REQ-0127 Access to input images for DAC-based Level 3 processing
\item DMS-REQ-0131 Calibration Images Available Within Specified Time
\item DMS-REQ-0161 Optimization of Cost, Reliability and Availability in Order
\item DMS-REQ-0162 Pipeline Throughput
\item DMS-REQ-0163 Re-processing Capacity
\item DMS-REQ-0164 Temporary Storage for Communications Links
\item DMS-REQ-0165 Infrastructure Sizing for "catching up"
\item DMS-REQ-0166 Incorporate Fault-Tolerance
\item DMS-REQ-0167 Incorporate Autonomics
\item DMS-REQ-0185 Archive Center
\item DMS-REQ-0186 Archive Center Disaster Recovery
\item DMS-REQ-0197 No Limit on Data Access Centers
\item DMS-REQ-0287 DIASource Precovery
\item DMS-REQ-0309 Raw Data Archiving Reliability
\item DMS-REQ-0313 Level 1 and 2 Catalog Access
\item DMS-REQ-0314 Compute Platform Heterogeneity
\item DMS-REQ-0318 Data Management Unscheduled Downtime
\item DMS-REQ-0344 Constraints on Level 1 Special Program Products Generation
\item DMS-REQ-0366 Subsets Support
\item DMS-REQ-0370 Older Release Behavior
\end{itemize}
DBB Storage \begin{itemize}
\item DMS-REQ-0008 Pipeline Availability
\item DMS-REQ-0077 Maintain Archive Publicly Accessible
\item DMS-REQ-0089 Solar System Objects Available Within Specified Time
\item DMS-REQ-0094 Keep Historical Alert Archive
\item DMS-REQ-0102 Provide Engineering and Facility Database Archive
\item DMS-REQ-0122 Access to catalogs for external Level 3 processing
\item DMS-REQ-0123 Access to input catalogs for DAC-based Level 3 processing
\item DMS-REQ-0126 Access to images for external Level 3 processing
\item DMS-REQ-0127 Access to input images for DAC-based Level 3 processing
\item DMS-REQ-0130 Calibration Data Products
\item DMS-REQ-0131 Calibration Images Available Within Specified Time
\item DMS-REQ-0161 Optimization of Cost, Reliability and Availability in Order
\item DMS-REQ-0162 Pipeline Throughput
\item DMS-REQ-0163 Re-processing Capacity
\item DMS-REQ-0164 Temporary Storage for Communications Links
\item DMS-REQ-0165 Infrastructure Sizing for "catching up"
\item DMS-REQ-0166 Incorporate Fault-Tolerance
\item DMS-REQ-0167 Incorporate Autonomics
\item DMS-REQ-0185 Archive Center
\item DMS-REQ-0186 Archive Center Disaster Recovery
\item DMS-REQ-0197 No Limit on Data Access Centers
\item DMS-REQ-0287 DIASource Precovery
\item DMS-REQ-0309 Raw Data Archiving Reliability
\item DMS-REQ-0310 Un-Archived Data Product Cache
\item DMS-REQ-0313 Level 1 and 2 Catalog Access
\item DMS-REQ-0314 Compute Platform Heterogeneity
\item DMS-REQ-0318 Data Management Unscheduled Downtime
\item DMS-REQ-0334 Persisting Data Products
\item DMS-REQ-0338 Targeted Coadds
\item DMS-REQ-0344 Constraints on Level 1 Special Program Products Generation
\item DMS-REQ-0346 Data Availability
\item DMS-REQ-0363 Access to Previous Data Releases
\item DMS-REQ-0364 Data Access Services
\item DMS-REQ-0365 Operations Subsets
\item DMS-REQ-0366 Subsets Support
\item DMS-REQ-0367 Access Services Performance
\item DMS-REQ-0368 Implementation Provisions
\item DMS-REQ-0370 Older Release Behavior
\item DMS-REQ-0372a Archiving Camera Test Data
\item DMS-REQ-0386a Archive Processing Provenance
\item DMS-REQ-0387b Serve Archived Provenance
\end{itemize}
LSP Portal \begin{itemize}
\item DMS-REQ-0119 DAC resource allocation for Level 3 processing
\item DMS-REQ-0123 Access to input catalogs for DAC-based Level 3 processing
\item DMS-REQ-0124 Federation with external catalogs
\item DMS-REQ-0127 Access to input images for DAC-based Level 3 processing
\item DMS-REQ-0160 Provide User Interface Services
\item DMS-REQ-0161 Optimization of Cost, Reliability and Availability in Order
\item DMS-REQ-0193 Data Access Centers
\item DMS-REQ-0194 Data Access Center Simultaneous Connections
\item DMS-REQ-0197 No Limit on Data Access Centers
\item DMS-REQ-0314 Compute Platform Heterogeneity
\item DMS-REQ-0318 Data Management Unscheduled Downtime
\item DMS-REQ-0341 Providing a Precovery Service
\item DMS-REQ-0363 Access to Previous Data Releases
\item DMS-REQ-0364 Data Access Services
\item DMS-REQ-0365 Operations Subsets
\item DMS-REQ-0366 Subsets Support
\item DMS-REQ-0367 Access Services Performance
\item DMS-REQ-0368 Implementation Provisions
\item DMS-REQ-0369 Evolution
\item DMS-REQ-0370 Older Release Behavior
\item DMS-REQ-0371 Query Availability
\item DMS-REQ-0382 HiPS Visualization
\item DMS-REQ-0385 MOC Visualization
\end{itemize}
LSP JupyterLab \begin{itemize}
\item DMS-REQ-0119 DAC resource allocation for Level 3 processing
\item DMS-REQ-0123 Access to input catalogs for DAC-based Level 3 processing
\item DMS-REQ-0124 Federation with external catalogs
\item DMS-REQ-0127 Access to input images for DAC-based Level 3 processing
\item DMS-REQ-0161 Optimization of Cost, Reliability and Availability in Order
\item DMS-REQ-0193 Data Access Centers
\item DMS-REQ-0194 Data Access Center Simultaneous Connections
\item DMS-REQ-0197 No Limit on Data Access Centers
\item DMS-REQ-0314 Compute Platform Heterogeneity
\item DMS-REQ-0318 Data Management Unscheduled Downtime
\end{itemize}
LSP Web API \begin{itemize}
\item DMS-REQ-0065 Provide Image Access Services
\item DMS-REQ-0075 Catalog Queries
\item DMS-REQ-0078 Catalog Export Formats
\item DMS-REQ-0089 Solar System Objects Available Within Specified Time
\item DMS-REQ-0119 DAC resource allocation for Level 3 processing
\item DMS-REQ-0120 Level 3 Data Product Self Consistency
\item DMS-REQ-0121 Provenance for Level 3 processing at DACs
\item DMS-REQ-0123 Access to input catalogs for DAC-based Level 3 processing
\item DMS-REQ-0124 Federation with external catalogs
\item DMS-REQ-0127 Access to input images for DAC-based Level 3 processing
\item DMS-REQ-0155 Provide Data Access Services
\item DMS-REQ-0161 Optimization of Cost, Reliability and Availability in Order
\item DMS-REQ-0193 Data Access Centers
\item DMS-REQ-0194 Data Access Center Simultaneous Connections
\item DMS-REQ-0197 No Limit on Data Access Centers
\item DMS-REQ-0287 DIASource Precovery
\item DMS-REQ-0290 Level 3 Data Import
\item DMS-REQ-0291 Query Repeatability
\item DMS-REQ-0292 Uniqueness of IDs Across Data Releases
\item DMS-REQ-0293 Selection of Datasets
\item DMS-REQ-0295 Transparent Data Access
\item DMS-REQ-0298 Data Product and Raw Data Access
\item DMS-REQ-0311 Regenerate Un-archived Data Products
\item DMS-REQ-0312 Level 1 Data Product Access
\item DMS-REQ-0313 Level 1 and 2 Catalog Access
\item DMS-REQ-0314 Compute Platform Heterogeneity
\item DMS-REQ-0318 Data Management Unscheduled Downtime
\item DMS-REQ-0322 Special Programs Database
\item DMS-REQ-0323 Calculating SSObject Parameters
\item DMS-REQ-0324 Matching DIASources to Objects
\item DMS-REQ-0331 Computing Derived Quantities
\item DMS-REQ-0332 Denormalizing Database Tables
\item DMS-REQ-0334 Persisting Data Products
\item DMS-REQ-0336b Regenerating Data Products from Previous Data Releases
\item DMS-REQ-0338 Targeted Coadds
\item DMS-REQ-0339 Tracking Characterization Changes Between Data Releases
\item DMS-REQ-0344 Constraints on Level 1 Special Program Products Generation
\item DMS-REQ-0345 Logging of catalog queries
\item DMS-REQ-0346 Data Availability
\item DMS-REQ-0347 Measurements in catalogs
\item DMS-REQ-0363 Access to Previous Data Releases
\item DMS-REQ-0364 Data Access Services
\item DMS-REQ-0365 Operations Subsets
\item DMS-REQ-0366 Subsets Support
\item DMS-REQ-0367 Access Services Performance
\item DMS-REQ-0368 Implementation Provisions
\item DMS-REQ-0369 Evolution
\item DMS-REQ-0370 Older Release Behavior
\item DMS-REQ-0371 Query Availability
\item DMS-REQ-0380 HiPS Service
\item DMS-REQ-0381 HiPS Linkage to Coadds
\item DMS-REQ-0384 Export MOCs As FITS
\item DMS-REQ-0387b Serve Archived Provenance
\item EP-DM-CON-ICD-0034 Citizen Science Data
\item EP-DM-CON-ICD-0036 DM Services
\end{itemize}
Alert Distribution SW \begin{itemize}
\item DMS-REQ-0002 Transient Alert Distribution
\item DMS-REQ-0004 Nightly Data Accessible Within 24 hrs
\item DMS-REQ-0008 Pipeline Availability
\item DMS-REQ-0161 Optimization of Cost, Reliability and Availability in Order
\item DMS-REQ-0162 Pipeline Throughput
\item DMS-REQ-0167 Incorporate Autonomics
\item DMS-REQ-0314 Compute Platform Heterogeneity
\item DMS-REQ-0318 Data Management Unscheduled Downtime
\item DMS-REQ-0342 Alert Filtering Service
\item DMS-REQ-0343 Performance Requirements for LSST Alert Filtering Service
\item DMS-REQ-0348 Pre-defined alert filters
\item OCS-DM-COM-ICD-0048 Alert Production Complete Event
\end{itemize}
EFD Transformation \begin{itemize}
\item DMS-REQ-0004 Nightly Data Accessible Within 24 hrs
\item DMS-REQ-0008 Pipeline Availability
\item DMS-REQ-0102 Provide Engineering and Facility Database Archive
\item DMS-REQ-0165 Infrastructure Sizing for "catching up"
\item DMS-REQ-0167 Incorporate Autonomics
\item DMS-REQ-0309 Raw Data Archiving Reliability
\item DMS-REQ-0314 Compute Platform Heterogeneity
\item DMS-REQ-0315 DMS Communication with OCS
\item DMS-REQ-0318 Data Management Unscheduled Downtime
\item DMS-REQ-0346 Data Availability
\item OCS-DM-COM-ICD-0003 Data Management CSC Command Response Model
\item OCS-DM-COM-ICD-0004 Data Management Exposed CSCs
\item OCS-DM-COM-ICD-0008 EFD Transformation Service CSC
\item OCS-DM-COM-ICD-0025 Expected Load of Queries from DM
\item OCS-DM-COM-ICD-0026 Engineering and Facilities Database Archiving by Data Management
\item OCS-DM-COM-ICD-0027 Multiple Physically Separated Copies
\item OCS-DM-COM-ICD-0028 Expected Data Volume
\item OCS-DM-COM-ICD-0029 Archive Latency
\item OCS-DM-COM-ICD-0030 EFD Transformation Service Interface
\end{itemize}
Header Service SW \begin{itemize}
\item DMS-REQ-0266 Exposure Catalog
\item OCS-DM-COM-ICD-0003 Data Management CSC Command Response Model
\item OCS-DM-COM-ICD-0009 Command Set Implementation by Data Management
\item OCS-DM-COM-ICD-0012 Start Command
\item OCS-DM-COM-ICD-0013 configure Successful Completion Response
\item OCS-DM-COM-ICD-0014 enable Command
\item OCS-DM-COM-ICD-0015 disable Command
\item OCS-DM-COM-ICD-0033 Header Service CSC
\item OCS-DM-COM-ICD-0034 Auxiliary Header Service CSC
\item OCS-DM-COM-ICD-0036 standby Command
\item OCS-DM-COM-ICD-0037 exit Command
\item OCS-DM-COM-ICD-0038 enterControl Command
\item OCS-DM-COM-ICD-0039 enterControl Successful Completion Response
\item OCS-DM-COM-ICD-0040 Command Completion Response
\item OCS-EFD-HS-0001 Fulfill requirements of a Commandable SAL Component (CSC)
\item OCS-EFD-HS-0002 Critical System
\item OCS-EFD-HS-0003 Write Headers for all images taken by all Cameras supported by LSST
\item OCS-EFD-HS-0004 Ability to capture metadata at the beginning of exposure
\item OCS-EFD-HS-0005 Ability to capture metadata during of exposure integration
\item OCS-EFD-HS-0006 Ability to capture metadata at end of readout
\item OCS-EFD-HS-0007 Write header and Publish Event after end of telemetry event
\item OCS-EFD-HS-0008 Write header and Publish Event within specified time of the end-of-telemetry Event
\item OCS-EFD-HS-0009 Adherence to the FITS Standard
\item OCS-EFD-HS-0010 Configuration of Header Keywords and source
\item OCS-EFD-HS-0011 Produce header even if some meta-data not avaiable
\item OCS-EFD-HS-0012 Publish an Event if monitoring detects any failure of the service.
\item OCS-EFD-HS-0013 Extract metadata from published configuration
\item OCS-EFD-HS-0014 Metadata Capture
\item OCS-EFD-HS-0015 Generate on-the-fly additional metadata as approved by the Project CCB.
\end{itemize}
Image Ingest and Processing \begin{itemize}
\item CA-DM-CON-ICD-0014 Provide science sensor data
\item CA-DM-CON-ICD-0015 Provide wavefront sensor data
\item CA-DM-CON-ICD-0016 Provide guide sensor data
\item CA-DM-CON-ICD-0017 Data Management load on image data interfaces
\item CA-DM-CON-ICD-0019 Camera engineering image data archiving
\item DM-TS-CON-ICD-0002 Timing
\item DM-TS-CON-ICD-0007 Timing
\item DMS-REQ-0004 Nightly Data Accessible Within 24 hrs
\item DMS-REQ-0008 Pipeline Availability
\item DMS-REQ-0018 Raw Science Image Data Acquisition
\item DMS-REQ-0020 Wavefront Sensor Data Acquisition
\item DMS-REQ-0022 Crosstalk Corrected Science Image Data Acquisition
\item DMS-REQ-0024 Raw Image Assembly
\item DMS-REQ-0068 Raw Science Image Metadata
\item DMS-REQ-0099 Level 1 Performance Report Definition
\item DMS-REQ-0131 Calibration Images Available Within Specified Time
\item DMS-REQ-0165 Infrastructure Sizing for "catching up"
\item DMS-REQ-0167 Incorporate Autonomics
\item DMS-REQ-0265 Guider Calibration Data Acquisition
\item DMS-REQ-0284 Level-1 Production Completeness
\item DMS-REQ-0294 Processing of Datasets
\item DMS-REQ-0301 Control of Level-1 Production
\item DMS-REQ-0309 Raw Data Archiving Reliability
\item DMS-REQ-0314 Compute Platform Heterogeneity
\item DMS-REQ-0315 DMS Communication with OCS
\item DMS-REQ-0318 Data Management Unscheduled Downtime
\item DMS-REQ-0321 Level 1 Processing of Special Programs Data
\item DMS-REQ-0344 Constraints on Level 1 Special Program Products Generation
\item DMS-REQ-0346 Data Availability
\item OCS-DM-COM-ICD-0003 Data Management CSC Command Response Model
\item OCS-DM-COM-ICD-0004 Data Management Exposed CSCs
\item OCS-DM-COM-ICD-0005 Main Camera Archiver
\item OCS-DM-COM-ICD-0006 Catch-up Archiver
\item OCS-DM-COM-ICD-0007 Prompt Processing CSC
\item OCS-DM-COM-ICD-0009 Command Set Implementation by Data Management
\item OCS-DM-COM-ICD-0012 Start Command
\item OCS-DM-COM-ICD-0013 configure Successful Completion Response
\item OCS-DM-COM-ICD-0014 enable Command
\item OCS-DM-COM-ICD-0015 disable Command
\item OCS-DM-COM-ICD-0032 Auxiliary Telescope Archiver CSC
\item OCS-DM-COM-ICD-0036 standby Command
\item OCS-DM-COM-ICD-0037 exit Command
\item OCS-DM-COM-ICD-0038 enterControl Command
\item OCS-DM-COM-ICD-0039 enterControl Successful Completion Response
\item OCS-DM-COM-ICD-0040 Command Completion Response
\item OCS-DM-COM-ICD-0043 Image Retrieval for Archiving Event
\item OCS-DM-COM-ICD-0044 Image Retrieval For Processing Event
\item OCS-DM-COM-ICD-0045 Image in OODS Event
\item OCS-DM-COM-ICD-0055 Archiver Resource Availability
\item OCS-DM-COM-ICD-0056 Prompt Processing Resource Availability
\end{itemize}
Observatory Operations Data Service SW \begin{itemize}
\item CA-DM-DAQ-ICD-0052 Correction constants for science sensors sourced by Data Management
\item DM-TS-CON-ICD-0003 Wavefront image archive access
\item DMS-REQ-0167 Incorporate Autonomics
\item DMS-REQ-0314 Compute Platform Heterogeneity
\item DMS-REQ-0318 Data Management Unscheduled Downtime
\item DMS-REQ-0346 Data Availability
\end{itemize}
OCS Batch SW \begin{itemize}
\item DMS-REQ-0008 Pipeline Availability
\item DMS-REQ-0101 Level 1 Calibration Report Definition
\item DMS-REQ-0131 Calibration Images Available Within Specified Time
\item DMS-REQ-0162 Pipeline Throughput
\item DMS-REQ-0265 Guider Calibration Data Acquisition
\item DMS-REQ-0289 Calibration Production Processing
\item DMS-REQ-0294 Processing of Datasets
\item DMS-REQ-0301 Control of Level-1 Production
\item DMS-REQ-0314 Compute Platform Heterogeneity
\item DMS-REQ-0315 DMS Communication with OCS
\item DMS-REQ-0318 Data Management Unscheduled Downtime
\item OCS-DM-COM-ICD-0003 Data Management CSC Command Response Model
\item OCS-DM-COM-ICD-0009 Command Set Implementation by Data Management
\item OCS-DM-COM-ICD-0012 Start Command
\item OCS-DM-COM-ICD-0013 configure Successful Completion Response
\item OCS-DM-COM-ICD-0014 enable Command
\item OCS-DM-COM-ICD-0015 disable Command
\item OCS-DM-COM-ICD-0035 OCS-Driven Batch CSC
\item OCS-DM-COM-ICD-0036 standby Command
\item OCS-DM-COM-ICD-0037 exit Command
\item OCS-DM-COM-ICD-0038 enterControl Command
\item OCS-DM-COM-ICD-0039 enterControl Successful Completion Response
\item OCS-DM-COM-ICD-0040 Command Completion Response
\end{itemize}
Planned Observation Publication SW \begin{itemize}
\item DMS-REQ-0353 Publishing predicted visit schedule
\end{itemize}
Campaign Management \begin{itemize}
\item DMS-REQ-0008 Pipeline Availability
\item DMS-REQ-0156 Provide Pipeline Execution Services
\item DMS-REQ-0163 Re-processing Capacity
\item DMS-REQ-0167 Incorporate Autonomics
\item DMS-REQ-0294 Processing of Datasets
\item DMS-REQ-0302 Production Orchestration
\item DMS-REQ-0303 Production Monitoring
\item DMS-REQ-0304 Production Fault Tolerance
\item DMS-REQ-0314 Compute Platform Heterogeneity
\item DMS-REQ-0318 Data Management Unscheduled Downtime
\end{itemize}
Workload/Workflow Management \begin{itemize}
\item DMS-REQ-0008 Pipeline Availability
\item DMS-REQ-0156 Provide Pipeline Execution Services
\item DMS-REQ-0163 Re-processing Capacity
\item DMS-REQ-0167 Incorporate Autonomics
\item DMS-REQ-0294 Processing of Datasets
\item DMS-REQ-0302 Production Orchestration
\item DMS-REQ-0303 Production Monitoring
\item DMS-REQ-0304 Production Fault Tolerance
\item DMS-REQ-0308 Software Architecture to Enable Community Re-Use
\item DMS-REQ-0314 Compute Platform Heterogeneity
\item DMS-REQ-0318 Data Management Unscheduled Downtime
\item DMS-REQ-0386a Archive Processing Provenance
\item DMS-REQ-0388 Provide Re-Run Tools
\item DMS-REQ-0389 Re-Runs on Similar Systems
\item DMS-REQ-0390 Re-Runs on Other Systems
\end{itemize}
Quality Control SW \begin{itemize}
\item DMS-REQ-0096 Generate Data Quality Report Within Specified Time
\item DMS-REQ-0097 Level 1 Data Quality Report Definition
\item DMS-REQ-0098 Generate DMS Performance Report Within Specified Time
\item DMS-REQ-0099 Level 1 Performance Report Definition
\item DMS-REQ-0100 Generate Calibration Report Within Specified Time
\item DMS-REQ-0101 Level 1 Calibration Report Definition
\item DMS-REQ-0314 Compute Platform Heterogeneity
\item DMS-REQ-0318 Data Management Unscheduled Downtime
\end{itemize}
DBB Ingest/Metadata Management SW \begin{itemize}
\item DMS-REQ-0008 Pipeline Availability
\item DMS-REQ-0068 Raw Science Image Metadata
\item DMS-REQ-0074 Difference Exposure Attributes
\item DMS-REQ-0077 Maintain Archive Publicly Accessible
\item DMS-REQ-0089 Solar System Objects Available Within Specified Time
\item DMS-REQ-0094 Keep Historical Alert Archive
\item DMS-REQ-0102 Provide Engineering and Facility Database Archive
\item DMS-REQ-0120 Level 3 Data Product Self Consistency
\item DMS-REQ-0122 Access to catalogs for external Level 3 processing
\item DMS-REQ-0123 Access to input catalogs for DAC-based Level 3 processing
\item DMS-REQ-0126 Access to images for external Level 3 processing
\item DMS-REQ-0127 Access to input images for DAC-based Level 3 processing
\item DMS-REQ-0130 Calibration Data Products
\item DMS-REQ-0131 Calibration Images Available Within Specified Time
\item DMS-REQ-0132 Calibration Image Provenance
\item DMS-REQ-0161 Optimization of Cost, Reliability and Availability in Order
\item DMS-REQ-0162 Pipeline Throughput
\item DMS-REQ-0163 Re-processing Capacity
\item DMS-REQ-0164 Temporary Storage for Communications Links
\item DMS-REQ-0165 Infrastructure Sizing for "catching up"
\item DMS-REQ-0166 Incorporate Fault-Tolerance
\item DMS-REQ-0167 Incorporate Autonomics
\item DMS-REQ-0176 Base Facility Infrastructure
\item DMS-REQ-0185 Archive Center
\item DMS-REQ-0186 Archive Center Disaster Recovery
\item DMS-REQ-0197 No Limit on Data Access Centers
\item DMS-REQ-0266 Exposure Catalog
\item DMS-REQ-0269 DIASource Catalog
\item DMS-REQ-0271 DIAObject Catalog
\item DMS-REQ-0273 SSObject Catalog
\item DMS-REQ-0287 DIASource Precovery
\item DMS-REQ-0291 Query Repeatability
\item DMS-REQ-0292 Uniqueness of IDs Across Data Releases
\item DMS-REQ-0293 Selection of Datasets
\item DMS-REQ-0299 Data Product Ingest
\item DMS-REQ-0309 Raw Data Archiving Reliability
\item DMS-REQ-0310 Un-Archived Data Product Cache
\item DMS-REQ-0313 Level 1 and 2 Catalog Access
\item DMS-REQ-0314 Compute Platform Heterogeneity
\item DMS-REQ-0317 DIAForcedSource Catalog
\item DMS-REQ-0318 Data Management Unscheduled Downtime
\item DMS-REQ-0322 Special Programs Database
\item DMS-REQ-0334 Persisting Data Products
\item DMS-REQ-0338 Targeted Coadds
\item DMS-REQ-0339 Tracking Characterization Changes Between Data Releases
\item DMS-REQ-0346 Data Availability
\item DMS-REQ-0363 Access to Previous Data Releases
\item DMS-REQ-0364 Data Access Services
\item DMS-REQ-0365 Operations Subsets
\item DMS-REQ-0366 Subsets Support
\item DMS-REQ-0369 Evolution
\item DMS-REQ-0370 Older Release Behavior
\item OCS-DM-COM-ICD-0047 Image Archived Event
\end{itemize}
DBB Lifetime Management SW \begin{itemize}
\item DMS-REQ-0310 Un-Archived Data Product Cache
\item DMS-REQ-0334 Persisting Data Products
\item DMS-REQ-0338 Targeted Coadds
\item DMS-REQ-0339 Tracking Characterization Changes Between Data Releases
\item DMS-REQ-0346 Data Availability
\end{itemize}
DBB Transport/Replication/Backup SW \begin{itemize}
\item DMS-REQ-0008 Pipeline Availability
\item DMS-REQ-0089 Solar System Objects Available Within Specified Time
\item DMS-REQ-0102 Provide Engineering and Facility Database Archive
\item DMS-REQ-0122 Access to catalogs for external Level 3 processing
\item DMS-REQ-0123 Access to input catalogs for DAC-based Level 3 processing
\item DMS-REQ-0126 Access to images for external Level 3 processing
\item DMS-REQ-0127 Access to input images for DAC-based Level 3 processing
\item DMS-REQ-0131 Calibration Images Available Within Specified Time
\item DMS-REQ-0161 Optimization of Cost, Reliability and Availability in Order
\item DMS-REQ-0162 Pipeline Throughput
\item DMS-REQ-0163 Re-processing Capacity
\item DMS-REQ-0164 Temporary Storage for Communications Links
\item DMS-REQ-0165 Infrastructure Sizing for "catching up"
\item DMS-REQ-0166 Incorporate Fault-Tolerance
\item DMS-REQ-0167 Incorporate Autonomics
\item DMS-REQ-0185 Archive Center
\item DMS-REQ-0186 Archive Center Disaster Recovery
\item DMS-REQ-0197 No Limit on Data Access Centers
\item DMS-REQ-0287 DIASource Precovery
\item DMS-REQ-0309 Raw Data Archiving Reliability
\item DMS-REQ-0313 Level 1 and 2 Catalog Access
\item DMS-REQ-0314 Compute Platform Heterogeneity
\item DMS-REQ-0318 Data Management Unscheduled Downtime
\item DMS-REQ-0344 Constraints on Level 1 Special Program Products Generation
\item DMS-REQ-0366 Subsets Support
\item DMS-REQ-0370 Older Release Behavior
\end{itemize}
SUIT \begin{itemize}
\item DMS-REQ-0119 DAC resource allocation for Level 3 processing
\item DMS-REQ-0123 Access to input catalogs for DAC-based Level 3 processing
\item DMS-REQ-0124 Federation with external catalogs
\item DMS-REQ-0127 Access to input images for DAC-based Level 3 processing
\item DMS-REQ-0160 Provide User Interface Services
\item DMS-REQ-0161 Optimization of Cost, Reliability and Availability in Order
\item DMS-REQ-0193 Data Access Centers
\item DMS-REQ-0194 Data Access Center Simultaneous Connections
\item DMS-REQ-0197 No Limit on Data Access Centers
\item DMS-REQ-0314 Compute Platform Heterogeneity
\item DMS-REQ-0318 Data Management Unscheduled Downtime
\item DMS-REQ-0341 Providing a Precovery Service
\item DMS-REQ-0363 Access to Previous Data Releases
\item DMS-REQ-0364 Data Access Services
\item DMS-REQ-0365 Operations Subsets
\item DMS-REQ-0366 Subsets Support
\item DMS-REQ-0367 Access Services Performance
\item DMS-REQ-0368 Implementation Provisions
\item DMS-REQ-0369 Evolution
\item DMS-REQ-0370 Older Release Behavior
\item DMS-REQ-0371 Query Availability
\item DMS-REQ-0382 HiPS Visualization
\end{itemize}
LSP JupyterLab SW \begin{itemize}
\item DMS-REQ-0119 DAC resource allocation for Level 3 processing
\item DMS-REQ-0123 Access to input catalogs for DAC-based Level 3 processing
\item DMS-REQ-0124 Federation with external catalogs
\item DMS-REQ-0127 Access to input images for DAC-based Level 3 processing
\item DMS-REQ-0161 Optimization of Cost, Reliability and Availability in Order
\item DMS-REQ-0193 Data Access Centers
\item DMS-REQ-0194 Data Access Center Simultaneous Connections
\item DMS-REQ-0197 No Limit on Data Access Centers
\item DMS-REQ-0314 Compute Platform Heterogeneity
\item DMS-REQ-0318 Data Management Unscheduled Downtime
\end{itemize}
LSP Web API SW \begin{itemize}
\item DMS-REQ-0065 Provide Image Access Services
\item DMS-REQ-0075 Catalog Queries
\item DMS-REQ-0078 Catalog Export Formats
\item DMS-REQ-0089 Solar System Objects Available Within Specified Time
\item DMS-REQ-0119 DAC resource allocation for Level 3 processing
\item DMS-REQ-0120 Level 3 Data Product Self Consistency
\item DMS-REQ-0121 Provenance for Level 3 processing at DACs
\item DMS-REQ-0123 Access to input catalogs for DAC-based Level 3 processing
\item DMS-REQ-0124 Federation with external catalogs
\item DMS-REQ-0127 Access to input images for DAC-based Level 3 processing
\item DMS-REQ-0155 Provide Data Access Services
\item DMS-REQ-0161 Optimization of Cost, Reliability and Availability in Order
\item DMS-REQ-0193 Data Access Centers
\item DMS-REQ-0194 Data Access Center Simultaneous Connections
\item DMS-REQ-0197 No Limit on Data Access Centers
\item DMS-REQ-0287 DIASource Precovery
\item DMS-REQ-0290 Level 3 Data Import
\item DMS-REQ-0291 Query Repeatability
\item DMS-REQ-0292 Uniqueness of IDs Across Data Releases
\item DMS-REQ-0293 Selection of Datasets
\item DMS-REQ-0295 Transparent Data Access
\item DMS-REQ-0298 Data Product and Raw Data Access
\item DMS-REQ-0299 Data Product Ingest
\item DMS-REQ-0311 Regenerate Un-archived Data Products
\item DMS-REQ-0312 Level 1 Data Product Access
\item DMS-REQ-0313 Level 1 and 2 Catalog Access
\item DMS-REQ-0314 Compute Platform Heterogeneity
\item DMS-REQ-0318 Data Management Unscheduled Downtime
\item DMS-REQ-0322 Special Programs Database
\item DMS-REQ-0323 Calculating SSObject Parameters
\item DMS-REQ-0324 Matching DIASources to Objects
\item DMS-REQ-0331 Computing Derived Quantities
\item DMS-REQ-0332 Denormalizing Database Tables
\item DMS-REQ-0334 Persisting Data Products
\item DMS-REQ-0336b Regenerating Data Products from Previous Data Releases
\item DMS-REQ-0338 Targeted Coadds
\item DMS-REQ-0339 Tracking Characterization Changes Between Data Releases
\item DMS-REQ-0344 Constraints on Level 1 Special Program Products Generation
\item DMS-REQ-0345 Logging of catalog queries
\item DMS-REQ-0346 Data Availability
\item DMS-REQ-0347 Measurements in catalogs
\item DMS-REQ-0363 Access to Previous Data Releases
\item DMS-REQ-0364 Data Access Services
\item DMS-REQ-0365 Operations Subsets
\item DMS-REQ-0366 Subsets Support
\item DMS-REQ-0367 Access Services Performance
\item DMS-REQ-0368 Implementation Provisions
\item DMS-REQ-0369 Evolution
\item DMS-REQ-0370 Older Release Behavior
\item DMS-REQ-0371 Query Availability
\item EP-DM-CON-ICD-0034 Citizen Science Data
\end{itemize}
Alert Production \begin{itemize}
\item DMS-REQ-0002 Transient Alert Distribution
\item DMS-REQ-0004 Nightly Data Accessible Within 24 hrs
\item DMS-REQ-0009 Simulated Data
\item DMS-REQ-0010 Difference Exposures
\item DMS-REQ-0029 Generate Photometric Zeropoint for Visit Image
\item DMS-REQ-0030 Generate WCS for Visit Images
\item DMS-REQ-0032 Image Differencing
\item DMS-REQ-0033 Provide Source Detection Software
\item DMS-REQ-0042 Provide Astrometric Model
\item DMS-REQ-0043 Provide Calibrated Photometry
\item DMS-REQ-0052 Enable a Range of Shape Measurement Approaches
\item DMS-REQ-0069 Processed Visit Images
\item DMS-REQ-0070 Generate PSF for Visit Images
\item DMS-REQ-0072 Processed Visit Image Content
\item DMS-REQ-0074 Difference Exposure Attributes
\item DMS-REQ-0097 Level 1 Data Quality Report Definition
\item DMS-REQ-0266 Exposure Catalog
\item DMS-REQ-0269 DIASource Catalog
\item DMS-REQ-0270 Faint DIASource Measurements
\item DMS-REQ-0271 DIAObject Catalog
\item DMS-REQ-0272 DIAObject Attributes
\item DMS-REQ-0274 Alert Content
\item DMS-REQ-0285 Level 1 Source Association
\item DMS-REQ-0288 Use of External Orbit Catalogs
\item DMS-REQ-0308 Software Architecture to Enable Community Re-Use
\item DMS-REQ-0312 Level 1 Data Product Access
\item DMS-REQ-0317 DIAForcedSource Catalog
\item DMS-REQ-0319 Characterizing Variability
\item DMS-REQ-0321 Level 1 Processing of Special Programs Data
\item DMS-REQ-0324 Matching DIASources to Objects
\item DMS-REQ-0327 Background Model Calculation
\item DMS-REQ-0328 Documenting Image Characterization
\item DMS-REQ-0333 Maximum Likelihood Values and Covariances
\item DMS-REQ-0344 Constraints on Level 1 Special Program Products Generation
\item DMS-REQ-0347 Measurements in catalogs
\item OCS-DM-COM-ICD-0049 WCS Information
\item OCS-DM-COM-ICD-0050 PSF Information
\item OCS-DM-COM-ICD-0051 Photometric Zeropoint Information
\item OCS-DM-COM-ICD-0052 Number of Alerts Information
\end{itemize}
Calibration SW \begin{itemize}
\item DMS-REQ-0101 Level 1 Calibration Report Definition
\item DMS-REQ-0131 Calibration Images Available Within Specified Time
\item DMS-REQ-0265 Guider Calibration Data Acquisition
\item DMS-REQ-0308 Software Architecture to Enable Community Re-Use
\end{itemize}
Data Release Production \begin{itemize}
\item DMS-REQ-0009 Simulated Data
\item DMS-REQ-0032 Image Differencing
\item DMS-REQ-0033 Provide Source Detection Software
\item DMS-REQ-0034 Associate Sources to Objects
\item DMS-REQ-0042 Provide Astrometric Model
\item DMS-REQ-0043 Provide Calibrated Photometry
\item DMS-REQ-0046 Provide Photometric Redshifts of Galaxies
\item DMS-REQ-0047 Provide PSF for Coadded Images
\item DMS-REQ-0052 Enable a Range of Shape Measurement Approaches
\item DMS-REQ-0103 Produce Images for EPO
\item DMS-REQ-0106 Coadded Image Provenance
\item DMS-REQ-0267 Source Catalog
\item DMS-REQ-0268 Forced-Source Catalog
\item DMS-REQ-0275 Object Catalog
\item DMS-REQ-0276 Object Characterization
\item DMS-REQ-0277 Coadd Source Catalog
\item DMS-REQ-0278 Coadd Image Method Constraints
\item DMS-REQ-0279 Deep Detection Coadds
\item DMS-REQ-0280 Template Coadds
\item DMS-REQ-0281 Multi-band Coadds
\item DMS-REQ-0308 Software Architecture to Enable Community Re-Use
\item DMS-REQ-0320 Processing of Data From Special Programs
\item DMS-REQ-0325 Regenerating L1 Data Products During Data Release Processing
\item DMS-REQ-0326 Storing Approximations of Per-pixel Metadata
\item DMS-REQ-0329 All-Sky Visualization of Data Releases
\item DMS-REQ-0330 Best Seeing Coadds
\item DMS-REQ-0331 Computing Derived Quantities
\item DMS-REQ-0333 Maximum Likelihood Values and Covariances
\item DMS-REQ-0335 PSF-Matched Coadds
\item DMS-REQ-0337 Detecting faint variable objects
\item DMS-REQ-0347 Measurements in catalogs
\item DMS-REQ-0349 Detecting extended  low surface brightness objects
\item DMS-REQ-0350 Associating Objects across data releases
\item DMS-REQ-0379 Produce All-Sky HiPS Map
\item DMS-REQ-0381 HiPS Linkage to Coadds
\item DMS-REQ-0383 Produce MOC Maps
\end{itemize}
MOPS and Forced Photometry \begin{itemize}
\item DMS-REQ-0004 Nightly Data Accessible Within 24 hrs
\item DMS-REQ-0089 Solar System Objects Available Within Specified Time
\item DMS-REQ-0273 SSObject Catalog
\item DMS-REQ-0286 SSObject Precovery
\item DMS-REQ-0287 DIASource Precovery
\item DMS-REQ-0288 Use of External Orbit Catalogs
\item DMS-REQ-0308 Software Architecture to Enable Community Re-Use
\item DMS-REQ-0317 DIAForcedSource Catalog
\item DMS-REQ-0319 Characterizing Variability
\item DMS-REQ-0321 Level 1 Processing of Special Programs Data
\item DMS-REQ-0341 Providing a Precovery Service
\item DMS-REQ-0344 Constraints on Level 1 Special Program Products Generation
\end{itemize}
Special Programs Productions \begin{itemize}
\item DMS-REQ-0320 Processing of Data From Special Programs
\item DMS-REQ-0322 Special Programs Database
\end{itemize}
Template Generation \begin{itemize}
\item DMS-REQ-0280 Template Coadds
\item DMS-REQ-0308 Software Architecture to Enable Community Re-Use
\end{itemize}
Science Plugins \begin{itemize}
\item DMS-REQ-0059 Bad Pixel Map
\item DMS-REQ-0060 Bias Residual Image
\item DMS-REQ-0061 Crosstalk Correction Matrix
\item DMS-REQ-0062 Illumination Correction Frame
\item DMS-REQ-0063 Monochromatic Flatfield Data Cube
\item DMS-REQ-0100 Generate Calibration Report Within Specified Time
\item DMS-REQ-0101 Level 1 Calibration Report Definition
\item DMS-REQ-0130 Calibration Data Products
\item DMS-REQ-0131 Calibration Images Available Within Specified Time
\item DMS-REQ-0132 Calibration Image Provenance
\item DMS-REQ-0265 Guider Calibration Data Acquisition
\item DMS-REQ-0282 Dark Current Correction Frame
\item DMS-REQ-0283 Fringe Correction Frame
\item DMS-REQ-0289 Calibration Production Processing
\item DMS-REQ-0308 Software Architecture to Enable Community Re-Use
\end{itemize}
Science Pipelines Distribution \begin{itemize}
\item CA-DM-DAQ-ICD-0052 Correction constants for science sensors sourced by Data Management
\item DMS-REQ-0059 Bad Pixel Map
\item DMS-REQ-0060 Bias Residual Image
\item DMS-REQ-0061 Crosstalk Correction Matrix
\item DMS-REQ-0062 Illumination Correction Frame
\item DMS-REQ-0063 Monochromatic Flatfield Data Cube
\item DMS-REQ-0130 Calibration Data Products
\item DMS-REQ-0132 Calibration Image Provenance
\item DMS-REQ-0265 Guider Calibration Data Acquisition
\item DMS-REQ-0282 Dark Current Correction Frame
\item DMS-REQ-0283 Fringe Correction Frame
\item DMS-REQ-0289 Calibration Production Processing
\item DMS-REQ-0308 Software Architecture to Enable Community Re-Use
\end{itemize}
Science Pipelines Libraries \begin{itemize}
\item DM-TS-CON-ICD-0008 LSST Stack Availability
\item DMS-REQ-0032 Image Differencing
\item DMS-REQ-0033 Provide Source Detection Software
\item DMS-REQ-0042 Provide Astrometric Model
\item DMS-REQ-0043 Provide Calibrated Photometry
\item DMS-REQ-0052 Enable a Range of Shape Measurement Approaches
\item DMS-REQ-0296 Pre-cursor, and Real Data
\item DMS-REQ-0308 Software Architecture to Enable Community Re-Use
\item DMS-REQ-0351 Provide Beam Projector Coordinate Calculation Software
\end{itemize}
Data Butler \begin{itemize}
\item DMS-REQ-0121 Provenance for Level 3 processing at DACs
\item DMS-REQ-0125 Software framework for Level 3 catalog processing
\item DMS-REQ-0128 Software framework for Level 3 image processing
\item DMS-REQ-0293 Selection of Datasets
\item DMS-REQ-0295 Transparent Data Access
\item DMS-REQ-0296 Pre-cursor, and Real Data
\item DMS-REQ-0308 Software Architecture to Enable Community Re-Use
\item DMS-REQ-0314 Compute Platform Heterogeneity
\item EP-DM-CON-ICD-0035 DM Software
\end{itemize}
Task Framework \begin{itemize}
\item DMS-REQ-0121 Provenance for Level 3 processing at DACs
\item DMS-REQ-0125 Software framework for Level 3 catalog processing
\item DMS-REQ-0128 Software framework for Level 3 image processing
\item DMS-REQ-0158 Provide Pipeline Construction Services
\item DMS-REQ-0294 Processing of Datasets
\item DMS-REQ-0304 Production Fault Tolerance
\item DMS-REQ-0305 Task Specification
\item DMS-REQ-0306 Task Configuration
\item DMS-REQ-0308 Software Architecture to Enable Community Re-Use
\item DMS-REQ-0314 Compute Platform Heterogeneity
\item DMS-REQ-0320 Processing of Data From Special Programs
\item EP-DM-CON-ICD-0035 DM Software
\end{itemize}
Distributed Database \begin{itemize}
\item DMS-REQ-0046 Provide Photometric Redshifts of Galaxies
\item DMS-REQ-0075 Catalog Queries
\item DMS-REQ-0077 Maintain Archive Publicly Accessible
\item DMS-REQ-0267 Source Catalog
\item DMS-REQ-0268 Forced-Source Catalog
\item DMS-REQ-0275 Object Catalog
\item DMS-REQ-0276 Object Characterization
\item DMS-REQ-0277 Coadd Source Catalog
\item DMS-REQ-0290 Level 3 Data Import
\item DMS-REQ-0291 Query Repeatability
\item DMS-REQ-0292 Uniqueness of IDs Across Data Releases
\item DMS-REQ-0293 Selection of Datasets
\item DMS-REQ-0313 Level 1 and 2 Catalog Access
\item DMS-REQ-0322 Special Programs Database
\item DMS-REQ-0331 Computing Derived Quantities
\item DMS-REQ-0332 Denormalizing Database Tables
\item DMS-REQ-0333 Maximum Likelihood Values and Covariances
\item DMS-REQ-0340 Access Controls of Level 3 Data Products
\item DMS-REQ-0345 Logging of catalog queries
\item DMS-REQ-0347 Measurements in catalogs
\item DMS-REQ-0350 Associating Objects across data releases
\item DMS-REQ-0363 Access to Previous Data Releases
\item DMS-REQ-0365 Operations Subsets
\item DMS-REQ-0366 Subsets Support
\item DMS-REQ-0367 Access Services Performance
\item DMS-REQ-0368 Implementation Provisions
\item DMS-REQ-0369 Evolution
\item DMS-REQ-0370 Older Release Behavior
\item DMS-REQ-0371 Query Availability
\end{itemize}
ADQL Translator \begin{itemize}
\item DMS-REQ-0075 Catalog Queries
\item DMS-REQ-0078 Catalog Export Formats
\item DMS-REQ-0123 Access to input catalogs for DAC-based Level 3 processing
\item DMS-REQ-0155 Provide Data Access Services
\item DMS-REQ-0291 Query Repeatability
\item DMS-REQ-0298 Data Product and Raw Data Access
\item DMS-REQ-0322 Special Programs Database
\item DMS-REQ-0323 Calculating SSObject Parameters
\item DMS-REQ-0331 Computing Derived Quantities
\item DMS-REQ-0340 Access Controls of Level 3 Data Products
\item DMS-REQ-0345 Logging of catalog queries
\item DMS-REQ-0364 Data Access Services
\item DMS-REQ-0368 Implementation Provisions
\item DMS-REQ-0369 Evolution
\item EP-DM-CON-ICD-0034 Citizen Science Data
\end{itemize}
Image/Cutout Server \begin{itemize}
\item DMS-REQ-0065 Provide Image Access Services
\item DMS-REQ-0127 Access to input images for DAC-based Level 3 processing
\item DMS-REQ-0155 Provide Data Access Services
\item DMS-REQ-0293 Selection of Datasets
\item DMS-REQ-0298 Data Product and Raw Data Access
\item DMS-REQ-0311 Regenerate Un-archived Data Products
\item DMS-REQ-0334 Persisting Data Products
\item DMS-REQ-0336b Regenerating Data Products from Previous Data Releases
\item DMS-REQ-0338 Targeted Coadds
\item DMS-REQ-0339 Tracking Characterization Changes Between Data Releases
\item DMS-REQ-0346 Data Availability
\item DMS-REQ-0368 Implementation Provisions
\item EP-DM-CON-ICD-0034 Citizen Science Data
\end{itemize}
Summit to Base Network \begin{itemize}
\item DMS-REQ-0165 Infrastructure Sizing for "catching up"
\item DMS-REQ-0166 Incorporate Fault-Tolerance
\item DMS-REQ-0167 Incorporate Autonomics
\item DMS-REQ-0168 Summit Facility Data Communications
\item DMS-REQ-0171 Summit to Base Network
\item DMS-REQ-0172 Summit to Base Network Availability
\item DMS-REQ-0173 Summit to Base Network Reliability
\item DMS-REQ-0174 Summit to Base Network Secondary Link
\item DMS-REQ-0175 Summit to Base Network Ownership and Operation
\item OCS-DM-COM-ICD-0053 Summit-Base Network Utilization
\end{itemize}
Base to Archive Network \begin{itemize}
\item DMS-REQ-0165 Infrastructure Sizing for "catching up"
\item DMS-REQ-0166 Incorporate Fault-Tolerance
\item DMS-REQ-0167 Incorporate Autonomics
\item DMS-REQ-0180 Base to Archive Network
\item DMS-REQ-0181 Base to Archive Network Availability
\item DMS-REQ-0182 Base to Archive Network Reliability
\item DMS-REQ-0183 Base to Archive Network Secondary Link
\item DMS-REQ-0188 Archive to Data Access Center Network
\item DMS-REQ-0189 Archive to Data Access Center Network Availability
\item DMS-REQ-0190 Archive to Data Access Center Network Reliability
\item DMS-REQ-0191 Archive to Data Access Center Network Secondary Link
\item OCS-DM-COM-ICD-0054 Base-Archive Network Utilization
\end{itemize}
Base LAN Network \begin{itemize}
\item DMS-REQ-0165 Infrastructure Sizing for "catching up"
\item DMS-REQ-0166 Incorporate Fault-Tolerance
\item DMS-REQ-0167 Incorporate Autonomics
\item DMS-REQ-0193 Data Access Centers
\item DMS-REQ-0194 Data Access Center Simultaneous Connections
\item DMS-REQ-0352 Base Wireless LAN (WiFi)
\end{itemize}
NCSA LAN Network \begin{itemize}
\item DMS-REQ-0163 Re-processing Capacity
\item DMS-REQ-0165 Infrastructure Sizing for "catching up"
\item DMS-REQ-0166 Incorporate Fault-Tolerance
\item DMS-REQ-0167 Incorporate Autonomics
\item DMS-REQ-0188 Archive to Data Access Center Network
\item DMS-REQ-0189 Archive to Data Access Center Network Availability
\item DMS-REQ-0190 Archive to Data Access Center Network Reliability
\item DMS-REQ-0191 Archive to Data Access Center Network Secondary Link
\item DMS-REQ-0193 Data Access Centers
\item DMS-REQ-0194 Data Access Center Simultaneous Connections
\end{itemize}
Network Management \begin{itemize}
\item DMS-REQ-0008 Pipeline Availability
\item DMS-REQ-0166 Incorporate Fault-Tolerance
\item DMS-REQ-0167 Incorporate Autonomics
\item DMS-REQ-0175 Summit to Base Network Ownership and Operation
\end{itemize}
Base Facility \begin{itemize}
\item DM-TS-CON-ICD-0003 Wavefront image archive access
\item DMS-REQ-0008 Pipeline Availability
\item DMS-REQ-0161 Optimization of Cost, Reliability and Availability in Order
\item DMS-REQ-0162 Pipeline Throughput
\item DMS-REQ-0163 Re-processing Capacity
\item DMS-REQ-0164 Temporary Storage for Communications Links
\item DMS-REQ-0165 Infrastructure Sizing for "catching up"
\item DMS-REQ-0166 Incorporate Fault-Tolerance
\item DMS-REQ-0167 Incorporate Autonomics
\item DMS-REQ-0170 Prefer Computing and Storage Down
\item DMS-REQ-0176 Base Facility Infrastructure
\item DMS-REQ-0178 Base Facility Co-Location with Existing Facility
\item DMS-REQ-0193 Data Access Centers
\item DMS-REQ-0196 Data Access Center Geographical Distribution
\item DMS-REQ-0297 DMS Initialization Component
\item DMS-REQ-0314 Compute Platform Heterogeneity
\item DMS-REQ-0315 DMS Communication with OCS
\item DMS-REQ-0316 Commissioning Cluster
\item DMS-REQ-0318 Data Management Unscheduled Downtime
\item DMS-REQ-0352 Base Wireless LAN (WiFi)
\item OCS-DM-COM-ICD-0027 Multiple Physically Separated Copies
\end{itemize}
NCSA Facility \begin{itemize}
\item DM-TS-CON-ICD-0003 Wavefront image archive access
\item DMS-REQ-0008 Pipeline Availability
\item DMS-REQ-0161 Optimization of Cost, Reliability and Availability in Order
\item DMS-REQ-0162 Pipeline Throughput
\item DMS-REQ-0163 Re-processing Capacity
\item DMS-REQ-0164 Temporary Storage for Communications Links
\item DMS-REQ-0165 Infrastructure Sizing for "catching up"
\item DMS-REQ-0166 Incorporate Fault-Tolerance
\item DMS-REQ-0167 Incorporate Autonomics
\item DMS-REQ-0170 Prefer Computing and Storage Down
\item DMS-REQ-0185 Archive Center
\item DMS-REQ-0186 Archive Center Disaster Recovery
\item DMS-REQ-0187 Archive Center Co-Location with Existing Facility
\item DMS-REQ-0193 Data Access Centers
\item DMS-REQ-0196 Data Access Center Geographical Distribution
\item DMS-REQ-0297 DMS Initialization Component
\item DMS-REQ-0314 Compute Platform Heterogeneity
\item DMS-REQ-0318 Data Management Unscheduled Downtime
\item OCS-DM-COM-ICD-0027 Multiple Physically Separated Copies
\end{itemize}
Archive Base Enclave \begin{itemize}
\item CA-DM-CON-ICD-0019 Camera engineering image data archiving
\item DM-TS-CON-ICD-0003 Wavefront image archive access
\item DMS-REQ-0004 Nightly Data Accessible Within 24 hrs
\item DMS-REQ-0010 Difference Exposures
\item DMS-REQ-0029 Generate Photometric Zeropoint for Visit Image
\item DMS-REQ-0030 Generate WCS for Visit Images
\item DMS-REQ-0069 Processed Visit Images
\item DMS-REQ-0077 Maintain Archive Publicly Accessible
\item DMS-REQ-0078 Catalog Export Formats
\item DMS-REQ-0089 Solar System Objects Available Within Specified Time
\item DMS-REQ-0094 Keep Historical Alert Archive
\item DMS-REQ-0102 Provide Engineering and Facility Database Archive
\item DMS-REQ-0106 Coadded Image Provenance
\item DMS-REQ-0131 Calibration Images Available Within Specified Time
\item DMS-REQ-0161 Optimization of Cost, Reliability and Availability in Order
\item DMS-REQ-0162 Pipeline Throughput
\item DMS-REQ-0163 Re-processing Capacity
\item DMS-REQ-0166 Incorporate Fault-Tolerance
\item DMS-REQ-0167 Incorporate Autonomics
\item DMS-REQ-0176 Base Facility Infrastructure
\item DMS-REQ-0186 Archive Center Disaster Recovery
\item DMS-REQ-0266 Exposure Catalog
\item DMS-REQ-0267 Source Catalog
\item DMS-REQ-0268 Forced-Source Catalog
\item DMS-REQ-0269 DIASource Catalog
\item DMS-REQ-0270 Faint DIASource Measurements
\item DMS-REQ-0271 DIAObject Catalog
\item DMS-REQ-0272 DIAObject Attributes
\item DMS-REQ-0273 SSObject Catalog
\item DMS-REQ-0274 Alert Content
\item DMS-REQ-0275 Object Catalog
\item DMS-REQ-0284 Level-1 Production Completeness
\item DMS-REQ-0287 DIASource Precovery
\item DMS-REQ-0291 Query Repeatability
\item DMS-REQ-0309 Raw Data Archiving Reliability
\item DMS-REQ-0310 Un-Archived Data Product Cache
\item DMS-REQ-0312 Level 1 Data Product Access
\item DMS-REQ-0314 Compute Platform Heterogeneity
\item DMS-REQ-0317 DIAForcedSource Catalog
\item DMS-REQ-0318 Data Management Unscheduled Downtime
\item DMS-REQ-0322 Special Programs Database
\item DMS-REQ-0324 Matching DIASources to Objects
\item DMS-REQ-0327 Background Model Calculation
\item DMS-REQ-0334 Persisting Data Products
\item DMS-REQ-0338 Targeted Coadds
\item DMS-REQ-0344 Constraints on Level 1 Special Program Products Generation
\item DMS-REQ-0363 Access to Previous Data Releases
\item DMS-REQ-0364 Data Access Services
\item DMS-REQ-0365 Operations Subsets
\item DMS-REQ-0366 Subsets Support
\item DMS-REQ-0367 Access Services Performance
\item DMS-REQ-0368 Implementation Provisions
\item DMS-REQ-0369 Evolution
\item DMS-REQ-0370 Older Release Behavior
\item OCS-DM-COM-ICD-0027 Multiple Physically Separated Copies
\item OCS-DM-COM-ICD-0028 Expected Data Volume
\item OCS-DM-COM-ICD-0029 Archive Latency
\item OCS-DM-COM-ICD-0030 EFD Transformation Service Interface
\end{itemize}
Archive NCSA Enclave \begin{itemize}
\item CA-DM-CON-ICD-0019 Camera engineering image data archiving
\item DM-TS-CON-ICD-0003 Wavefront image archive access
\item DMS-REQ-0004 Nightly Data Accessible Within 24 hrs
\item DMS-REQ-0010 Difference Exposures
\item DMS-REQ-0029 Generate Photometric Zeropoint for Visit Image
\item DMS-REQ-0030 Generate WCS for Visit Images
\item DMS-REQ-0069 Processed Visit Images
\item DMS-REQ-0077 Maintain Archive Publicly Accessible
\item DMS-REQ-0078 Catalog Export Formats
\item DMS-REQ-0089 Solar System Objects Available Within Specified Time
\item DMS-REQ-0094 Keep Historical Alert Archive
\item DMS-REQ-0102 Provide Engineering and Facility Database Archive
\item DMS-REQ-0106 Coadded Image Provenance
\item DMS-REQ-0131 Calibration Images Available Within Specified Time
\item DMS-REQ-0161 Optimization of Cost, Reliability and Availability in Order
\item DMS-REQ-0162 Pipeline Throughput
\item DMS-REQ-0163 Re-processing Capacity
\item DMS-REQ-0166 Incorporate Fault-Tolerance
\item DMS-REQ-0167 Incorporate Autonomics
\item DMS-REQ-0185 Archive Center
\item DMS-REQ-0186 Archive Center Disaster Recovery
\item DMS-REQ-0266 Exposure Catalog
\item DMS-REQ-0267 Source Catalog
\item DMS-REQ-0268 Forced-Source Catalog
\item DMS-REQ-0269 DIASource Catalog
\item DMS-REQ-0270 Faint DIASource Measurements
\item DMS-REQ-0271 DIAObject Catalog
\item DMS-REQ-0272 DIAObject Attributes
\item DMS-REQ-0273 SSObject Catalog
\item DMS-REQ-0274 Alert Content
\item DMS-REQ-0275 Object Catalog
\item DMS-REQ-0284 Level-1 Production Completeness
\item DMS-REQ-0287 DIASource Precovery
\item DMS-REQ-0291 Query Repeatability
\item DMS-REQ-0309 Raw Data Archiving Reliability
\item DMS-REQ-0310 Un-Archived Data Product Cache
\item DMS-REQ-0312 Level 1 Data Product Access
\item DMS-REQ-0314 Compute Platform Heterogeneity
\item DMS-REQ-0317 DIAForcedSource Catalog
\item DMS-REQ-0318 Data Management Unscheduled Downtime
\item DMS-REQ-0322 Special Programs Database
\item DMS-REQ-0324 Matching DIASources to Objects
\item DMS-REQ-0327 Background Model Calculation
\item DMS-REQ-0334 Persisting Data Products
\item DMS-REQ-0338 Targeted Coadds
\item DMS-REQ-0344 Constraints on Level 1 Special Program Products Generation
\item DMS-REQ-0363 Access to Previous Data Releases
\item DMS-REQ-0364 Data Access Services
\item DMS-REQ-0365 Operations Subsets
\item DMS-REQ-0366 Subsets Support
\item DMS-REQ-0367 Access Services Performance
\item DMS-REQ-0368 Implementation Provisions
\item DMS-REQ-0369 Evolution
\item DMS-REQ-0370 Older Release Behavior
\item OCS-DM-COM-ICD-0027 Multiple Physically Separated Copies
\item OCS-DM-COM-ICD-0028 Expected Data Volume
\item OCS-DM-COM-ICD-0029 Archive Latency
\item OCS-DM-COM-ICD-0030 EFD Transformation Service Interface
\end{itemize}
Commissioning Cluster Enclave \begin{itemize}
\item DMS-REQ-0161 Optimization of Cost, Reliability and Availability in Order
\item DMS-REQ-0166 Incorporate Fault-Tolerance
\item DMS-REQ-0167 Incorporate Autonomics
\item DMS-REQ-0176 Base Facility Infrastructure
\item DMS-REQ-0314 Compute Platform Heterogeneity
\item DMS-REQ-0316 Commissioning Cluster
\item DMS-REQ-0318 Data Management Unscheduled Downtime
\end{itemize}
DAC Chile Enclave \begin{itemize}
\item DMS-REQ-0004 Nightly Data Accessible Within 24 hrs
\item DMS-REQ-0075 Catalog Queries
\item DMS-REQ-0077 Maintain Archive Publicly Accessible
\item DMS-REQ-0078 Catalog Export Formats
\item DMS-REQ-0089 Solar System Objects Available Within Specified Time
\item DMS-REQ-0094 Keep Historical Alert Archive
\item DMS-REQ-0102 Provide Engineering and Facility Database Archive
\item DMS-REQ-0119 DAC resource allocation for Level 3 processing
\item DMS-REQ-0120 Level 3 Data Product Self Consistency
\item DMS-REQ-0121 Provenance for Level 3 processing at DACs
\item DMS-REQ-0123 Access to input catalogs for DAC-based Level 3 processing
\item DMS-REQ-0127 Access to input images for DAC-based Level 3 processing
\item DMS-REQ-0131 Calibration Images Available Within Specified Time
\item DMS-REQ-0161 Optimization of Cost, Reliability and Availability in Order
\item DMS-REQ-0162 Pipeline Throughput
\item DMS-REQ-0166 Incorporate Fault-Tolerance
\item DMS-REQ-0167 Incorporate Autonomics
\item DMS-REQ-0176 Base Facility Infrastructure
\item DMS-REQ-0193 Data Access Centers
\item DMS-REQ-0194 Data Access Center Simultaneous Connections
\item DMS-REQ-0196 Data Access Center Geographical Distribution
\item DMS-REQ-0284 Level-1 Production Completeness
\item DMS-REQ-0287 DIASource Precovery
\item DMS-REQ-0291 Query Repeatability
\item DMS-REQ-0309 Raw Data Archiving Reliability
\item DMS-REQ-0310 Un-Archived Data Product Cache
\item DMS-REQ-0311 Regenerate Un-archived Data Products
\item DMS-REQ-0312 Level 1 Data Product Access
\item DMS-REQ-0313 Level 1 and 2 Catalog Access
\item DMS-REQ-0314 Compute Platform Heterogeneity
\item DMS-REQ-0318 Data Management Unscheduled Downtime
\item DMS-REQ-0322 Special Programs Database
\item DMS-REQ-0323 Calculating SSObject Parameters
\item DMS-REQ-0324 Matching DIASources to Objects
\item DMS-REQ-0334 Persisting Data Products
\item DMS-REQ-0336b Regenerating Data Products from Previous Data Releases
\item DMS-REQ-0338 Targeted Coadds
\item DMS-REQ-0339 Tracking Characterization Changes Between Data Releases
\item DMS-REQ-0340 Access Controls of Level 3 Data Products
\item DMS-REQ-0341 Providing a Precovery Service
\item DMS-REQ-0344 Constraints on Level 1 Special Program Products Generation
\item DMS-REQ-0345 Logging of catalog queries
\item DMS-REQ-0363 Access to Previous Data Releases
\item DMS-REQ-0364 Data Access Services
\item DMS-REQ-0365 Operations Subsets
\item DMS-REQ-0366 Subsets Support
\item DMS-REQ-0367 Access Services Performance
\item DMS-REQ-0368 Implementation Provisions
\item DMS-REQ-0369 Evolution
\item DMS-REQ-0370 Older Release Behavior
\item OCS-DM-COM-ICD-0029 Archive Latency
\end{itemize}
DAC US Enclave \begin{itemize}
\item DMS-REQ-0004 Nightly Data Accessible Within 24 hrs
\item DMS-REQ-0075 Catalog Queries
\item DMS-REQ-0077 Maintain Archive Publicly Accessible
\item DMS-REQ-0078 Catalog Export Formats
\item DMS-REQ-0089 Solar System Objects Available Within Specified Time
\item DMS-REQ-0094 Keep Historical Alert Archive
\item DMS-REQ-0102 Provide Engineering and Facility Database Archive
\item DMS-REQ-0119 DAC resource allocation for Level 3 processing
\item DMS-REQ-0120 Level 3 Data Product Self Consistency
\item DMS-REQ-0121 Provenance for Level 3 processing at DACs
\item DMS-REQ-0123 Access to input catalogs for DAC-based Level 3 processing
\item DMS-REQ-0127 Access to input images for DAC-based Level 3 processing
\item DMS-REQ-0131 Calibration Images Available Within Specified Time
\item DMS-REQ-0161 Optimization of Cost, Reliability and Availability in Order
\item DMS-REQ-0162 Pipeline Throughput
\item DMS-REQ-0166 Incorporate Fault-Tolerance
\item DMS-REQ-0167 Incorporate Autonomics
\item DMS-REQ-0185 Archive Center
\item DMS-REQ-0186 Archive Center Disaster Recovery
\item DMS-REQ-0193 Data Access Centers
\item DMS-REQ-0194 Data Access Center Simultaneous Connections
\item DMS-REQ-0196 Data Access Center Geographical Distribution
\item DMS-REQ-0284 Level-1 Production Completeness
\item DMS-REQ-0287 DIASource Precovery
\item DMS-REQ-0291 Query Repeatability
\item DMS-REQ-0309 Raw Data Archiving Reliability
\item DMS-REQ-0310 Un-Archived Data Product Cache
\item DMS-REQ-0311 Regenerate Un-archived Data Products
\item DMS-REQ-0312 Level 1 Data Product Access
\item DMS-REQ-0313 Level 1 and 2 Catalog Access
\item DMS-REQ-0314 Compute Platform Heterogeneity
\item DMS-REQ-0318 Data Management Unscheduled Downtime
\item DMS-REQ-0322 Special Programs Database
\item DMS-REQ-0323 Calculating SSObject Parameters
\item DMS-REQ-0324 Matching DIASources to Objects
\item DMS-REQ-0334 Persisting Data Products
\item DMS-REQ-0336b Regenerating Data Products from Previous Data Releases
\item DMS-REQ-0338 Targeted Coadds
\item DMS-REQ-0339 Tracking Characterization Changes Between Data Releases
\item DMS-REQ-0340 Access Controls of Level 3 Data Products
\item DMS-REQ-0341 Providing a Precovery Service
\item DMS-REQ-0344 Constraints on Level 1 Special Program Products Generation
\item DMS-REQ-0345 Logging of catalog queries
\item DMS-REQ-0363 Access to Previous Data Releases
\item DMS-REQ-0364 Data Access Services
\item DMS-REQ-0365 Operations Subsets
\item DMS-REQ-0366 Subsets Support
\item DMS-REQ-0367 Access Services Performance
\item DMS-REQ-0368 Implementation Provisions
\item DMS-REQ-0369 Evolution
\item DMS-REQ-0370 Older Release Behavior
\item EP-DM-CON-ICD-0002 EPO is an Authorized Science User
\item EP-DM-CON-ICD-0034 Citizen Science Data
\item OCS-DM-COM-ICD-0029 Archive Latency
\end{itemize}
Offline Production Enclave \begin{itemize}
\item DM-TS-CON-ICD-0003 Wavefront image archive access
\item DMS-REQ-0004 Nightly Data Accessible Within 24 hrs
\item DMS-REQ-0008 Pipeline Availability
\item DMS-REQ-0034 Associate Sources to Objects
\item DMS-REQ-0046 Provide Photometric Redshifts of Galaxies
\item DMS-REQ-0047 Provide PSF for Coadded Images
\item DMS-REQ-0059 Bad Pixel Map
\item DMS-REQ-0060 Bias Residual Image
\item DMS-REQ-0061 Crosstalk Correction Matrix
\item DMS-REQ-0062 Illumination Correction Frame
\item DMS-REQ-0063 Monochromatic Flatfield Data Cube
\item DMS-REQ-0103 Produce Images for EPO
\item DMS-REQ-0106 Coadded Image Provenance
\item DMS-REQ-0130 Calibration Data Products
\item DMS-REQ-0131 Calibration Images Available Within Specified Time
\item DMS-REQ-0132 Calibration Image Provenance
\item DMS-REQ-0161 Optimization of Cost, Reliability and Availability in Order
\item DMS-REQ-0162 Pipeline Throughput
\item DMS-REQ-0163 Re-processing Capacity
\item DMS-REQ-0166 Incorporate Fault-Tolerance
\item DMS-REQ-0167 Incorporate Autonomics
\item DMS-REQ-0185 Archive Center
\item DMS-REQ-0186 Archive Center Disaster Recovery
\item DMS-REQ-0267 Source Catalog
\item DMS-REQ-0268 Forced-Source Catalog
\item DMS-REQ-0275 Object Catalog
\item DMS-REQ-0277 Coadd Source Catalog
\item DMS-REQ-0278 Coadd Image Method Constraints
\item DMS-REQ-0279 Deep Detection Coadds
\item DMS-REQ-0280 Template Coadds
\item DMS-REQ-0281 Multi-band Coadds
\item DMS-REQ-0282 Dark Current Correction Frame
\item DMS-REQ-0283 Fringe Correction Frame
\item DMS-REQ-0284 Level-1 Production Completeness
\item DMS-REQ-0286 SSObject Precovery
\item DMS-REQ-0287 DIASource Precovery
\item DMS-REQ-0289 Calibration Production Processing
\item DMS-REQ-0314 Compute Platform Heterogeneity
\item DMS-REQ-0318 Data Management Unscheduled Downtime
\item DMS-REQ-0320 Processing of Data From Special Programs
\item DMS-REQ-0325 Regenerating L1 Data Products During Data Release Processing
\item DMS-REQ-0329 All-Sky Visualization of Data Releases
\item DMS-REQ-0330 Best Seeing Coadds
\item DMS-REQ-0334 Persisting Data Products
\item DMS-REQ-0335 PSF-Matched Coadds
\item DMS-REQ-0341 Providing a Precovery Service
\item EP-DM-CON-ICD-0037 EPO Compute Cluster
\end{itemize}
Prompt Base Enclave \begin{itemize}
\item CA-DM-CON-ICD-0007 Provide Data Management Conditions data
\item CA-DM-CON-ICD-0008 Data Management Conditions data latency
\item CA-DM-CON-ICD-0014 Provide science sensor data
\item CA-DM-CON-ICD-0015 Provide wavefront sensor data
\item CA-DM-CON-ICD-0016 Provide guide sensor data
\item CA-DM-CON-ICD-0017 Data Management load on image data interfaces
\item CA-DM-CON-ICD-0019 Camera engineering image data archiving
\item DM-TS-CON-ICD-0002 Timing
\item DM-TS-CON-ICD-0003 Wavefront image archive access
\item DM-TS-CON-ICD-0004 Use OCS for data transport
\item DM-TS-CON-ICD-0006 Data
\item DM-TS-CON-ICD-0007 Timing
\item DM-TS-CON-ICD-0009 Calibration Data Products
\item DM-TS-CON-ICD-0011 Data Format
\item DMS-REQ-0004 Nightly Data Accessible Within 24 hrs
\item DMS-REQ-0008 Pipeline Availability
\item DMS-REQ-0018 Raw Science Image Data Acquisition
\item DMS-REQ-0020 Wavefront Sensor Data Acquisition
\item DMS-REQ-0022 Crosstalk Corrected Science Image Data Acquisition
\item DMS-REQ-0024 Raw Image Assembly
\item DMS-REQ-0068 Raw Science Image Metadata
\item DMS-REQ-0096 Generate Data Quality Report Within Specified Time
\item DMS-REQ-0097 Level 1 Data Quality Report Definition
\item DMS-REQ-0098 Generate DMS Performance Report Within Specified Time
\item DMS-REQ-0099 Level 1 Performance Report Definition
\item DMS-REQ-0100 Generate Calibration Report Within Specified Time
\item DMS-REQ-0101 Level 1 Calibration Report Definition
\item DMS-REQ-0102 Provide Engineering and Facility Database Archive
\item DMS-REQ-0161 Optimization of Cost, Reliability and Availability in Order
\item DMS-REQ-0162 Pipeline Throughput
\item DMS-REQ-0164 Temporary Storage for Communications Links
\item DMS-REQ-0165 Infrastructure Sizing for "catching up"
\item DMS-REQ-0166 Incorporate Fault-Tolerance
\item DMS-REQ-0167 Incorporate Autonomics
\item DMS-REQ-0176 Base Facility Infrastructure
\item DMS-REQ-0265 Guider Calibration Data Acquisition
\item DMS-REQ-0284 Level-1 Production Completeness
\item DMS-REQ-0309 Raw Data Archiving Reliability
\item DMS-REQ-0314 Compute Platform Heterogeneity
\item DMS-REQ-0315 DMS Communication with OCS
\item DMS-REQ-0318 Data Management Unscheduled Downtime
\item DMS-REQ-0353 Publishing predicted visit schedule
\item OCS-DM-COM-ICD-0003 Data Management CSC Command Response Model
\item OCS-DM-COM-ICD-0004 Data Management Exposed CSCs
\item OCS-DM-COM-ICD-0005 Main Camera Archiver
\item OCS-DM-COM-ICD-0006 Catch-up Archiver
\item OCS-DM-COM-ICD-0007 Prompt Processing CSC
\item OCS-DM-COM-ICD-0008 EFD Transformation Service CSC
\item OCS-DM-COM-ICD-0009 Command Set Implementation by Data Management
\item OCS-DM-COM-ICD-0017 Data Management Telemetry Interface Model
\item OCS-DM-COM-ICD-0018 Data Management Telemetry Time Stamp
\item OCS-DM-COM-ICD-0019 Data Management Events and Telemetry Required by the OCS
\item OCS-DM-COM-ICD-0020 Image and Visit Processing and Archiving Status
\item OCS-DM-COM-ICD-0021 Data Quality Metrics
\item OCS-DM-COM-ICD-0022 System Health Metrics
\item OCS-DM-COM-ICD-0025 Expected Load of Queries from DM
\item OCS-DM-COM-ICD-0026 Engineering and Facilities Database Archiving by Data Management
\item OCS-DM-COM-ICD-0027 Multiple Physically Separated Copies
\item OCS-DM-COM-ICD-0028 Expected Data Volume
\item OCS-DM-COM-ICD-0030 EFD Transformation Service Interface
\end{itemize}
Prompt NCSA Enclave \begin{itemize}
\item CA-DM-CON-ICD-0019 Camera engineering image data archiving
\item DMS-REQ-0002 Transient Alert Distribution
\item DMS-REQ-0004 Nightly Data Accessible Within 24 hrs
\item DMS-REQ-0008 Pipeline Availability
\item DMS-REQ-0010 Difference Exposures
\item DMS-REQ-0029 Generate Photometric Zeropoint for Visit Image
\item DMS-REQ-0030 Generate WCS for Visit Images
\item DMS-REQ-0069 Processed Visit Images
\item DMS-REQ-0070 Generate PSF for Visit Images
\item DMS-REQ-0072 Processed Visit Image Content
\item DMS-REQ-0074 Difference Exposure Attributes
\item DMS-REQ-0096 Generate Data Quality Report Within Specified Time
\item DMS-REQ-0097 Level 1 Data Quality Report Definition
\item DMS-REQ-0098 Generate DMS Performance Report Within Specified Time
\item DMS-REQ-0099 Level 1 Performance Report Definition
\item DMS-REQ-0100 Generate Calibration Report Within Specified Time
\item DMS-REQ-0101 Level 1 Calibration Report Definition
\item DMS-REQ-0102 Provide Engineering and Facility Database Archive
\item DMS-REQ-0131 Calibration Images Available Within Specified Time
\item DMS-REQ-0161 Optimization of Cost, Reliability and Availability in Order
\item DMS-REQ-0162 Pipeline Throughput
\item DMS-REQ-0165 Infrastructure Sizing for "catching up"
\item DMS-REQ-0166 Incorporate Fault-Tolerance
\item DMS-REQ-0167 Incorporate Autonomics
\item DMS-REQ-0185 Archive Center
\item DMS-REQ-0266 Exposure Catalog
\item DMS-REQ-0269 DIASource Catalog
\item DMS-REQ-0270 Faint DIASource Measurements
\item DMS-REQ-0271 DIAObject Catalog
\item DMS-REQ-0272 DIAObject Attributes
\item DMS-REQ-0273 SSObject Catalog
\item DMS-REQ-0274 Alert Content
\item DMS-REQ-0284 Level-1 Production Completeness
\item DMS-REQ-0309 Raw Data Archiving Reliability
\item DMS-REQ-0314 Compute Platform Heterogeneity
\item DMS-REQ-0317 DIAForcedSource Catalog
\item DMS-REQ-0318 Data Management Unscheduled Downtime
\item DMS-REQ-0319 Characterizing Variability
\item DMS-REQ-0320 Processing of Data From Special Programs
\item DMS-REQ-0321 Level 1 Processing of Special Programs Data
\item DMS-REQ-0327 Background Model Calculation
\item DMS-REQ-0328 Documenting Image Characterization
\item DMS-REQ-0343 Performance Requirements for LSST Alert Filtering Service
\item DMS-REQ-0344 Constraints on Level 1 Special Program Products Generation
\end{itemize}


\input{refs}
\end{document}
